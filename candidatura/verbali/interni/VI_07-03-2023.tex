\documentclass[a4paper,12pt]{article}
\usepackage{graphicx}
\usepackage[utf8]{inputenc}
\usepackage[T1]{fontenc}
\usepackage[italian]{babel}
\usepackage{geometry}
\usepackage{fancyhdr}
\usepackage{hyperref}
\usepackage{makecell}

\geometry{left=2.5cm,right=2.5cm}%prima i left e right erano 3cm.
\pagestyle{fancy}
\setlength\headheight{63pt}

\fancyhead[L]{\includegraphics[scale=0.08]{Project_Origin_full_logo.png}}
\fancyfoot[L]{Verbale del 07-03-2023}

\renewcommand{\headrulewidth}{0.5pt}
\renewcommand{\footrulewidth}{0.5pt}

\begin{document}
%%FRONT PAGE
\newcommand{\makefrontpage}{
	\begin{titlepage}
		\begin{center}
		\includegraphics[width=0.4\textwidth]{Project_Origin_full_logo.png}\\

		\vspace{1.5cm}
		{\LARGE \textbf{Verbale di Riunione del 07-03-2023}}
		\\\vspace{0.5cm}
		\textbf{Project Origin - Personal Identity Wallet}
		\\\vspace{0.2cm}		
		\href{mailto://projectorigin2023@gmail.com}{projectorigin2023@gmail.com}
		\\\vspace{0.5cm}
		\large{Verbale ad uso Interno}
		\vfill
		
		
		\vspace{2cm}
		%\vspace{2cm}

		%\begin{flushright}
		%Firma del responsabile: \underline{\hspace{5cm}} \
		%\end{flushright}
		
		\end{center}
	\end{titlepage}
}
\makefrontpage
%%FINE FRONT PAGE
\section*{Informazioni generali}
\begin{tabular}{ll}%sistemare i ritorni a capo
\textbf{Data:}  07/03/2023 \\
\textbf{Luogo:}  Stanza virtuale Discord \\
\textbf{Ora di inizio:}  14:30 \\
\textbf{Presenze:} \\
					\quad - Andrei Cristian Bobirica\\
					\quad - Alessio Andreetto\\
					\quad - Elton Ibra\\
					\quad - Michele Beschin\\
					\quad - Teodor Mihail Corbu\\
					\quad - Riccardo Lotto
\end{tabular}

\section*{Ordine del giorno}
\begin{enumerate}
\item Conoscenza dei membri del gruppo
\item Decisione nome, logo ed email del gruppo
\item Discussione capitolati %vediamo se lasciarlo come cosa del primo giorno o metterla solo dopo che il prof ha mandato la mail del 10-03
\end{enumerate}

\section*{Discussione}
\subsection*{Conoscenza dei membri del gruppo}
Dopo aver creato il gruppo telegram per le comunicazioni, si è deciso di iniziare una riunione online ad un orario che che andasse bene a tutti. È seguito poi un confronto per stabilire il tempo da dedicare al progetto, sulla base degli impegni di ciascuno dei membri del gruppo, per capire come organizzare il lavoro.
\section*{Decisione nome, logo ed email del gruppo}
Ogni membro del gruppo ha proposto un possibile nome per il gruppo, e alla fine si è deciso di scegliere con un sondaggio sul gruppo telegram per stabilire il nome. Successivamente verrà creato un recapito di posta elettronica, e analogamente verrà scelto un logo.
\section*{Discussione capitolati}
Durante la riunione emerge un forte interesse per i capitolati C2 e C3, tuttavia si decide di non prendere decisioni, in quanto il professore fornirà aggiornamenti sulla disponibilità dei singoli capitolati.

\end{document}