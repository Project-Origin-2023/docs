\section{Strumenti Necessari}
Per poter avviare il progetto occorre avere installato ed eseguire i seguenti passi:
%\begin{itemize}
%   \item installare Brew (versione minima 4.1.11);
%   \item tramite Brew si installa npm (versione minima 9.8.1);
%   \item sempre tramite Brew si installa Nodejs (versione minima v20.6.1)
%   \item installare Docker
%\end{itemize}
\subsection{Requisiti minimi}
Per poter avviare il progetto occorre avere, sulla propria macchina, queste specifiche minime:
%\begin{itemize}
%   \item microprocessore i5;
%   \item 8GB di memoria RAM;
%   \item 100GB memoria SSD (consigliato, visto che non si sa con esattezza quanto occupa OpenId)
%   \item sistema operativo Ubuntu, Linux (versione 22), oppure MacOS versione I5 (potrebbe dare problemi con la versione M1)
%\end{itemize}
Le specifiche minime cosnigliate sono:
%\begin{itemize}
%   \item microprocessore i7;
%   \item 12GB di memoria RAM;
%   \item 100GB memoria SSD (consigliato, visto che non si sa con esattezza quanto occupa OpenId)
%   \item sistema operativo Ubuntu, Linux (versione 22), oppure MacOS versione I5 (potrebbe dare problemi con la versione M1)
%\end{itemize}
\subsection{Script di avvio}

Specifiche minime:12 gb ram, i7 microprocessore, 100 gb ssd, non sappiamo quanto occupa openid, sistema operativo ubuntu, oppure linux, oppure MacOS, versione di linux 12.4.5 (penso), MAC os versione i5 (non m1 per non dare problemi)
Pacchetti da preinstallare: npm, brew, tramite brew si installa npm (versione npm 9.8.1), nodejs versione minima v20.6.1, Brew verisone Homebrew 4.1.11, tramite brew si installa npm e nodejs
    docker: versione 24.0.6 (o 20.10.24), docker compose versione 1.29.2-1
Altre cose: sapere i permessi di amministratore del proprio computer
Credenziali di accesso necessarie per il progetto:
    database per origin wallet: dbuser:admin, dbhost:10.5.0.33, dbdatabase:originwallet, dbpassword: admin, dbport:5432
    database per origin issuer: dbuser:admin, dbhost:10.5.0.31, dbdatabase:originissuer, dbpassword: admin, dbport:5432
    njinx proxy manager: admin@admin.com, password adminadmin
  %%  degli utenti: dentro issuer c'èun utente amministratore (sys\_admin dell'issuerApp(vedere riga sotto)
        dentro wallet, utente normale: mario.rossi@gmail.com, password Mariorossi123! (già preesistente nel database)
        LASCIAR STARE L'AUTENTICAZIONE DI NJINX
Utente admin issuer:
    email: ADMIN@admin.com
    password: ADmin1234?
Script di avvio:
%- clone del repo da github (git clone)->(da Personal Identity Wallet)->cd scripts-> sh deploy.sh-> fa partire tutto il processo di deployment. L'utente deve esapettare fin quando non gli viene richiesta la password. Una volta immessa la pssword deve aspettare finchè non si completa il processo di popolamento del database. Si deve aspettare qualche minuto
%- avviare docker
%- fare sh script/deploy.sh