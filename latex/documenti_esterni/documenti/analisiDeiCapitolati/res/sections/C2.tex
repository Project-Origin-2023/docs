\section*{C2 - Lumos Minima}
\begin{itemize}
    \item \textbf{Proponente}: Imola Informatica
    \item \textbf{Obiettivo}: Creare un sistema per l’ottimizzazione dell’illuminazione pubblica
\end{itemize}

\subsection*{Descrizione}
Con questo progetto proponiamo lo sviluppo di un sistema per l’ottimizzazione dell’illuminazione pubblica che permetta ai gestori di sfruttare la possibilità di regolare l’intensità
della luce emessa dagli impianti d’illuminazione. Un sistema così congegnato consentirebbe, da un lato, di garantire sicurezza stradale e sociale, e dall’altro permetterebbe di
risparmiare energia, dunque, risorse economiche e ambientali.

\subsection*{Obiettivo}
L’obiettivo è quello di sviluppare un’applicazione web responsive tramite la quale il gestore
di un sistema di illuminazione pubblico possa eseguire le seguenti azioni:
\begin{itemize}
    \item Rilevamento della presenza di persone in prossimità della fonte luminosa;
    \item Aumento/riduzione dell’intensità luminosa (in modalità manuale o automatica);
    \item Rilevamento (automatico o segnalato manualmente) del guasto di un impianto d'illuminazione;
    \item  Inserimento e gestione di un impianto luminoso.
\end{itemize}

\subsection*{Tecnologie richieste}
Nel capitolato non sono imposti vincoli riguardo le tecnologie richieste, tranne per il fatto
che si deve realizzare un’applicazione web responsive.

\subsection*{Note positive}
\begin{itemize}
    \item Utilizzo di sensori iot
    \item Libertà nelle tecnologie utilizzate
\end{itemize}
\subsection*{Criticità}
\begin{itemize}
    \item Difficoltà nella gestione della comunicazione tra i sensori
    \item Utilità per l’utente finale
\end{itemize}