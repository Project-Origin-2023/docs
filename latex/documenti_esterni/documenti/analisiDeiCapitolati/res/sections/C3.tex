\section*{C3 - Personal Identity Wallet}
\begin{itemize}
    \item \textbf{Proponente}: Infocert
    \item \textbf{Obiettivo}: Realizzazione di un sistema d'identità digitale
\end{itemize}

 \subsection*{Descrizione}
 Tutti i giorni utilizziamo delle identità digitali per accedere a servizi online (per esempio
 accedi con Google). Tuttavia questo non basta in quei contesti in cui è richiesto valore
 legale (per esempio per accesso a dati sanitari o servizi bancari). In molti casi sono stati
 creati sistemi di autenticazione ad-hoc, ma sono limitati a singoli servizi.
 Per questo motivo diversi stati hanno realizzato sistemi d'identità digitale (in Italia SPID
 e CIE). L’Unione Europea ha recentemente emanato un regolamento per la realizzazione
 di un sistema d'identità digitale utilizzabile da tutti i cittadini europei in tutti i paesi
 dell’Unione Europea.
 
 \subsection*{Obiettivo}
 Il capitolato prevede la realizzazione di un sistema di autenticazione dove un ente rilascia
certificati d'identità ad un utente, che le memorizza in un "wallet", e le può utilizzare
per accedere a servizi. Sono previsti tre attori:
\begin{itemize}
    \item \textbf{Emittente}: Entità che rilascia certificati
    \item \textbf{Holder}: Persona fisica, che memorizza credenziali d'identità all’interno di un wallet
    \item \textbf{Verifier}: Entità che richiede delle credenziali per accedere a dei servizi
\end{itemize}

\subsection*{Tecnologie richieste}

È necessario realizzare le seguenti componenti architetturali:
\begin{itemize}
    \item Componente back-office (web app) per consentire all’issuer di rilasciare certificati d'identità;
    \item Demo di interazione utente (web app) per consentire all’utente di navigare e richiedere le credenziali da un issuer, e di utilizzarle per accedere ad un servizio;
    \item App front-end per l’utente (web app) dove l’utente memorizza le credenziali;
    \item Componente di comunicazione per consentire lo scambio di credenziali secondo un protocollo standard.
\end{itemize}
Per la web app non ci sono vincoli imposti, si può scegliere di realizzare un unico applicativo oppure una webapp per ogni componente.
Le credenziali devono rispettare il formato JSON W3C e devono essere scambiate con il
protocollo OpenID4VP.

\subsection*{Note positive}
\begin{itemize}
    \item Attualità del tema proposto
    \item Libertà nelle tecnologie utilizzate
\end{itemize}
\subsection*{Criticità}
\begin{itemize}
    \item Complessità dello standard europeo di riferimento
\end{itemize}

