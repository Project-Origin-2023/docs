\section{Introduzione}

\subsection{Scopo del documento}
Il documento si prefigge di esporre e analizzare tutti i requisiti espliciti e impliciti per la realizzazione del progetto Personal Identity Wallet\glo, proposto dall'azienda Infocert. 
Il documento costituirà una base di partenza fondamentale per la fase di progettazione del software, in modo da garantire che essa sia conforme alle richieste fatte dall'azienda 
proponente Infocert.


\subsection{Scopo del prodotto}
Lo scopo del prodotto è quello di creare una versione semplificata di un applicativo per implementare e rilasciare un "portafoglio di identità digitale" conforme a un insieme di 
standard, in modo che possa essere utilizzato con qualsiasi servizio conforme in qualsiasi paese dell'UE. \\
In particolare, si dovrà realizzare una web app\glo avendo queste componenti architetturali:
\begin{itemize}
    \item Un componente back-office per consentire al dipendente dell'organizzazione emittente di verificare\glo manualmente la richiesta di credenziali e autorizzarne l'emissione; 
    \item Un componente di interazione con l'utente dimostrativo per consentire all'utente (titolare) di navigare e richiedere specifiche credenziali da un emittente 
(ad esempio, il sito di una demo universitaria); 
    \item Un componente di interazione con l'utente dimostrativo per consentire all'utente (titolare) di navigare un sito verificatore\glo e fornire le credenziali richieste;
    \item Un'app front-end per l'utente per archiviare e gestire le proprie credenziali; 
    \item Un componente di comunicazione per consentire lo scambio di credenziali/presentazioni secondo un protocollo standard - il componente di comunicazione sarà implementato 
tre volte nei tre contesti (lato emittente, lato titolare, lato verificatore).
\end{itemize}

\subsection{Note Esplicative}
Alcuni termini utilizzati nel documento possono avere significati ambigui a seconda del contesto. Al fine di evitare equivoci, è stato creato un Glossario contenente tali termini 
e il loro significato specifico. Per segnalare che un termine è presente nel Glossario, sarà aggiunta una "g" a pedice accanto al termine corrispondente nel testo.

\subsection{Riferimenti}
1. Normativi: 
\begin{itemize}
    \item \textbf{Norme di progetto}: contengono le norme e gli strumenti per gli analisti;
    \item \textbf{Capitolato d’appalto C3}: \url{https://www.math.unipd.it/~tullio/IS-1/2022/Progetto/C3.pdf};
    \item \textbf{VE-2023-03-02}: verbale esterno. Primo incontro con Infocert.
\end{itemize}
2. Informativi: 
\begin{itemize}
    \item \textbf{Glossario 0.0.2};
    \item \textbf{Slide del corso di Ingegneria del Software – Analisi dei Requisiti:}: \url{https://www.math.unipd.it/~tullio/IS-1/2022/Dispense/T06.pdf};
    \item \textbf{Slide del corso di Ingegneria del Software – Diagrammi dei Casi d’Uso}: \url{https://www.math.unipd.it/~rcardin/swea/2022/Diagrammi%20Use%20Case.pdf}.
\end{itemize}