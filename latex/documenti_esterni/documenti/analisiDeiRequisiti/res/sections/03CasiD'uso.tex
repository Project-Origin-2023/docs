\section{Casi d'Uso}
\subsection{Introduzione}
In questa sezione sono presentati i casi d'uso che risultano rilevanti per il prodotto Personal Identity Wallet. 
Essi sono stati individuati e definiti attraverso l'analisi del capitolato d'appalto, gli incontri con il proponente e le riunioni interne del team Project Origin.
 Ciascun caso d'uso rappresenta un insieme di scenari che hanno lo stesso obiettivo finale per un utente generico del sistema, definito attore.
 Le norme e le convenzioni adottate per la stesura di ogni caso d'uso sono descritte in dettaglio all'interno del documento Norme di Progetto.
 \subsection{Codice identificativo}
Ciascun caso d'uso viene categorizzato utilizzando la seguente notazione:\\ 
\begin{center}\begin{verbatim}
    CU{XX}.{YY}
\end{verbatim}\end{center}
Ogni caso d’uso è inoltre definito secondo la seguente struttura:
\begin{itemize}
    \item ID: il codice del caso d’uso secondo la convenzione specificata precedentemente;
    \item Nome:  specifica il titolo del caso d’uso;
    \item Attori:  indica gli attori principali (ad esempio l’utente generico) e secondari (ad esempio entità di autenticazione esterne) del caso d’uso;
    \item Descrizione:  riporta una breve descrizione del caso d’uso;
    \item Precondizione:  specifica le condizioni che sono identificate come vere prima del verificarsi degli eventi del caso d’uso;
    \item Postcondizione:  specifica  le  condizioni  che  sono  identificate  come  vere  dopo  il verificarsi degli eventi del caso d’uso;
    \item Scenario  principale:  rappresenta  il  flusso  degli  eventi,  a  volte  attraverso  l'uso di  una  lista  numerata,  specificando  per  ciascun  evento:  titolo,  descrizione,  attori coinvolti e casi d’uso generati;
    \item Inclusioni:  usate per non descrivere più volte lo stesso flusso di eventi, inserendo il comportamento comune in un caso d’uso a parte;
    \item Estensioni:  descrivono i casi d’uso che non fanno parte del flusso principale degli eventi, allo stesso modo di quanto descritto in “Scenario principale”.
\end{itemize}
Alcuni  casi  d’uso  possono  essere  associati  ad  un Diagramma UML  dei  casi  d'uso riportante lo stesso titolo e codice.
\subsection{Attori}
\textbf{Attori principali:}
\begin{itemize}
    \item\textbf{Utente generico:} si riferisce all’utente che non ha ancora eseguito il login al sistema;
    \item\textbf{Utente autenticato:} si riferisce all’utente che ha effettuato il login al sistema;
    \item\textbf{Holder;}
    \item\textbf{Issuer;}
    \item\textbf{Verifier.}
\end{itemize}


\subsection{Elenco dei casi d'uso}
\subsubsection{UC01 - Richiesta di credenziali per Wallet}
\begin{itemize}
\item \textbf{Attori:} Holder, Issuer;
\item \textbf{Descrizione:} Un utente (Holder) deve essere in grado di navigare sul sito dell'emittente (Issuer) e presentare una richiesta per avere una credenziale;
\item \textbf{Pre-condizione:} L’utente (Holder) deve richiedere delle credenziali;
\item \textbf{Post-condizione:} L’utente (Holder) è riuscito a presentare la richiesta per le credenziali nel sito dell’ Issuer ed ora la sua richiesta deve essere esaminata.
\end{itemize}

\subsubsection{UC02 - Fornitura delle credenziali per Wallet}
\begin{itemize}
\item \textbf{Attori:} Issuer, Holder;
\item \textbf{Descrizione:} Un utente amministratore back-office (Issuer) deve essere in grado di approvare o rifiutare l'emissione di una credenziale per un utente (Holder);
\item \textbf{Pre-condizione:} L’Issuer deve esaminare ed approvare la richiesta del Holder;
\item \textbf{Post-condizione:} L’Issuer ha approvato la richiesta del Holder e ha fornito la credenziale. Ora l'Holder ha la possibilità di ottenere la credenziale sul proprio Wallet.
\end{itemize}

\subsubsection{UC03 - Ottenimento delle credenziali per Wallet}
\begin{itemize}
\item \textbf{Attori:} Holder, Issuer;
\item \textbf{Descrizione:} Un utente (Holder) deve essere in grado di ottenere una credenziale sul proprio Wallet;
\item\textbf{Pre-condizione:} La credenziale è stata approvata e fornita dal Issuer ma l'Holder non è in possesso delle credenziali sul proprio Wallet;
\item \textbf{Post-condizione:} l'Holder possiede all’interno del proprio Wallet la credenziale.
\end{itemize}

\subsubsection{UC04 - Visualizzazione delle credenziali}
\begin{itemize}
\item \textbf{Attori:} Holder;
\item \textbf{Descrizione:} Un utente (Holder) deve essere in grado di visualizzare l'insieme delle proprie credenziali ricevute sul suo Wallet;
\item \textbf{Pre-condizione:} L’utente (Holder) non ha ancora visualizzato le credenziali;
\item \textbf{Post-condizione:} L’utente (Holder) è riuscito a visualizzare le credenziali.
\end{itemize}

\subsubsection{UC05 - Cancellazione delle credenziali}
\begin{itemize}
\item \textbf{Attori:} Holder;
\item \textbf{Descrizione:} Un utente (Holder) deve essere in grado di rimuovere una credenziale dal proprio Wallet;
\item \textbf{Pre-condizione:} L’utente (Holder) vuole rimuovere una credenziale dal Wallet personale;
\item \textbf{Post-condizione:} L’utente (Holder) è riuscito a rimuovere con successo la credenziale dal Wallet personale.
\end{itemize}

\subsubsection{UC06 - Richiesta credenziali per presentazione}
\begin{itemize}
\item \textbf{Attori:} Verifier, Holder;
\item \textbf{Descrizione:} Un Verifier deve essere in grado di richiedere a un utente (Holder) che sta navigando sul suo sito web di fornire una credenziale (presentazione) che è memorizzata nel Wallet del Holder;
\item \textbf{Pre-condizione:} Il Verifier vuole richiedere una credenziale al utente (Holder);
\item \textbf{Post-condizione:} Il Verifier ha presentato una richiesta di una credenziale al utente (Holder) ed ora l'Holder può fornire la propria credenziale.
\end{itemize}

\subsubsection{UC07 - Fornitura delle credenziali per presentazione}
\begin{itemize}
\item \textbf{Attori:} Holder, Verifier;
\item \textbf{Descrizione:} Un utente (Holder) deve essere in grado di fornire al Verifier la credenziale (presentazione) richiesta;
\item \textbf{Pre-condizione:} L’utente vuole fornire la credenziale presente nel Wallet personale al Verifier;
\item \textbf{Post-condizione:} La credenziale dell’utente è stata fornita al Verifier.
\end{itemize}

