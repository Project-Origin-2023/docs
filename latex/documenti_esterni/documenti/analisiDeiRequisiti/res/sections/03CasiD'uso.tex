\section{Casi d'Uso}
\subsection{Introduzione}
In questa sezione sono presentati i casi d'uso che risultano rilevanti per il prodotto Personal Identity Wallet. 
Essi sono stati individuati e definiti attraverso l'analisi del capitolato d'appalto, gli incontri con il proponente e le riunioni interne del team Project Origin.
 Ciascun caso d'uso rappresenta un insieme di scenari che hanno lo stesso obiettivo finale per un utente generico del sistema, definito attore.
 Le norme e le convenzioni adottate per la stesura di ogni caso d'uso sono descritte in dettaglio all'interno del documento Norme di Progetto.
 \subsection{Codice identificativo}
Ciascun caso d'uso viene categorizzato utilizzando la seguente notazione:\\ 
\begin{center}\begin{verbatim}
    CU{XX}.{YY}
\end{verbatim}\end{center}
Ogni caso d’uso è inoltre definito secondo la seguente struttura:
\begin{itemize}
    \item ID: il codice del caso d’uso secondo la convenzione specificata precedentemente;
    \item Nome:  specifica il titolo del caso d’uso;
    \item Attori:  indica gli attori principali (ad esempio l’utente generico) e secondari (ad esempio entità di autenticazione esterne) del caso d’uso;
    \item Descrizione:  riporta una breve descrizione del caso d’uso;
    \item Precondizione:  specifica le condizioni che sono identificate come vere prima del verificarsi degli eventi del caso d’uso;
    \item Postcondizione:  specifica  le  condizioni  che  sono  identificate  come  vere  dopo  il verificarsi degli eventi del caso d’uso;
    \item Scenario  principale:  rappresenta  il  flusso  degli  eventi,  a  volte  attraverso  l'uso di  una  lista  numerata,  specificando  per  ciascun  evento:  titolo,  descrizione,  attori coinvolti e casi d’uso generati;
    \item Inclusioni:  usate per non descrivere più volte lo stesso flusso di eventi, inserendo il comportamento comune in un caso d’uso a parte;
    \item Estensioni:  descrivono i casi d’uso che non fanno parte del flusso principale degli eventi, allo stesso modo di quanto descritto in “Scenario principale”.
\end{itemize}
Alcuni  casi  d’uso  possono  essere  associati  ad  un Diagramma UML  dei  casi  d'uso riportante lo stesso titolo e codice.
\subsection{Attori}
\textbf{Attori principali:}
\begin{itemize}
    \item\textbf{Utente generico:} si riferisce all’utente che non ha ancora eseguito il login al sistema;
    \item\textbf{Utente autenticato:} si riferisce all’utente che ha effettuato il login al sistema;
    \item\textbf{Issuer;}
    \item\textbf{Verifier.}
\end{itemize}


\subsection{Elenco dei casi d'uso}
\subsubsection{UC01 - Richiesta di credenziali (documenti personali e certificati)}
\begin{itemize}
\item \textbf{Attori:} utente generico;
\item \textbf{Descrizione:} un utente (Holder) deve essere in grado di navigare sul sito dell'emittente e richiedere una credenziale per il suo portafoglio digitale;
\item \textbf{Pre-condizione:} l’utente deve richiedere delle credenziali;
\item \textbf{Post-condizione:} l’utente è riuscito a richiedere le credenziali nel sito dell’ Issuer.
\end{itemize}

\subsubsection{UC02 - Fornitura delle credenziali}
\begin{itemize}
\item \textbf{Attori:} utente generico, Issuer;
\item \textbf{Descrizione:} un utente back-office (lato Issuer) deve essere in grado di emettere una credenziale per un Holder;
\item \textbf{Pre-condizione:} l’Issuer deve emettere all’utente generico richiedente la credenziale;
\item \textbf{Post-condizione:} l’Issuer ha fornito la credenziale all'utente generico richiedente. Il richiedente è in possesso della credenziale.
\end{itemize}

\subsubsection{UC03 - Ottenimento delle credenziali}
\begin{itemize}
\item \textbf{Attori:} utente generico;
\item \textbf{Descrizione:} un utente (Holder) deve essere in grado di ottenere una credenziale dal sito web dell'Issuer;
\item\textbf{Pre-condizione:} l’utente non è in possesso delle credenziali;
\item \textbf{Post-condizione:} l’utente possiede all’interno del wallet le credenziali.
\end{itemize}

\subsubsection{UC04 - Visualizzazione delle credenziali}
\begin{itemize}
\item \textbf{Attori:} utente autenticato;
\item \textbf{Descrizione:} un utente (Holder) deve essere in grado di visualizzare l'insieme delle credenziali ricevute tramite l'applicazione web;
\item \textbf{Pre-condizione:} l’utente non ha ancora visualizzato le credenziali;
\item \textbf{Post-condizione:} l’utente è riuscito a visualizzare le credenziali.
\end{itemize}

\subsubsection{UC05 - Cancellazione delle credenziali}
\begin{itemize}
\item \textbf{Attori:} utente autenticato;
\item \textbf{Descrizione:} un utente (Holder) deve essere in grado di rimuovere una credenziale dall'applicazione web;
\item \textbf{Pre-condizione:} l’utente vuole rimuovere una credenziale dal wallet personale;
\item \textbf{Post-condizione:} l’utente è riuscito a rimuovere con successo la credenziale dal wallet personale.
\end{itemize}

\subsubsection{UC06 - Richiesta credenziali (presentazione)}
\begin{itemize}
\item \textbf{Attori:} Verifier, utente generico;
\item \textbf{Descrizione:} un Verifier deve essere in grado di richiedere a un utente che sta navigando sul suo sito web di fornire una credenziale (presentazione) che è memorizzata nel portafoglio dell'utente;
\item \textbf{Pre-condizione:} il Verifier vuole richiedere le credenziali dell’utente;
\item \textbf{Post-condizione:} il Verifier è riuscito ad ottenere le credenziali dell’utente.
\end{itemize}

\subsubsection{UC07 - Fornitura delle credenziali (presentazione)}
\begin{itemize}
\item \textbf{Attori:} utente autenticato, Verifier;
\item \textbf{Descrizione:} un utente (Holder) deve essere in grado di fornire al Verifier la credenziale (presentazione) richiesta;
\item \textbf{Pre-condizione:} l’utente vuole fornire la credenziale presente nel wallet personale ad un'entità Verifier;
\item \textbf{Post-condizione:} la credenziale dell’utente è stata fornita al Verifier.
\end{itemize}

\subsubsection{UC08 - Validazione credenziali (facoltativo)}
\begin{itemize}
\item \textbf{Attori:} Verifier;
\item \textbf{Descrizione:} un Verifier deve essere in grado di validare la correttezza della credenziale (presentazione) ricevuta;
\item \textbf{Pre-condizione:} il Verifier vuole validare la correttezza della credenziale dell’utente;
\item \textbf{Post-condizione:} il Verifier è riuscito a validare la correttezza della credenziale dell’utente.
\end{itemize}