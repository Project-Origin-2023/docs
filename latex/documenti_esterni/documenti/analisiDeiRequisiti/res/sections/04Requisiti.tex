\section{Requisiti}
\subsection{Introduzione}
In questa sezione sono elencati i casi d'uso rilevanti per la realizzazione del prodotto Personal Identity Wallet.
\subsection{Elenco dei requisiti}
\subsubsection*{Requisiti funzionali} %al posto di subsubsection si possono mettere dei textbf centrati, come le tabelle
    \begin{longtable}{|c|c|c|}
        \hline
        \textbf{Codice} & \textbf{Descrizione} & \textbf{Riferimento} \\
        \hline
        %requisiti funzionali
        RF01-O & L'utente richiede le credenziali& UC01\\
        RF02-O & L'utente ottiene le credenziali& UC03\\
        RF03-O & L'utente memorizza le credenziali nel wallet\glo{}& UC03\\
        RF04-O & L'utente consuma le credenziali presso un ente & UC07\\
        RF05-0 & L'issuer\glo{} fornisce credenziali all'utente & UC02\\
        RF06-O & Un ente chiede all'utente di fornire delle credenziali& UC06\\
        RF07-O & L'utente visualizza le credenziali memorizzate nel wallet & UC04\\
        RF08-O & L'utente elimina le credenziali memorizzate nel wallet & UC05\\
        \hline
    \end{longtable}
\subsubsection*{Requisiti non funzionali}
    \begin{longtable}{|c|c|}
        \hline
        \textbf{Codice} & \textbf{Descrizione} \\
        \hline
        %requisiti non funzionali
        RN01-O & Le credenziali devono rispettare il formato JSON W3C\\
        RN02-O & Le credenziali devono essere scambiate con il protocollo OpenID4VC tra issuer e holder\\
        RN03-O & Le credenziali devono essere scambiate con il protocollo OpenID4VP tra holder e verifier\\
        %requisito sull'usabilità?
        RN04-O & Le credenziali possono essere consumate solo dall'effettivo holder\\
        RN05-O & Le credenziali non possono essere modificate\\
        RN06-O & Le credenziali non possono essere intercettate\\
        \hline
    \end{longtable}