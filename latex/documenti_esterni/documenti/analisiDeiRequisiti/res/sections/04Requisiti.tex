\section{Requisiti}
\subsection{Introduzione}
In questa sezione sono elencati i casi d'uso rilevanti per la realizzazione del prodotto Personal Identity Wallet.
\subsection{Elenco dei requisiti}
\subsubsection*{Requisiti funzionali} %al posto di subsubsection si possono mettere dei textbf centrati, come le tabelle
    \begin{tabular}{|p{0.08\textwidth}|p{0.7\textwidth}|p{0.13\textwidth}|}
        %requisiti funzionali
        \hline
        \textbf{Codice} & \textbf{Descrizione} & \textbf{Riferimento} \\
        \hline
        RF01-O & L'utente inserisce le credenziali nel portale del wallet per iscriversi & UC1\\
        RF02-O & L'utente visualizza un messaggio di errore per dati immessi non corretti, non risulta registrato al wallet & UC4\\
        RF03-O & L'utente può accedere al portale wallet attraverso le credenziali di accesso & UC2\\
        RF04-O & L'utente visualizza un messaggio di errore per credenziali sbagliate al login & UC4\\ 
        RF05-O & L'utente può eseguire il logout dal portale wallet & UC3\\
        RF06-O & L'utente richiede una credenziale PID identificativa nella piattaforma dell'issuer & UC5.1\\
        RF07-O & L'utente richiede una credenziale EAA nella piattaforma dell'issuer & UC5.2\\
        RF08-O & L'issuer rilascia la credenziale al wallet & UC6\\
        RF09-O & L'holder può verificare nel proprio wallet lo stato della richiesta credenziale fatta all'issuer & UC7\\
        RF10-O & L'utente ottiene correttamente la credenziale nel propio wallet & UC8\\
        RF11-O & L'utente visualizza un errore sul wallet che notifica l'errore di rilascio della credenziale & UC9\\
        RF12-O & L'utente visualizza una lista di credenziali memorizzate all'interno del proprio wallet& UC10\\
        RF13-O & L'utente all'interno della propria piattaforma wallet visualizza la credenziale identificativa PID & UC11.1\\
        RF14-O & L'utente all'interno della propria piattaforma wallet visualizza la credenziale identificativa EAA & UC11.2\\
        RF15-O & L'utente elimina le credenziali memorizzate nel wallet & UC12\\
        RF16-O & Il verifier richiede all'utente una credenziale presente sul wallet personale da verificare & UC13\\
        RF17-O & L'utente fornisce tramite il proprio wallet una credenziale al verifier da verificare & UC14\\
        RF18-O & L'utente riesce a visualizzare un messaggio di errore nella propria piattaforma wallet che notifica l'errore di verifica & UC15\\
        \hline
    \end{tabular}

\clearpage
\subsubsection*{Requisiti non funzionali}
    \begin{longtable}{|c|c|}
        \hline
        \textbf{Codice} & \textbf{Descrizione} \\
        \hline
        %requisiti non funzionali
        RN01-O & Le credenziali devono rispettare il formato JSON W3C\\
        RN02-O & Le credenziali devono essere scambiate con il protocollo OpenID4VC tra issuer e holder\\
        RN03-O & Le credenziali devono essere scambiate con il protocollo OpenID4VP tra holder e verifier\\
        %requisito sull'usabilità?
        RN04-O & Le credenziali possono essere consumate solo dall'effettivo holder\\
        RN05-O & Le credenziali non possono essere modificate\\
        RN06-O & Le credenziali non possono essere intercettate\\
        \hline
    \end{longtable}