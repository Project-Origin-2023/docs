\section{Requisiti}
\subsection{Introduzione}
In questa sezione sono elencati i casi d'uso rilevanti per la realizzazione del prodotto Personal Identity Wallet.
\subsection{Elenco dei requisiti}
\subsubsection*{Requisiti funzionali} %al posto di subsubsection si possono mettere dei textbf centrati, come le tabelle
    \begin{longtable}{|p{0.08\textwidth}|p{0.7\textwidth}|p{0.13\textwidth}|}
        %requisiti funzionali
        \hline
        \textbf{Codice} & \textbf{Descrizione} & \textbf{Riferimento} \\
        \hline
        RF01-O & L'utente inserisce le credenziali nel portale del wallet per iscriversi & UC1\\
        RF02-O & L'utente visualizza un messaggio di errore per dati immessi non corretti, non risulta registrato al wallet & UC4\\
        RF03-O & L'utente può accedere al portale wallet attraverso le credenziali di accesso & UC2\\
        RF04-O & L'utente visualizza un messaggio di errore per credenziali sbagliate al login & UC4\\ 
        RF05-O & L'utente può eseguire il logout dal portale wallet & UC3\\
        RF06-O & L'utente inserisce le credenziali nel portale Issuer sistema per poter iscriversi & UC5\\
        RF07-O & L'utente visualizza un messaggio di errore durante la registrazione nel portale Issuer sistema per dati immessi non corretti. & UC8\\
        RF08-O & L'utente visualizza un messaggio di errore durante il login nel portale Issuer sistema per dati immessi non corretti. & UC8\\
        RF09-O & L'utente può accedere attraverso le credenziali al portale dell'Issuer sistema & UC6\\
        RF10-O & L'utente può eseguire il logout dal portale dell'Issuer sistema & UC7\\
        RF11-O & L'utente richiede una credenziale PID identificativa nella portale dell'Issuer sistema & UC9.1\\
        RF12-O & L'utente richiede una credenziale EAA nel portale dell'Issuer sistema & UC9.2\\
        RF13-O & L'issuer admin effettua il lgin con le proprie credenziali speciali al portale Issuer sistema & UC6\\ 
        RF14-O & L'issuer admin accede alla propria dashboard amministrativa & UC6\\
        RF15-O & L'issuer admin esamina le richiesta di credenziale nel portale issuer sistema  & UC10\\
        RF16-O & L'issuer admin approva o rifiuta la richiesta di credenziale credenziale portale issuer sistema & UC10\\
        RF17-O & Se la richiesta di credenziale  è approvata, l'issuer sistema genera credenziale richiesta & UC10\\
        RF18-O & L'holder può verificare nel portale Issuer sistema lo stato della richiesta credenziale & UC11\\ 
        RF19-O & Data una richiesta approvata e una credenziale generata l'utente può ottenere nel proprio wallet tale credenziale dal portale issuer sistema & UC12 \\
        RF20-O & L'utente ottiene correttamente la credenziale nel propio wallet & UC12\\
        RF21-O & L'utente visualizza un errore sul wallet che notifica l'errore di rilascio della credenziale & UC13\\
        RF22-O & L'utente visualizza una lista di credenziali memorizzate all'interno del proprio wallet& UC15\\
        RF23-O & L'utente all'interno della propria portale wallet visualizza la credenziale identificativa PID & UC15.1\\
        RF24-O & L'utente all'interno della propria portale wallet visualizza la credenziale identificativa EAA & UC15.2\\
        RF25-O & L'utente elimina le credenziali memorizzate nel wallet & UC16\\
        RF26-O & Il verifier richiede all'utente una credenziale presente sul wallet personale da verificare & UC17\\
        RF27-O & L'utente fornisce tramite il proprio wallet una credenziale al verifier da verificare & UC18\\
        RF28-O & L'utente riesce a visualizzare un messaggio di errore nella piattaforma verifier che notifica l'errore di verifica & UC19\\
        \hline
    \end{longtable}

\subsubsection*{Requisiti non funzionali}
    \begin{longtable}{|c|c|}
        \hline
        \textbf{Codice} & \textbf{Descrizione} \\
        \hline
        %requisiti non funzionali
        RN01-O & Le credenziali devono rispettare il formato JSON W3C\\
        RN02-O & Le credenziali devono essere scambiate con il protocollo OpenID4VC tra issuer e holder\\
        RN03-O & Le credenziali devono essere scambiate con il protocollo OpenID4VP tra holder e verifier\\
        %requisito sull'usabilità?
        RN04-O & Le credenziali possono essere consumate solo dall'effettivo holder\\
        RN05-O & Le credenziali non possono essere modificate\\
        RN06-O & Le credenziali non possono essere intercettate\\
        \hline
    \end{longtable}