\section{P}
\textbf{Processo}: Insieme delle attività correlate e coese che trasformano i bisogni in prodotti (il risultato di un processo si
chiama prodotto). Opera secondo regole consumando risorse.\\
\textbf{Progettista}: Si occupa di definire l'architettura del sistema alla base del prodotto software. Segue la fase dello sviluppo del prodotto.\\
\textbf{Programmatore}: Partecipa sia alla realizzazione che alla manutenzione del prodotto. E competente nella codifica e nella realizzazione di componenti necessarie all’esecuzione delle prove di verifica e validazione. Il codice prodotto dal
programma deve essere mantenibile nel tempo.\\
\textbf{Proof of concept}: Con il termine proof-of-concept si intende una realizzazione incompleta o abbozzata di un determinato progetto o metodo, allo scopo di provarne la fattibilità o dimostrare la fondatezza di alcuni principi o concetti costituenti.\\
