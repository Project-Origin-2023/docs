\section{Lavoro svolto}
Il gruppo inizialmente ha deciso di adottare un metodo a cascata durante la stesura dei documenti iniziai e del way of working. Successivamente con l'analisi dei requisiti e l'inizio della codifica il gruppo ha adottato un metodo agile di tipo scrum al fine di organizzarsi meglio e preventivare in anticipo le attività da svolgere per raggiungere l'obiettivo prefissato.
Ogni sprint presenta le seguenti caratteristiche:
\begin{itemize}
    \item{Durata:} 2 settimane.
    \item{Attività svolte::} Attività che il gruppo si è prefissato e che ha portato a termine.
    \item{Difficoltà riscontrate:} A fine dello sprint, tramite un lavoro di retrospettiva si riflette sulle difficoltà incontrate.
    \item{Resoconto dei costi.}
\end{itemize}

\subsection{Way of working}
Parlare a grandi linee del metodo a cascata e rotazione ruoli


\begin{longtable}{|p{0.085\textwidth}|c|c|c|c|c|c|c|p{0.07\textwidth}|}
    \hline
    Nome &\begin{tabular}[c]{@{}c@{}} dal 26/04\\ al 30/04 \end{tabular} & \begin{tabular}[c]{@{}c@{}}dal 01/05\\ al 04/05\end{tabular} & \begin{tabular}[c]{@{}c@{}}dal 04/05\\ al 12/05\end{tabular} & \begin{tabular}[c]{@{}c@{}}dal 13/05\\ al 19/05\end{tabular} & \begin{tabular}[c]{@{}c@{}}dal 20/05\\ al 26/05\end{tabular} & \begin{tabular}[c]{@{}c@{}}dal 27/5\\ al 03/06\end{tabular} & \begin{tabular}[c]{@{}c@{}}dal 04/06\\ al 09/06\end{tabular} & Totale ore\\
    \hline
    Ibra Elton & AN - 4 ore & VE - 2 ora& AN - 3 ore & RE - 3 ore & AN - 3 ore & VE -2 ore & PR - 4 ore  & 20\\
    \hline
    Beschin Michele & AN - 3 ore & AN - 3 ore & RE - 4 ore & AN - 2 ore & VE - 2 ore & AN - 4 ore & PR - 3 ore & 21 \\
    \hline
    Lotto Riccardo & AN - 3 ore & RE - 4 ore & AN - 3 ore & VE - 1 ore & AN - 4 ore & VE - 2 ore & PT - 3 ore & 20\\
    \hline
    Bobirica Andrei Cristian & AN - 4 ore & AN - 3 ore & VE - 1 ore & AN - 2 ore & VE - 2 ore & AN - 4 ore & RE - 5 ore & 22\\
    \hline
    Andreetto Alessio & AN - 3 ore & VE - 2 ore & AN - 3 ore & VE - 2 ore & AN - 4 ore & RE - 4 ore & AN - 3 ore & 21\\
    \hline
    Corbu Teodor Mihail & AN - 3 ore & AN - 4 ore & VE - 2 ore & AN - 2 ore & RE - 3 ore & AN - 3 ore & VE - 2 ore & 19\\
    \hline
\end{longtable}

\begin{longtable}{|c|c|}
    \hline
    \textbf{Legenda} & \\
    \hline
    RE & Responsabile \\
    \hline
    AN & Analista \\
    \hline
    VE & Verificatore \\
    \hline
    PR & Programmatore \\
    \hline
    PT & Progettista \\
    \hline
    AM & Amministratore \\
    \hline
\end{longtable}




\subsection{Requirements and technology baseline}
In vista della prima revisione RTB il gruppo ha scelto di cambiare metodologia di lavoro. Per facilitare la stesura del codice si è scelto di optare per una metodologia agile suddivisa in sprint.


\subsubsection{Sprint 1}
\begin{itemize}
\item \textbf{Durata:} 29/05/2023 - 09/06/2023 
\item \textbf{Attività svolte:}
\begin{itemize}
    \item Ampliamento e revisione della sezione "casi d'uso" all'interno del documento Analisi dei requisiti;
    \item analisi e studio dei framework e librerie da utilizzare per lo sviluppo del codice;
    \item creazione e configurazione dell'ambiente di sviluppo;
    \item Inizio sviluppo piattaforma issuer.
\end{itemize}
\item \textbf{Difficoltà riscontrate:}
\begin{itemize}
    \item Lo sviluppo della piattaforma holder non era compatibile parallelamente a quella issuer. 
    \item Difficoltà nel gestire tutte le istanze all'interno di Docker\glo, si è optato per mantenere su container docker soltanto le API.
\end{itemize}
\item \textbf{Resoconto dei costi:}
\end{itemize}

\subsubsection{Sprint 2}
\begin{itemize}
    \item \textbf{Durata:} 12/06/2023 - 23/06/2023 
    \item \textbf{Attività svolte:}
    \begin{itemize}
        \item È stata implementata la libreria grafica Material UI su un branch separato;
        \item é stato implementato il sistema di login per issuer app, 
        \item implementazione del front-end per la richiesta di credenziale nell'issuer.
    \end{itemize}
    \item \textbf{Difficoltà riscontrate:}
    \begin{itemize}
        \item Al momento attuale lo sviluppo del verifier non è compatibile parallelamente ad issuer.
    \end{itemize}
    \item \textbf{Resoconto dei costi:}
    \end{itemize}

\subsubsection{Sprint 3}
\begin{itemize}
    \item \textbf{Durata:} 26/06/2023 - 07/07/2023 
    \item \textbf{Attività svolte:} 
    \item \textbf{Difficoltà riscontrate:}
    \begin{itemize}
        \item In vista del PoC abbiamo sperimentato l’utilizzo della libreria walt.id per la generazione di credenziali.
        Abbiamo riscontrato delle problematiche riguardanti bug/mancanza di documentazione/parti di funzionalità non complete.
    \end{itemize}
    \item \textbf{Resoconto dei costi:}
    \end{itemize}

\subsubsection{Sprint 4}
\begin{itemize}
    \item \textbf{Durata:} 10/07/2023 - 21/07/2023 
    \item \textbf{Attività svolte:}
    \begin{itemize}
        \item la parte grafica prima implementata su un branch a parte è stata implementata anche sull'applicativo issuer,
        \item assieme alla parte grafica è stata aggiunta una navbar nell'applicativo issuer 
        \item il database ha subito delle modifiche in particolare alcuni campi oltre che cambiare nome hanno cambiato valori registrati,
        \item i macro obiettivi che volevamo raggiungere sono stati suddivisi in task più piccoli da gestire per il gruppo e per i componenti singoli.
    \end{itemize}
    \item \textbf{Difficoltà riscontrate:}
    \item \textbf{Resoconto dei costi:}
    \end{itemize}