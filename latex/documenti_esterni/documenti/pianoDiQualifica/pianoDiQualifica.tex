\documentclass[a4paper]{article}
\usepackage[normalem]{ulem}
\usepackage{eurosym}
\usepackage[font=small,labelfont=bf]{caption}

% impostazioni generali
\input{../../../template/package.tex}

% dati della prima pagina
% Configurazione della pagina iniziale
\newcommand{\doctitle}{\textit{Verbale} interno del 11 maggio 2023}
\newcommand{\rev}{0.1.0} % versione
\newcommand{\resp}{Beschin Michele} % inserire responsabile
\newcommand{\red}{Andreetto Alessio} % cognome nome dei redattori
\newcommand{\ver}{Lotto Riccardo} % cognome nome del verificatore
\newcommand{\uso}{Interno} %interno o esterno
\newcommand{\dest}{\textit{ProjectOrigin}
	\\ Prof. Vardanega Tullio 
	\\ Prof. Cardin Riccardo}
\newcommand{\describedoc}{\textit{Verbale} riguardante il meeting tenuto il 11 maggio 2023}

 % per modificare la prima pagina editare questo file

\makeindex

\makeatletter
\renewcommand\paragraph{
\@startsection {paragraph}{4}{0mm}{-\baselineskip}{.5\baselineskip}{\normalfont \normalsize \bfseries }}
\makeatother

\begin{document}

% Prima pagina
\thispagestyle{empty}
\renewcommand{\arraystretch}{1.3}

\begin{titlepage}
	\begin{center}
		
	\includegraphics[scale = 0.20]{../../../template/images/logo.png}
	\\[1cm]
	\href{mailto:projectorigin2023@gmail.com}		      	
	{\large{\textit{projectorigin2023@gmail.com} } }\\[2.5cm]
	\Huge \textbf{\doctitle} \\[1cm]
	 \large
			 \begin{tabular}{r|l}
                        \textbf{Versione} & \rev{} \\
                        \textbf{Responsabile} & \resp{} \\
                        \textbf{Redattori} & \red{} \\ 
                        \textbf{Verificatori} &  \ver{} \\
                        \textbf{Uso} & \uso{} \\                        
                        \textbf{Destinatari} & \parbox[t]{5cm}{ \dest{} }
                \end{tabular} 
                \\[3.3cm]
                \large \textbf{Descrizione} \\ \describedoc{} 
     \end{center}
\end{titlepage}

% Diario delle modifiche
\input{../../../template/changelog.tex}
\changelogTable{
0.3.0 & 2023-05-10 & Corbu Teodor & Verificatore & Verifica documento \\
0.2.1 & 2023-05-10 & Ibra Elton & Analista & Stesura sottocapitoli dei \S\ Casi d'uso \\
0.2.0 & 2023-05-09 & Corbu Teodor & Verificatore & Verifica documento \\
0.1.1 & 2023-05-09 & Ibra Elton & Analista & Inizio stesura \S\ Casi d'uso \\  
0.1.0 & 2023-05-04 & Ibra Elton & Verificatore & Verifica documento\\    
0.0.3 & 2023-05-03 & Corbu Teodor & Analista & Stesura \S\ Descrizione Generale \\
0.0.2 & 2023-05-03 & Corbu Teodor & Analista & Stesura \S\ Introduzione \\
0.0.1 & 2023-05-02 & Corbu Teodor & Analista & Creazione struttura documento\\
} % editare questo
\pagebreak


% Indice
{
    \hypersetup{linkcolor=black}
    \tableofcontents
    \listoffigures %elenco figure
    \listoftables  %elenco tabelle
}
\pagebreak

% sezioni comuni 
% Informazioni generali
% \setcounter{table}{0} forse questo serve se l'indice delle tabelle parte da numeri sospetti
\section{Introduzione}
\subsection{Scopo del documento}
La Specifica Tecnica si pone come obiettivo di descrivere in modo esaustivo l'organizzazione della struttura del software, delle tecnologie adottate e delle scelte architetturali compiute dal gruppo durante le fasi di progettazione e di codifica del prodotto.\\
All’interno del documento si possono trovare gli schemi delle classi per delineare l'architettura e le funzionalità chiave del prodotto, con l'obiettivo di fornire una comprensione completa e chiara del sistema e delle interazioni interne.\\
Il documento contiene anche una sezione per i requisiti che vengono soddisfatti dal prodotto; questo permette al gruppo  di valutare il progresso del lavoro e di tener traccia degli obiettivi imposti.

\subsection{Scopo del prodotto} 
Lo scopo del prodotto è quello di creare una versione semplificata di un applicativo per implementare e rilasciare un "portafoglio di identità digitale" conforme a un insieme di 
standard, in modo che possa essere utilizzato con qualsiasi servizio, che adotti tale struttura, in qualsiasi paese dell'UE. \\
In particolare, si dovrà realizzare una web app\glo{} avendo queste componenti architetturali:
\begin{itemize}
    \item Un componente back-office per consentire al dipendente dell'organizzazione emittente di verificare\glo{} manualmente la richiesta di credenziali e autorizzarne l'emissione; 
    \item Un componente di interazione con l'utente dimostrativo per consentire all'utente (titolare) di navigare e richiedere specifiche credenziali da un emittente 
(ad esempio, il sito di una demo universitaria); 
    \item Un componente di interazione con l'utente dimostrativo per consentire all'utente (titolare) di navigare un sito verificatore\glo{} e fornire le credenziali richieste;
    \item Un'app front-end per l'utente per archiviare e gestire le proprie credenziali; 
    \item Un componente di comunicazione per consentire lo scambio di credenziali/presentazioni secondo un protocollo standard - il componente di comunicazione sarà implementato 
tre volte nei tre contesti (lato emittente, lato titolare, lato verificatore).
\end{itemize}

\subsection{Note Esplicative}
Alcuni termini utilizzati nel documento possono avere significati ambigui a seconda del contesto. Al fine di evitare equivoci, è stato creato un \Glodocumento contenente tali termini 
e il loro significato specifico. Per segnalare che un termine è presente nel \Glodocumento, sarà aggiunta una "g" a pedice accanto al termine corrispondente nel testo.

\subsection{Riferimenti}
\textbf{1. Normativi:} 
\begin{itemize}
    \item \textbf{\NdPdocumento}: contengono le norme e gli strumenti per gli analisti;
    \item \textbf{Capitolato d’appalto C3}: \url{https://www.math.unipd.it/~tullio/IS-1/2022/Progetto/C3.pdf};
    \item \textbf{Regolamento del progetto didattico:}: \url{https://www.math.unipd.it/~tullio/IS-1/2022/Dispense/PD02.pdf}.
\end{itemize}

\textbf{2. Informativi:} 
\begin{itemize}
    \item \textbf{Analisi dei Requisiti v1.0.0};
    \item \textbf{Qualità di prodotto – slide T8 di Ingegneria del Software: }: \url{https://www.math.unipd.it/~tullio/IS-1/2022/Dispense/T08.pdf};
    \item \textbf{Qualità di processo – slide T9 di Ingegneria del Software: }: \url{https://www.math.unipd.it/~tullio/IS-1/2022/Dispense/T09.pdf};
    \item \textbf{Verifica e Validazione: introduzione – slide T10 di Ingeneria del Software:}: \url{https://www.math.unipd.it/~tullio/IS-1/2022/Dispense/T10.pdf};
    \item \textbf{Verifica e Validazione: introduzione – slide T11 di Ingeneria del Software:}: \url{https://www.math.unipd.it/~tullio/IS-1/2022/Dispense/T11.pdf};
    \item \textbf{Verifica e Validazione: introduzione – slide T12 di Ingeneria del Software:}: \url{https://www.math.unipd.it/~tullio/IS-1/2022/Dispense/T12.pdf}.

\end{itemize}

\newpage %forse si può togliere il newpage
\section{Qualità di Processo}
Per garantire la qualità dei processi il gruppo utilizza come riferimento lo standard \textbf{ISO/IEC 12207:1995}.
Dopo aver studiato i vari processi, si sono scelti quelli più adatti alle necessità del progetto.
\subsection{Processi primari}


\begin{longtable}{ 
    >{\centering}M{0.20\textwidth} 
    >{\centering}M{0.50\textwidth}
    >{\centering}M{0.17\textwidth} 
    }
\rowcolorhead
\headertitle{Processo} &
\centering \headertitle{Descrizione} &	
\headertitle{Metriche} 
\endfirsthead
\endhead

Fornitura & Scelta delle procedure e delle risorse necessarie per garantire che i servizi siano forniti in modo tempestivo & - \tabularnewline
Sviluppo & Realizzazione di un prodotto software di qualità, che soddisfi le esigenze del cliente& - \tabularnewline
\end{longtable}

\subsection{Processi di supporto}
\begin{longtable}{ 
    >{\centering}M{0.20\textwidth} 
    >{\centering}M{0.50\textwidth}
    >{\centering}M{0.17\textwidth} 
    }
\rowcolorhead
\headertitle{Processo} &
\centering \headertitle{Descrizione} &	
\headertitle{Metriche} 
\endfirsthead
\endhead

Verifica & Si determina se i requisiti del prodotto sono soddisfatti& - \tabularnewline
Gestione della qualità & Viene garantita la conformità dei processi e la conformità con gli standard prefissati& - \tabularnewline
Documentazione & Controllo della leggibilità della documentazione prodotta& - \tabularnewline %per questo usiamo l'indice di gulpease
\end{longtable}
\subsection{Processi organizzativi}
\begin{longtable}{ 
    >{\centering}M{0.20\textwidth} 
    >{\centering}M{0.50\textwidth}
    >{\centering}M{0.17\textwidth} 
    }
\rowcolorhead
\headertitle{Processo} &
\centering \headertitle{Descrizione} &	
\headertitle{Metriche} 
\endfirsthead
\endhead

Gestione organizzativa & Controllo e organizzazione delle prestazioni di un processo& - \tabularnewline
\end{longtable}
\section{Qualità di Prodotto}
Per valutare tale qualità del prodotto software, il gruppo ha deciso di fare riferimento allo standard \textbf{ISO/IEC 9126}, per soddisfare alcuni degli standard di qualità tra quelli proposti
dal medesimo, ovvero quelli che il gruppo ha ritenuto necessari per il progetto. Lo standard definisce delle caratteristiche generali e delle metriche.
\subsection{Obiettivi}
\begin{longtable}{ 
    >{\centering}M{0.20\textwidth} 
    >{\centering}M{0.50\textwidth}
    >{\centering}M{0.17\textwidth} 
    }
\rowcolorhead
\headertitle{Tipologia} &
\centering \headertitle{Descrizione} &	
\headertitle{Metriche} 
\endfirsthead
\endhead

Funzionalità & Il prodotto deve fornire le funzionalità necessarie per soddisfare i requisiti stabiliti nell'Analisi dei Requisiti & MPD01 \tabularnewline
Affidabilità & Capacità del prodotto di funzionare, evitando errori& MPD02 \tabularnewline
Efficienza & Il prodotto deve garantire prestazioni adeguate rispetto alle risorse utilizzate& MPD03 \tabularnewline
Usabilità & Capacità del prodotto di essere utilizzato correttamente dall'utente& MPD04 \tabularnewline
Manutenibilità & Capacità del prodotto di poter essere modificato& MPD05, MPD06 \tabularnewline
Portabilità & Capacità del prodotto di poter essere utilizzato in un altro ambiente di esecuzione& MPD07 \tabularnewline
\end{longtable}
\subsection{Metriche}

\begin{longtable}{ 
    >{\centering}M{0.10\textwidth} 
    >{\centering}M{0.25\textwidth}
    >{\centering}M{0.25\textwidth} 
    >{\centering\arraybackslash}M{0.25\textwidth} 
    }
\rowcolorhead
\headertitle{\textbf{ID}} &
\centering \headertitle{\textbf{Nome}} &
\centering \headertitle{\textbf{Valore minimo}} &
\centering \headertitle{\textbf{Valore ottimale}}
\endfirsthead
\endhead
%\textbf{MPR001} & \centering Comprensibilità delle funzioni & Indica la percentuale di funzionalità comprese da un utente generico & 80\% & 100\% \tabularnewline
\textbf{MPD01} & \centering Copertura dei Requisiti  & 100\% requisiti obbligatori &  100\% \tabularnewline
\textbf{MPD02} & \centering Tasso di fallimenti & 0-40\% & 0-10\% \tabularnewline
\textbf{MPD03} & \centering Tempo di caricamento & 0-12 secondi & 0-6 secondi \tabularnewline
\textbf{MPD04} & \centering Tempo di apprendimento & 0-30 minuti & 0-10 minuti \tabularnewline
\textbf{MPD05} & \centering Complessità Ciclomatica & 0-10 & 0-5 \tabularnewline
\textbf{MPD06} & \centering Densità commenti & 20-30\% & 5-10\% \tabularnewline
\textbf{MPD07} & \centering Browser supportati & 70\% & 100\% \tabularnewline
\end{longtable}
\pagebreak


\end{document}
