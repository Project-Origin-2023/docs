\section{Introduzione}
\subsection{Scopo del documento}
Lo scopo del seguente documento è quello di definire le metodologie che il gruppo intende attuare per garantire la qualità del prodotto per tutta la durata del progetto.
Per rispettare questi obiettivi verrà definito un sistema di verifica\glo e validazione che consentirà di correggere eventuali errori il prima possibile, evitando gli sprechi di tempo.
\subsection{Scopo del prodotto} %qui possiamo anche chiedere feedback all'azienda, però forse non è il massimo il copy-paste dall'analisi dei capitolati, DA RIVEDERE
Lo scopo del progetto è quello di realizzare un sistema di autenticazione dove un ente rilascia
certificati di identità ad un utente, che le memorizza in un wallet\glo, e le può utilizzare
per accedere a servizi. Per farlo, occorre realizzare una webapp.
\subsection{Glossario}
Per evitare ambiguità sui termini, il gruppo ha deciso di redarre un glossario. Tali termini verranno contrassegnati con una lettera g (esempio\textsubscript{g}) a fine della parola. In tale documento sono contenenuti i principali termini tecnici, specifici del progetto, con la relativa definizione.
%Aggiungere riferimento al nostro glossario, che al momento non c'è
\subsection{Standard di progetto}
Come gruppo abbiamo deciso di gestire i processi di ciclo di vita del software istanziando alcune parti dello standard \textbf{ISO/IEC 12207}.
\subsection{Riferimenti normativi}%qui mettiamo il riferimento alle regole del progetto di Tullio e altre robe
\subsubsection{Riferimenti Normativi}
\begin{itemize}
    \item \textit{Norme di Progetto}: v 0.0.4
\end{itemize}
\subsubsection{Riferimenti Informativi}
\begin{itemize}
    \item Regolamento del progetto didattico:\\ \url{https://www.math.unipd.it/~tullio/IS-1/2022/Dispense/PD02.pdf}
    \item Processi di ciclo di vita: \url{https://www.math.unipd.it/~tullio/IS-1/2022/Dispense/T02.pdf}
    \item Qualità di processo: \url{https://www.math.unipd.it/~tullio/IS-1/2022/Dispense/T09.pdf}
    \item Qualità di prodotto: \url{https://www.math.unipd.it/~tullio/IS-1/2022/Dispense/T08.pdf}
    \item Indice di Gulpease: \url{https://it.wikipedia.org/wiki/Indice_Gulpease}
    %\item Standard SPICE: vediamo se metterlo o meno
\end{itemize}

