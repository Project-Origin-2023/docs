\section{Qualità di Processo}
Per garantire la qualità dei processi il gruppo utilizza come riferimento lo standard \textbf{ISO/IEC 12207:1995}.
Dopo aver studiato i vari processi, si sono scelti quelli più adatti alle necessità del progetto.
\subsection{Processi primari}


\begin{longtable}{ 
    >{\centering}M{0.20\textwidth} 
    >{\centering}M{0.50\textwidth}
    >{\centering}M{0.17\textwidth} 
    }
\rowcolorhead
\headertitle{Processo} &
\centering \headertitle{Descrizione} &	
\headertitle{Metriche} 
\endfirsthead
\endhead

Fornitura & Scelta delle procedure e delle risorse necessarie per garantire che i servizi siano forniti in modo tempestivo & MPR01, MPR02, MPR03, MPR04, MPR05, MPR06 \tabularnewline
Sviluppo & Realizzazione di un prodotto software di qualità, che soddisfi le esigenze del cliente& MPR07 \tabularnewline
\end{longtable}

\subsection{Processi di supporto}
\begin{longtable}{ 
    >{\centering}M{0.20\textwidth} 
    >{\centering}M{0.50\textwidth}
    >{\centering}M{0.17\textwidth} 
    }
\rowcolorhead
\headertitle{Processo} &
\centering \headertitle{Descrizione} &	
\headertitle{Metriche} 
\endfirsthead
\endhead

Verifica\glo & Si determina se i requisiti del prodotto sono soddisfatti& MPR08 \tabularnewline
Gestione della qualità & Viene garantita la conformità dei processi e la conformità con gli standard prefissati& MPR09 \tabularnewline
Documentazione & Controllo della leggibilità della documentazione prodotta& MPR10 \tabularnewline
\end{longtable}
\subsection{Processi organizzativi}
\begin{longtable}{ 
    >{\centering}M{0.20\textwidth} 
    >{\centering}M{0.50\textwidth}
    >{\centering}M{0.17\textwidth} 
    }
\rowcolorhead
\headertitle{Processo} &
\centering \headertitle{Descrizione} &
\headertitle{Metriche} 
\endfirsthead
\endhead

Gestione organizzativa & Controllo e organizzazione delle prestazioni di un processo\glo& - \tabularnewline
\end{longtable}
\subsection{Metriche}
%tabella delle metriche utilizzate, con id, nome, valore minimo e valore ottimale
\begin{longtable}{ 
    >{\centering}M{0.10\textwidth} 
    >{\centering}M{0.25\textwidth}
    >{\centering}M{0.25\textwidth} 
    >{\centering\arraybackslash}M{0.25\textwidth} 
    }
\rowcolorhead
\headertitle{\textbf{ID}} &
\centering \headertitle{\textbf{Nome}} &
\centering \headertitle{\textbf{Valore minimo}} &
\centering \headertitle{\textbf{Valore ottimale}}
\endfirsthead
\endhead
%\textbf{MPR001} & \centering Comprensibilità delle funzioni & Indica la percentuale di funzionalità comprese da un utente generico & 80\% & 100\% \tabularnewline
\textbf{MPR01} & \centering Cost Variance (CV)  & $\geq$ 0\euro &  0\euro \tabularnewline
\textbf{MPR02} & \centering Schedule Variance (SV)  & $\geq$ -20\% & 0\% \tabularnewline
\textbf{MPR03} & \centering Budget Variance (BV)  & $\geq$ 0\euro & 0\euro \tabularnewline
\textbf{MPR04} & \centering Estimated At Completion (EAC)  & $\leq$ preventivo -10\%\\ $\geq$ preventivo +10\%& preventivo \tabularnewline
\textbf{MPR05} & \centering Estimated To Complete (ETC)  & $\geq$ 0\euro & $\leq$ EAC \tabularnewline
\textbf{MPR06} & \centering Planned Value (PV)  & $\geq$ 0 \euro & $\leq$ Budget at Completion \tabularnewline
\textbf{MPR07} & \centering Requirements Stability Index (RSI) & $\geq$ 70\% & 100\% \tabularnewline
\textbf{MPR08} & \centering Code Coverage & $\geq$ 70\% & 100\% \tabularnewline
\textbf{MPR09} & \centering Metriche Soddisfatte & $\geq$ 70\% & 100\% \tabularnewline
\textbf{MPR10} & \centering Indice di Gulpease & $\geq$ 40 & $\geq$ 60 \tabularnewline
\end{longtable}