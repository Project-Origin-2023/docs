\section{Architettura}

\subsection{Introduzione}
Per la Realizzazione delle 3 webapp è stata adottata un architettura a microservizi, separando le funzioni di back-end da quelle del front-end. I vari microservizi di ciascuna webapp sono stati sviluppati su container docker differenti.
\subsubsection{Container Front-end}
\begin{itemize}
    \item \textbf{originIssuer:} È la componete di front-end della webapp Issuer. Essa si interfaccia con l'utente per inviare segnali e ricevere dati dal Back-end \textit{originIssuerApi}. Per la realizzazione del codice è stato adottato il pattern architetturale MVVM (Model-View-ViewModel).
    \item \textbf{originWallet:} È la componete di front-end della webapp Wallet. Essa si interfaccia con l'utente per inviare segnali e ricevere dati dal Back-end \textit{originWalletApi}. Per la realizzazione del codice è stato adottato il pattern architetturale MVVM (Model-View-ViewModel).
    \item \textbf{originVerifier:}È la componete di front-end della webapp Verifier. Essa si interfaccia con l'utente per inviare segnali e ricevere dati dal Back-end \textit{originVerfierApi}. Per la realizzazione del codice è stato adottato il pattern architetturale MVVM (Model-View-ViewModel).
\end{itemize}

\subsubsection{Container Back-end}
\begin{itemize}
    \item \textbf{originIssuerApi:} È la componente di back-end della webapp Issuer che si occupa di gestire le richieste provenienti dal front-end e di comunicare con il database \textit{originIssuerDB} per la memorizzazione dei dati.
    \item \textbf{originWalletApi:} È la componente di back-end della webapp Wallet che si occupa di gestire le richieste provenienti dal front-end e di comunicare con il database \textit{originWalletDB} per la memorizzazione dei dati.
\end{itemize}

\subsubsection{Container database}
\begin{itemize}
    \item \textbf{originIssuerDB:} È la componente della webapp Issuer che va a eseguire operazioni sul database per la richiesta e la memorizzazione di dati. 
    \item \textbf{originWalletDB:} È la componente della webapp Wallet che va a eseguire operazioni sul database per la richiesta e la memorizzazione di dati.
\end{itemize}

\subsubsection{Conteiner WaltId per strandard openID}
\begin{itemize}
    \item \textbf{openIdIssuer:} È una componente della libreria WaltID per mantenere lo standard openId di comunicazione tra Issuer e le altre webapp, al fine di rispettare le richieste del capitolato.
    \item \textbf{openIdWallet:} È una componente della libreria WaltID per mantenere lo standard openId di comunicazione tra Wallet e le altre webapp, al fine di rispettare le richieste del capitolato.
    \item \textbf{openIdVerifier:}È una componente della libreria WaltID per mantenere lo standard openId di comunicazione tra Verifier e le altre webapp, al fine di rispettare le richieste del capitolato.
\end{itemize}
 
\subsubsection{Container di supporto}
\begin{itemize}
    \item \textbf{adminer:} È un container che permette agli sviluppatori di gestire il database tramite interfaccia web.
    \item \textbf{nginx:} Lo utilizziamo come server proxy per gestire il reindirezzamento del traffico http tramite domini verso i container interni della rete docker
\end{itemize}






\subsection{Componenti Back-end}
\subsubsection{OriginiIssuerApi}
\subsubsection{OriginWalletApi}

\subsection{Componenti Front-end}
\subsubsection{originIssuer} 
\begin{itemize}
    \item 
\end{itemize}
\subsubsection{originWallet}
\begin{itemize}
    \item \textbf{App:} questa è la pagina iniziale dell'applicazione, dove viene definito il \textit{routing} delle pagine.
    \item \textbf{components/Navbar:} questa è la componente che definisce la \textit{navbar} dell'applicazione, che differisce dal tipo di utente che è loggato (user, guest, ...).
    \item \textbf{controller/LoginController:} questa è la componente che gestisce la pagina di login dell'applicazione. Esso crea la corrispondente 
    \textit{LoginViewModel} e la corrispondente \textit{LoginView}. Il \textit{controller} si occupa di gestire gli eventi provenienti 
    dalla \textit{view} e di aggiornare la \textit{viewModel}, che a sua volta aggiorna il \textit{model} presente nel back-end.
    \item \textbf{controller/LoginViewModel:} Viene creato dal \textit{controller} ma non ha nessun riferimento ad esso, posside solo il riferimento al modello dei dati che esso posside.
    \item \textbf{controller/LoginView:} viene creato dal \textit{controller} ma non ha nessun riferimento ad esso, posside solo il riferimento al modello dei dati che esso posside e qualche indicazione di \textit{handling} dei dati.
    \item \textbf{ViewModel:} componente che si occupa del collegamento con il back-end, ha quindi un riferimento al \textit{model}. È unico per tutta l'applicazione\\
    \\NB. Tutte le componenti seguono la struttura MVVM sopra descritta per la componente \textit{Login}.
    \item \textbf{ListCredential:} componente che si occupa di mostrare la lista delle credenziali dell'utente presenti nel \textit{Wallet}. Da qui si può andare nel dettaglio di una singola credenziale.
    \item \textbf{DetailCredential:} componente che si occupa di mostrare i dettagli di una credenziale presente nel \textit{Wallet}. Da qui si può eliminare una credenziale.\\
    \\NB. Le seguenti componenti hanno nomi controintuitivi ma sono imposti dallo standard \textit{openID}.
    \item \textbf{InitiateIssuance:} componete che si  occupa dell'accettazione di una richiesta di credenziale da parte di un \textit{Issuer} e vengo reindirizzato alla pagina \textit{ListCredential}.
    \item \textbf{CredentialRequest:} Componente che si occupa del re indirizzamento di una richiesta di presentazione di una credenziale parte del \textit{Verifier}. 
\end{itemize}
\subsubsection{originVerifier}

\subsection{Componenti}


\subsection{Diagramma delle classi}

\subsection{Design pattern}

