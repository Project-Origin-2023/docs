\section{Architettura}

\subsection{Introduzione}
Per la Realizzazione delle 3 webapp è stata adottata un architettura a microservizi, separando le funzioni di back-end da quelle del front-end. I vari microservizi di ciascuna webapp sono stati sviluppati su container docker differenti.
\subsubsection{Container Front-end}
\begin{itemize}
    \item \textbf{originIssuer:} È la componete di front-end della webapp Issuer. Essa si interfaccia con l'utente per inviare segnali e ricevere dati dal Back-end \textit{originIssuerApi}. Per la realizzazione del codice è stato adottato il pattern architetturale MVVM (Model-View-ViewModel).
    \item \textbf{originWallet:} È la componete di front-end della webapp Wallet. Essa si interfaccia con l'utente per inviare segnali e ricevere dati dal Back-end \textit{originWalletApi}. Per la realizzazione del codice è stato adottato il pattern architetturale MVVM (Model-View-ViewModel).
    \item \textbf{originVerifier:}È la componete di front-end della webapp Verifier. Essa si interfaccia con l'utente per inviare segnali e ricevere dati dal Back-end \textit{originVerfierApi}. Per la realizzazione del codice è stato adottato il pattern architetturale MVVM (Model-View-ViewModel).
\end{itemize}

\subsubsection{Container Back-end}
\begin{itemize}
    \item \textbf{originIssuerApi:} È la componente di back-end della webapp Issuer che si occupa di gestire le richieste provenienti dal front-end e di comunicare con il database \textit{originIssuerDB} per la memorizzazione dei dati.
    \item \textbf{originWalletApi:} È la componente di back-end della webapp Wallet che si occupa di gestire le richieste provenienti dal front-end e di comunicare con il database \textit{originWalletDB} per la memorizzazione dei dati.
\end{itemize}

\subsubsection{Container database}
\begin{itemize}
    \item \textbf{originIssuerDB:} È la componente della webapp Issuer che va a eseguire operazioni sul database per la richiesta e la memorizzazione di dati. 
    \item \textbf{originWalletDB:} È la componente della webapp Wallet che va a eseguire operazioni sul database per la richiesta e la memorizzazione di dati.
\end{itemize}

\subsubsection{Conteiner WaltId per strandard openID}
\begin{itemize}
    \item \textbf{openIdIssuer:} È una componente della libreria WaltID per mantenere lo standard openId di comunicazione tra Issuer e le altre webapp, al fine di rispettare le richieste del capitolato.
    \item \textbf{openIdWallet:} È una componente della libreria WaltID per mantenere lo standard openId di comunicazione tra Wallet e le altre webapp, al fine di rispettare le richieste del capitolato.
    \item \textbf{openIdVerifier:}È una componente della libreria WaltID per mantenere lo standard openId di comunicazione tra Verifier e le altre webapp, al fine di rispettare le richieste del capitolato.
\end{itemize}
 
\subsubsection{Container di supporto}
\begin{itemize}
    \item \textbf{adminer:} È un container che permette agli sviluppatori di gestire il database tramite interfaccia web.
    \item \textbf{nginx:} Lo utilizziamo come server proxy per gestire il reindirezzamento del traffico http tramite domini verso i container interni della rete docker
\end{itemize}






\subsection{Componenti Back-end}
\subsubsection{OriginiIssuerApi}
\subsubsection{OriginWalletApi}

\subsection{Componenti Front-end}
\subsubsection{originIssuer} 
\subsubsection{originWallet}
\subsubsection{originVerifier}

\subsection{Componenti}


\subsection{Diagramma delle classi}

\subsection{Design pattern}

