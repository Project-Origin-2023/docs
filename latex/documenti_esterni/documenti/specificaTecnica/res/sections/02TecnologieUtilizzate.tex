\section{Tecnologie Utilizzate}
Essendo la struttura delle webapp a microservizi  ono state utilizzate le segueni tecnologie:
\begin{itemize}
    \item Docker,
    \item Docker Container.
\end{itemize}

Per il deployment dei container è stata gestita la configurazione tramite un file docker compose \textit{docker-compose.yaml}.
\subsection{Tecnologie Back-end}
\begin{itemize}
    \item Node.js + Javascript
    \item Express: per la creazione di endpoint e creazione di server Http. 
\end{itemize}

\subsection{Tecnologie Front-end}
Per la realizzazione della parte grafica degli applicativi web sono state usate le seguenti tecnologie:
\begin{itemize}
    \item Vite + React + Javascript,
    \item Axios per eseguire le chiamate http,
    \item Material UI.
\end{itemize}

Vite nello specifico ci è servito per gestire meglio le dipendenze, fare il building della webapplication in maniera migliore e avere più velocità. \\\

Material-UI è una libreria di componenti UI (User Interface) per React fatto da Google. Material-UI è una libreria che viene utilizzata principalmente per realizzare interfacce utente dinamiche e personalizzate.
Segue il design system “Material Design”, che si occupa di fornire linee guida di design per la realizzazione di applicazioni e siti web.\\

\subsection{Tecnologie Database}
\begin{itemize}
    \item BDMS PostegreSQL
\end{itemize}


\subsection{Tecnologie Di Supporto}
\begin{itemize}
    \item nginx: Proxy manager per gestire il flusso di dati in arrivo dall'esterno delle nostre componenti e reindirizzarlo all'interno dei nostri componenti docker.
    \item WaltID: utilizzato per rispettare lo standard OpenIDV4 CI-VP con container messi a disposizione da waltID.
\end{itemize}

\subsection{Autenticazione e sicurezza}