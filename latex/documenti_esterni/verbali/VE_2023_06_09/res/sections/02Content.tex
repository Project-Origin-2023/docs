\section{Ordine del giorno}
\begin{enumerate}
\item Casi d'uso
\item Database
\item PoC
\item Walt.id
\item Capitolato
\end{enumerate}

\subsection{Casi d'uso}
Durante le prime fasi di consultazione con l'azienda, è stata sollevata una domanda riguardante i casi d'uso. È stato suggerito che sarebbe preferibile astrarre i casi d'uso per soddisfare le esigenze specifiche:
\begin{itemize}
  \item Wallet conteniment;
  \item Issuer verifier: rialasciare e ricevere.
\end{itemize}
Se generalizziamo adeguatamente i casi d'uso, potremmo creare un'unica web app che integri entrambe le funzionalità.
L'architettura dell'applicazione deve essere attentamente progettata per gestire correttamente l'interazione tra le diverse componenti e la struttura sottostante.
È importante verificare che i casi d'uso rientrino all'interno delle specifiche richieste.
Si potrebbe considerare l'implementazione delle API come una libreria in esecuzione, in modo da fornire un'istanza pronta all'uso che può essere integrata nella web app principale.


\subsection{Database}
Si è discusso del database issuer per tenere traccia delle credenziali da revoca, è stato implementato un database emittente(Issuer) nel nostro sistema. Tuttavia, è importante chiarire che il nostro database 
serve come mezzo per registrare i login degli utenti, non come database per gestire specificatamente le credenziali da revoca. 
Per quanto riguarda la verifica e gestione delle credenziali da revocare, facciamo affidamento su un sistema di login fornito da "walt.id".
Tuttavia, è importante notare la revoca non è necessaria. Per quanto riguarda le preapproviazioni necessarie per l'approvazione delle credenziali, abbiamo implementato una procedura di login esplicito sul sito dell'emittente (issuer) per gestirle in modo adeguato.

\subsection{PoC}
L'obiettivo che ci siamo posti è molto più avanzato rispetto a un semplice Proof of Concept (PoC); potremmo definirlo come una sorta di mockup. Il nostro scopo è mostrare ampie porzioni di funzionalità, suddividendole in due categorie distintive per una comunicazione chiara:
\begin{itemize}
  \item Aspetto funzionale: Questo aspetto metterà in mostra le interfacce utente, evidenziando le funzionalità e le operazioni disponibili per gli utenti. Attraverso una rappresentazione visiva, dimostreremo l'aspetto intuitivo e l'interazione fluida del sistema.
  \item Aspetto tecnologico: In questo ambito, ci concentreremo sulle funzionalità che sono state effettivamente implementate e che sono funzionanti. Mostreremo il flusso di comunicazione tra i vari componenti del sistema, evidenziando l'integrazione tra i moduli e dimostrando la capacità del sistema di eseguire le operazioni richieste.
\end{itemize}
Sarà di fondamentale importanza comunicare in modo chiaro e conciso, garantendo che le informazioni trasmesse siano comprensibili e che le dimostrazioni siano efficaci per illustrare le caratteristiche e il funzionamento del sistema.

\subsection{Wald.id}
Attualmente non siamo sicuri se la versione di "walt.id" che stiamo utilizzando sia allineata all'ultima libreria disponibile a febbraio. Tuttavia, ci stiamo focalizzando sull'automazione dell'emissione delle credenziali e ci affidiamo a "walt.id" per questa funzionalità. La responsabilità di mantenere e aggiornare la libreria spetta al suo sviluppatore, mentre noi ci concentriamo sull'utilizzo delle funzionalità di pre-autorizzazione disponibili.

\subsection{Capitolato}
Il capitolato richiede una verifica delle richieste, e il prodotto finale soddisferà tale requisito. Tuttavia, nel Proof of Concept (PoC) iniziale, questa funzionalità non sarà presente.
La logica del riutilizzo del PoC è sicuramente una scelta appropriata. Consigliato di identificare inizialmente una serie di componenti e poi cercare implementazioni singole di tali funzionalità, che successivamente verranno assemblate per creare il prodotto finale.
L'approccio consigliato consiste nel cercare le funzionalità richieste, identificare esempi specifici di implementazione per ognuna di esse e fermarsi a quel punto. È fondamentale stabilire obiettivi chiari e adottare un taglio deciso nell'implementazione, riducendo al minimo le componenti superflue.