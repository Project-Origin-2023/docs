\section{Ordine del giorno}
\begin{enumerate}
\item File di configurazione idp-config.json e verifier-config.json 
\item Store di una credenziale
\item Controllo di flusso
\end{enumerate}

\subsection{File di configurazione idp-config.json e verifier-config.json} 
Con questi 2 file di configurazione ci riferiamo alla parte di verifica delle credenziali. Nello specifico idp-config.json va ad effettuare la mappatura tra le verifiable credential e i profili interni. Questo passaggio serve agli autenticatori per effettuare il mapping tra i dati. 
Questo meccanismo risulta fondamentale nella comunicazione tra wallet e verifier. Noi da wallet offriamo una verificabile id a un verifier e lui ci fornisce un access token per accedere ai suoi servizi. Il problema riscontrato da parte nostra è l'ottenimento di tale token. 
Da parte dell'azienda ci è stato consigliato inoltre di utilizzare dei DID di tipo key in ogni campo dati che necessita di DID sia esso qualsiasi attore del sistema. 
Discutendo con l'azienda inoltre si è giunti alla conclusione che durante l'issuing di una credenziale il campo DID del verifier fa rimeriemnto all'issuer in quanto verificatore dei documenti immessi al momento di richiesta credenziale.

\subsection{Store di una credenziale}
Discutendo con l'azienda si è giunti alla conclusione che lo store della credenziale non va eseguito su un unico repository fornito da SSI Kit. 
Abbiamo già creato 2 container:
\begin{itemize}
  \item walletssikit,
  \item issuing-Kit.
\end{itemize}

In questa maniera una volta istanziata una credenziale possiamo andarla a memorizzare nel repository wallet. 
Detto ciò rimane un po' nebbioso per il nostro gruppo tale procedura pertanto andremmo ad approfondirla nel breve periodo.
\subsection{Controllo di flusso}
Andremo a creare una simulazione del flusso, presente nella documentazione di waltid, tramite una collection di chiamate. 
Tale flusso verrà discusso assieme all'azienda al fine di chiarire dubbi da noi riscontrati su come strutturare i container docker per seguire il flow indicato in openid.


