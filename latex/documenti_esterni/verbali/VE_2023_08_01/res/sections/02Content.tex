\section{Ordine del giorno}
\begin{enumerate}
\item Esposizione POC con azienda proponente Infocert
\item Problemi con walt.id
\item Link rilasciatici da Infocert
\end{enumerate}

\subsection{Esposizione POC con azienda proponente Infocert} 
Il gruppo, il 1 agosto 2023, ha esposto la POC realizzata. Durante la discussione sono emersi consigli e sono stati risolti i dubbi.




\subsection{Problemi con walt.id}
Walt.id afferma di avere il protocollo attivo (in parte è vero).
Se andiamo sul sito openID, wallt.id è un client testato, ma soltanto il client, non il wallet, quindi solo una parte è testata.
Il problema è che loro in parte hanno implementato il protocollo, ma solo lato client.
Il client id è un mock, non riusciamo a impostarlo perché non ci sono tutti i pezzi (da completare), manca la parte di autorizzazione iniziale.
Ci sono due strade: adottiamo la strada di usare il loro mock e fare il giro come se fosse un'autenticazione vera.
Altrimenti, l'altra strada è utilizzare un'altra libreria per questa parte, ma non sappiamo se sia compatibile con i tempi (consigliata la prima strada).
Consigliato lavorare di incapsulamento.
Simulare le chiamate con valori mockati, tutto fisso. Esempio emissione token, la prima chiamata prevede anche il passaggio delle credenziali del client id, ma non funziona.
Scriviamo il codice e aggiungiamo commenti. In questo modo, quando la libreria sarà completata e funzionante, sarà pronta all'uso.

\subsection{Link rilasciatici da Infocert}
\begin{itemize}
\item \textbf{OpenID for Verifiable Credentials - Libraries:} \url{https://openid.net/sg/openid4vc/libraries/}
\item \textbf{Public Roadmap | walt.id:} \url{https://walt-id.notion.site/fcde1687baab42378b3047d4a22eeaca?v=1140dd17c17b4726a70cc1465d20866d}
\item \textbf{OID4VCI:} \url{https://github.com/Sphereon-Opensource/OID4VCI}
\end{itemize}
