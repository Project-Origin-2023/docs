\section{Binding}
\subsection{Breve descrizione}
Il termine "Binding" si riferisce al momento in cui l'User Agent viene reindirizzato da un portale all'altro durante l'autenticazione. 
In particolare, questo avviene quando il Service Provider reindirizza l'utente all'Identity Provider e, una volta che l'Identity 
Provider ha verificato l'identità dell'utente, lo reindirizza nuovamente al Service Provider per completare il processo\glo\ di 
autenticazione.
\subsection{Binding HTTP Redirect}
Il Service Provider invia allo User Agent un messaggio HTTP di redirezione, cioè avente uno Status Code con
valore 302 (“Found”) o 303 (“See Other”);
\\ Il Location Header del messaggio HTTP contiene l’URI\glo\ di destinazione del servizio di Single
Sign-On esposto dall’ Identity Provider.
\\ Il Pacchetto HTTP trasporta i parametri tutti URL-encoded codificato in formato
Base64 e compresso con algoritmo DEFLATE\glo.
\\ Il messaggio all'interno è la risorsa richiesta originaria a cui 
trasferire il controllo una volta terminata l'autenticazione, 
algoritmo e firma per la codifica delle informazioni.
\\ Una volta avute queste informazioni il User Agent fa una 
richiesta GET all' Identity Provider con tutte le informazioni 
sopracitate sotto forma di URLENCODED\glo.

\subsection{Binding HTTP POST}
Il Service Provider invia allo User Agent un messaggio HTTP con uno Status code\glo\ avente valore 200 (“OK”). 
Questo messaggio HTTP contiene un form HTML codificato come valore di un elemento nascosto del form. 
L'utilizzo di questa metodologia consente di superare i limiti di dimensione della query string. \\
L’intero messaggio SAML in formato XML può essere firmato tramite la XML Digital Signature\glo\ e il risultato viene codificato 
in formato Base64. La risorsa richiesta originariamente è inclusa nel messaggio e viene utilizzata per trasferire il controllo 
a termine dell'autenticazione.\\
Infine, il browser dell’utente elabora la risposta HTTP e invia una richiesta HTTP POST verso il componente Single Sign-On 
dell’Identity Provider.
