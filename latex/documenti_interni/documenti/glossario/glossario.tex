\documentclass[a4paper]{article}
\usepackage[normalem]{ulem}
\usepackage{eurosym}
\usepackage[font=small,labelfont=bf]{caption}

% impostazioni generali
\input{../../../template/package.tex}
%versioni documenti
\newcommand{\AdRversione}{1.0.0}
\newcommand{\NdPversione}{1.0.0}
\newcommand{\PdPversione}{1.0.1}
\newcommand{\PdQversione}{1.0.0}
\newcommand{\Gloversione}{1.0.0}
\newcommand{\Stversione}{0.1.2}
\newcommand{\MUversione}{0.1.0}
%nome documento con versione 
\newcommand{\AdRdocumento}{\textit{Analisi dei Requisiti v.\AdRversione} }
\newcommand{\NdPdocumento}{\textit{Norme di Progetto v.\NdPversione} }
\newcommand{\PdPdocumento}{\textit{Piano di Progetto v.\PdPversione} }
\newcommand{\PdQdocumento}{\textit{Piano di Qualifica v.\PdQversione} }
\newcommand{\Glodocumento}{\textit{Glossario v.\Gloversione} }
\newcommand{\Stdocumento}{\textit{Glossario v.\Stversione} }
\newcommand{\MUDocumento}{\textit{Glossario v.\MUversione} }

% dati della prima pagina 
% Configurazione della pagina iniziale
\newcommand{\doctitle}{\textit{Verbale} interno del 11 maggio 2023}
\newcommand{\rev}{0.1.0} % versione
\newcommand{\resp}{Beschin Michele} % inserire responsabile
\newcommand{\red}{Andreetto Alessio} % cognome nome dei redattori
\newcommand{\ver}{Lotto Riccardo} % cognome nome del verificatore
\newcommand{\uso}{Interno} %interno o esterno
\newcommand{\dest}{\textit{ProjectOrigin}
	\\ Prof. Vardanega Tullio 
	\\ Prof. Cardin Riccardo}
\newcommand{\describedoc}{\textit{Verbale} riguardante il meeting tenuto il 11 maggio 2023}

 % per modificare la prima pagina editare questo file

\makeindex

\makeatletter
\renewcommand\paragraph{
\@startsection {paragraph}{4}{0mm}{-\baselineskip}{.5\baselineskip}{\normalfont \normalsize \bfseries }}
\makeatother

\begin{document}

% Prima pagina
\thispagestyle{empty}
\renewcommand{\arraystretch}{1.3}

\begin{titlepage}
	\begin{center}
		
	\includegraphics[scale = 0.20]{../../../template/images/logo.png}
	\\[1cm]
	\href{mailto:projectorigin2023@gmail.com}		      	
	{\large{\textit{projectorigin2023@gmail.com} } }\\[2.5cm]
	\Huge \textbf{\doctitle} \\[1cm]
	 \large
			 \begin{tabular}{r|l}
                        \textbf{Versione} & \rev{} \\
                        \textbf{Responsabile} & \resp{} \\
                        \textbf{Redattori} & \red{} \\ 
                        \textbf{Verificatori} &  \ver{} \\
                        \textbf{Uso} & \uso{} \\                        
                        \textbf{Destinatari} & \parbox[t]{5cm}{ \dest{} }
                \end{tabular} 
                \\[3.3cm]
                \large \textbf{Descrizione} \\ \describedoc{} 
     \end{center}
\end{titlepage}

% Diario delle modifiche
\input{../../../template/changelog.tex}
\changelogTable{
0.3.0 & 2023-05-10 & Corbu Teodor & Verificatore & Verifica documento \\
0.2.1 & 2023-05-10 & Ibra Elton & Analista & Stesura sottocapitoli dei \S\ Casi d'uso \\
0.2.0 & 2023-05-09 & Corbu Teodor & Verificatore & Verifica documento \\
0.1.1 & 2023-05-09 & Ibra Elton & Analista & Inizio stesura \S\ Casi d'uso \\  
0.1.0 & 2023-05-04 & Ibra Elton & Verificatore & Verifica documento\\    
0.0.3 & 2023-05-03 & Corbu Teodor & Analista & Stesura \S\ Descrizione Generale \\
0.0.2 & 2023-05-03 & Corbu Teodor & Analista & Stesura \S\ Introduzione \\
0.0.1 & 2023-05-02 & Corbu Teodor & Analista & Creazione struttura documento\\
} % editare questo
\pagebreak


% Indice
{
    \hypersetup{linkcolor=black}
    \tableofcontents
    %\listoffigures %elenco figure
    %\listoftables  %elenco tabelle
}
\pagebreak

% sezioni comuni 
%si può aggiungere una sezione dei riferimenti, dove si citano il capitolato, wikipedia, treccani e i libri di SWE
\section{A}
\textbf{Analista}: Persona che conosce il dominio del problema e definisce i requisiti espliciti e impliciti. Si occupa di redigere il documento Analisi dei Requisiti.\\
\textbf{Asincrona}: Modalità di svolgimento delle attività in cui i partecipanti si scambiano informazioni senza dover comunicare in tempo reale.\\
\textbf{Attore}: L'attore rappresenta un ruolo o una responsabilità in un determinato scenario di utilizzo di un sistema software e viene identificato durante l'analisi dei requisiti e nella modellizzazione dei casi d'uso. L'attore può essere sia principale, ovvero l'attore principale che utilizza il sistema, sia secondario, ovvero un attore che supporta l'attore principale nell'utilizzo del sistema o fornisce informazioni o servizi al sistema stesso.\\

\pagebreak
\section{B}
\textbf{Base64}: È un sistema di codifica che consente la traduzione di dati binari in stringhe di testo ASCII, rappresentando i dati sulla base di 64 caratteri ASCII diversi. Viene usato principalmente come codifica di dati binari nelle e-mail, per convertire i dati nel formato ASCII.
\pagebreak
\section{C}
\textbf{Capitolato}: Un capitolato d’appalto è un documento del committente che specifica cosa richiede che
sia presente nel prodotto e i suoi vincoli.\\
\textbf{Caso d'uso}: Il caso d'uso in informatica è una tecnica usata nei processi di ingegneria del software per effettuare in maniera esaustiva e non ambigua, la raccolta dei requisiti al fine di produrre software di qualità.\\


\pagebreak
\section{D}
\textbf{Deflate}: Deflate (stilizzato come DEFLATE) è un algoritmo per la compressione dati senza perdita che è stato introdotto dal programma PKZIP, e quindi formalizzato nella RFC 1951. È tuttora ampiamente utilizzato per le sue ottime prestazioni e l'assenza di brevetti.
\textbf{Discord}: Piattaforma gratuita che fornisce servizi di chat vocale, testuale e video tra singoli membri o in server dedicati.\\

\pagebreak
\section{E}
\textbf{Efficacia}: Con efficacia si intende la misura della capacità di raggiungere un obiettivo; è strettamente
legato a quanto ciò che viene fatto rispetta i requisiti.\\
\textbf{Efficienza}: Con efficienza di intende la misura per cui si impiegano il minimo numero di risorse per
raggiungere un obiettivo.\\
\textbf{Endpoint}: Un endpoint è un punto di accesso specifico in un'applicazione o un sistema informatico, spesso utilizzato per interagire con le risorse o i servizi. Gli endpoint possono essere URL in una API web o indirizzi IP in una rete. Forniscono un modo strutturato per comunicare con un'applicazione o una risorsa tramite richieste e risposte.\\
\pagebreak
%\section{F}
\textbf{FOAF (Friend of a Friend)}: FOAF (Friend of a Friend) è un vocabolario RDF (Resource Description Framework) che fornisce una rappresentazione semantica delle informazioni sociali delle persone e delle relazioni tra di loro. FOAF è stato progettato per consentire la creazione di profili sociali interoperabili e per facilitare il collegamento delle informazioni personali tra diverse applicazioni e reti sociali.
%\pagebreak
\section{G}
\textbf{Git}: sistema di controllo di versione distribuito che tiene traccia dei cambiamenti nei file
\pagebreak
\section{H}
\textbf{Holder}: Gli utenti che raccolgono credenziali da diverse fonti e le conservano nel loro portafoglio di identità. Il portafoglio di identità può essere un servizio ospitato o un'applicazione eseguita su un dispositivo dell'utente.\\

\pagebreak
\section{I}
\textbf{Issuer}: Istituzioni che rilasciano credenziali ai Holder (ad esempio, UniPD).\\

\pagebreak
\section{J}
\textbf{JSON}: JSON, acronimo di JavaScript Object Noatation, è una formula adatta all'interscambio di dati fra applicazioni client/server.\\
\textbf{JSON-LD}: JSON-LD (JavaScript Object Notation for Linked Data) è un formato di serializzazione dei dati che consente di rappresentare informazioni strutturate nel formato JSON (JavaScript Object Notation) arricchito con elementi semantici per il web semantico. JSON-LD è progettato per consentire l'interoperabilità e l'integrazione dei dati nel contesto del web delle risorse collegate (Linked Data).


\pagebreak
\section*{K}
\textbf{Keep (Google)}: È un servizio di Google per prendere annotazioni.
\pagebreak
%\section{L}
\textbf{Linking Open Data}: Linking Open Data è un'iniziativa che mira a creare un web di dati collegati utilizzando principi come l'apertura, la condivisione e la connessione dei dati in modo interconnesso. L'obiettivo principale di Linking Open Data è rendere disponibili e collegare insieme diversi dataset aperti e pubblicamente accessibili su Internet utilizzando tecnologie come RDF (Resource Description Framework) e SPARQL (SPARQL Protocol and RDF Query Language).
%\pagebreak
\section*{M}
\textbf{MaterialUI}: Una libreria di componenti React che segue le linee guida di Material Design, permettendo agli sviluppatori di creare facilmente interfacce basate su questo stile\\
\pagebreak
%\section{O}
\textbf{OWL(Web Ontology Language)}: OWL (Web Ontology Language) è un linguaggio standard del World Wide Web Consortium (W3C) per la rappresentazione di ontologie nel contesto del web semantico. Le ontologie sono modelli di conoscenza formali che descrivono concetti, relazioni e vincoli all'interno di un dominio specifico.
%\pagebreak
\section{P}
\textbf{Processo}: Insieme delle attività correlate e coese che trasformano i bisogni in prodotti (il risultato di un processo si
chiama prodotto). Opera secondo regole consumando risorse.\\
\textbf{Progettista}: Si occupa di definire l'architettura del sistema alla base del prodotto software. Segue la fase dello sviluppo del prodotto.
\pagebreak
\section*{Q}
\textbf{Qualità}: Insieme delle caratteristiche di un'entità, che ne determinano la capacità di soddisfare esigenze sia espresse che implicite.\\
\pagebreak
\section{R}
\textbf{Repository}: Archivio centralizzato dove vengono memorizzate le informazioni e i dati in formato digitale sulla base di metadati che ne permettono la rapida individuazione.\\
\textbf{Responsabile}: Ha il compito di pianificare le attività, coordinare e controllare tutti i membri del team.
Si occupa anche di approvare i documenti e rappresenta il team presso l'azienda proponente.\\

\pagebreak
\section{S}
\textbf{ISO/IEC 9126}: Con la sigla ISO/IEC 9126 si individua una serie di normative e linee guida. Il modello propone un approccio alla qualità in modo tale che le società di software possano migliorare l'organizzazione e i processi e, quindi come conseguenza concreta, la qualità del prodotto sviluppato.\\

\pagebreak
\section{T}
\textbf{Teams (Microsoft)}: Piattaforma di comunicazione e collaborazione unificata che combina chat di lavoro persistente, teleconferenza, condivisione di contenuti.\\
\textbf{Telegram}: Servizio di messaggistica istantanea e broadcasting basato su cloud.\\
\textbf{TLS}: Transport Layer Security (TLS) e il suo predecessore Secure Sockets Layer (SSL) sono dei protocolli crittografici di presentazione usati nel campo delle telecomunicazioni e
dell'informatica che permettono una comunicazione sicura dalla sorgente al destinatario (end-to-end) su reti TCP/IP (come ad esempio Internet) fornendo autenticazione, 
integrità dei dati e confidenzialità operando al di sopra del livello di trasporto.

\pagebreak
\section{U}
\textbf{URI}: In informatica, lo Uniform Resource Identifier è una sequenza di caratteri che identifica universalmente ed univocamente una risorsa. Sono esempi di URI: un indirizzo web, un documento, un indirizzo di posta elettronica, il codice ISBN di un libro, un numero di telefono col prefisso internazionale.\\
\textbf{UML, diagrammi}: UML è un linguaggio di modellazione e di specifica basato sul paradigma orientato agli oggetti.
 La notazione UML è semi-grafica e semi-formale; un modello UML è costituito da una collezione organizzata di diagrammi correlati, costruiti componendo elementi grafici (con significato
formalmente definito), elementi testuali formali, ed elementi di testo libero. Ha una semantica molto
precisa e un grande potere descrittivo.\\
\pagebreak
\section{V}
\textbf{Verifica}: Accertamento che l’esecuzione delle attività di processi svolti nella fase in esame non causino errori.\\
\textbf{Verificatore}: È presente per l'intera durata del progetto e si occupa di svolgere le attività di Verifica e Validazione.\\
\textbf{Verifier}: Entità interessate a consumare credenziali (ad esempio, una banca online che chiede le credenziali di registrazione universitaria per offrire un conto studente). Le credenziali fornite a un verificatore da un titolare possono essere confezionate in "presentazioni verificabili".\\
\pagebreak
\section{W}
\textbf{Wallet}: Applicazione in cui l'utente memorizza credenziali che può utilizzare presso dei provider.\\ %forse servirà anche la definizione di provider
\textbf{Web App}: Applicazione fruibile via web per mezzo di un network, come Internet, che offre determinati servizi all'utente. Una web app non necessita di essere installata.\\
\textbf{W3C Data Model}: Il "W3C Data Model" è un modello di dati definito dal World Wide Web Consortium (W3C), un'organizzazione che sviluppa standard per il World Wide Web. Il modello di dati del W3C è progettato per rappresentare informazioni strutturate in modo interoperabile e standardizzato, consentendo la condivisione e lo scambio di dati tra diverse applicazioni e piattaforme web.\\
\textbf{World Wide Web Consortium (W3C)}: Il World Wide Web Consortium (W3C) è un'organizzazione internazionale che si occupa dello sviluppo di standard aperti per il World Wide Web. Fondata nel 1994 da Tim Berners-Lee, il creatore del Web, e con sede presso il Massachusetts Institute of Technology (MIT) negli Stati Uniti, il W3C lavora per stabilire linee guida e specifiche tecniche che promuovono l'interoperabilità, l'accessibilità e l'evoluzione del Web.\\
\textbf{Web of Things (WoT)}: Web of Things (WoT) è un concetto che si riferisce alla connessione e all'interoperabilità delle cose fisiche e degli oggetti intelligenti tramite il Web. WoT mira a estendere i principi del World Wide Web per consentire alle cose di essere integrate nel contesto del Web e di comunicare tra di loro utilizzando standard aperti e protocolli web. L'obiettivo principale di WoT è consentire agli oggetti di essere facilmente accessibili e controllabili attraverso il Web, in modo simile a come le pagine web sono accessibili agli utenti.
\pagebreak
\section*{X}
\textbf{XML Digital Signature}: La firma XML definisce una sintassi XML per le firme digitali ed è definita nella raccomandazione W3C Sintassi ed elaborazione della firma XML. Funzionalmente, ha molto in comune con PKCS \#7, ma è più estensibile e orientato alla firma di documenti XML. Viene utilizzato da varie tecnologie Web come SOAP, SAML e altre.\\
\pagebreak
\section*{Z}
\textbf{Zoom}: Piattaforma di comunicazione che combina chat di lavoro persistente, teleconferenza, telelavoro, formazione a distanza e relazioni sociali.\\



\end{document}
