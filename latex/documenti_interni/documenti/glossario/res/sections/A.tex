\section{A}
\textbf{Analista}: Persona che conosce il dominio del problema e definisce i requisiti espliciti e impliciti. Si occupa di redigere il documento Analisi dei Requisiti.\\
\textbf{API}: Un insieme di subroutine o di funzioni che un programma, oppure un'applicazione, possono richiamare al fine di chiedere al sistema operativo
di svolgere un determinato compito. Le API di Windows consistono di oltre mille funzioni richiamabili da programmi scritti in C, C++,
Pascal e in altri linguaggi al fine di creare finestre, di aprire file e di svolgere qualche altra funzione essenziale. Ad esempio, un'applicazione 
che voglia visualizzare un messaggio sullo schermo può richiamare la funzione API di Windows chiamata MessageBox.\\
\textbf{Asincrona}: Modalità di svolgimento delle attività in cui i partecipanti si scambiano informazioni senza dover comunicare in tempo reale.\\
\textbf{Attore}: L'attore rappresenta un ruolo o una responsabilità in un determinato scenario di utilizzo di un sistema software e viene identificato durante l'analisi dei requisiti e nella modellizzazione dei casi d'uso. L'attore può essere sia principale, ovvero l'attore principale che utilizza il sistema, sia secondario, ovvero un attore che supporta l'attore principale nell'utilizzo del sistema o fornisce informazioni o servizi al sistema stesso.\\
