\section{H}
\textbf{Hash}: Un hash in crittografia è una funzione matematica che trasforma un input (come un testo o dati) in una stringa di lunghezza fissa, chiamata hash, che rappresenta in modo univoco l'input originale. Questo processo è unidirezionale, il che significa che è difficile ottenere l'input originale da un hash. Gli hash sono ampiamente utilizzati per garantire l'integrità dei dati e la sicurezza delle informazioni.\\
\textbf{Holder}: Gli utenti che raccolgono credenziali da diverse fonti e le conservano nel loro portafoglio di identità. Il portafoglio di identità può essere un servizio ospitato o un'applicazione eseguita su un dispositivo dell'utente.\\
\textbf{HTTP redirect}: Il reindirizzamento URL, noto anche come inoltro URL, è una tecnica per fornire più di un indirizzo URL a una pagina, un modulo, un intero sito Web o un'applicazione Web. 
HTTP ha un tipo speciale di risposta, chiamato HTTP redirect, per questa operazione.\\
\textbf{HTTP response}: Un\textit{ HTTP response} viene effettuata da un server a un client. Lo scopo della risposta è quello di fornire al client la risorsa richiesta,
o informare il client che l'azione richiesta è stata eseguita; oppure per informare il client che si è verificato un errore nell'elaborazione della sua richiesta.
Una risposta HTTP contiene:
\begin{itemize}
    \item Una linea di stato.
    \item Una serie di intestazioni HTTP o campi di intestazione.
    \item Un corpo del messaggio, che di solito è necessario.
\end{itemize}
