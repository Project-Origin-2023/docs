\section{S}

\textbf{SAML Response}: Acronimo di Security Assertion Markup Language è uno statndard informatico per lo scambio di dati di autenticazione e autorizzazione tra domini di sicurezza distinti. In particolare SAML Response viene inviata dall'identity provider al service provider, se l'utente ha avuto successo nel processo di autenticazione, contiene l'asserzione con il NameID / gli attributi dell'utente.\\
\textbf{SAML v2}: SAML 2.0 è un protocollo basato su XML che utilizza token di sicurezza contenenti asserzioni per passare informazioni su un principale (di solito un utente finale) tra un'autorità SAML, denominata Identity Provider, e un consumatore SAML, denominato Service Provider. SAML 2.0 abilita il single sign-on (SSO) cross-domain basato sul Web, che aiuta a ridurre il sovraccarico amministrativo della distribuzione di più token di autenticazione all'utente.\\
\textbf{SSL}: Un protocollo e un metodo di crittografia proposto da Netscape per proteggere le informazioni che circolano su Internet. 
Definisce i meccanismi di trasporto delle informazioni tra un browser e un server Web al fine di eseguire transazioni sicure su Internet. 
Le funzioni base sono la cifratura dei dati, la loro convalida (con l'aggiunta di altri dati cifrati di riscontro) e l'autenticazione della 
fonte (mediante l'aggiunta ai dati stessi di una firma digitale). Assieme a SHTTP (Secure HTTP) è uno dei due standard di sicurezza che vengono 
utilizzati su Internet. Per cifrare i dati, SSL utilizza il sistema di chiave pubblica e chiave privata definito dalla RSA dove si chiede che il server 
disponga di una coppia unica di chiavi tra loro correlate matematicamente e che vengono utilizzate per iniziare ciascuna transazione. 
Affinchè la comunicazione possa aver luogo, il client deve possedere un file di riconoscimento che viene distribuito dall'autorità di certificazione, 
che può essere interna oppure esterna all'azienda. Nel file di riconoscimento è inserito il nome del certificatore e una chiave radice pubblica abbinata 
univocamente a quel particolare server. Quando il client contatta il server, questo risponde fornendo i propri dati d'identificazione e se questi 
coincidono con quanto registrato nel file di certificazione è possibile aprire una sessione sicura in modalità SSL. La stessa procedura vale nei 
confronti del client quando anche questo deve essere certificato. Il file di riconoscimento si chiama keyring e contiene la chiave pubblica e privata 
del proprietario e una o più certificazioni. Il keyring può essere autocertificato quando non si vuole ricorrere a un'autorità esterna di certificazione, 
ma in questo caso il livello di sicurezza è inferiore.\\
\textbf{Status Code}: Gli status code (o response codes) indicano se una specifica richiesta HTTP è stata completata correttamente. Sono parte integrante del protocollo HTTP (acronimo di Hypertext Transfer Protocol), il protocollo usato da client e server per comunicare e scambiare informazioni.\\
