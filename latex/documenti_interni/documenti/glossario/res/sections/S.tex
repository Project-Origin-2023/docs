\section{S}
\textbf{SAML Response}: Acronimo di Security Assertion Markup Language è uno statndard informatico per lo scambio di dati di autenticazione e autorizzazione tra domini di sicurezza distinti. In particolare SAML Response viene inviata dall'identity provider al service provider, se l'utente ha avuto successo nel processo di autenticazione, contiene l'asserzione con il NameID / gli attributi dell'utente.\\
\textbf{SAML v2}: SAML 2.0 è un protocollo basato su XML che utilizza token di sicurezza contenenti asserzioni per passare informazioni su un principale (di solito un utente finale) tra un'autorità SAML, denominata Identity Provider, e un consumatore SAML, denominato Service Provider. SAML 2.0 abilita il single sign-on (SSO) cross-domain basato sul Web, che aiuta a ridurre il sovraccarico amministrativo della distribuzione di più token di autenticazione all'utente.\\
\textbf{Status Code}: Gli status code (o response codes) indicano se una specifica richiesta HTTP è stata completata correttamente. Sono parte integrante del protocollo HTTP (acronimo di Hypertext Transfer Protocol), il protocollo usato da client e server per comunicare e scambiare informazioni.