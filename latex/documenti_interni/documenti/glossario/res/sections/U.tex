\section{U}
\textbf{URI}: In informatica, lo Uniform Resource Identifier è una sequenza di caratteri che identifica universalmente ed univocamente una risorsa. Sono esempi di URI: un indirizzo web, un documento, un indirizzo di posta elettronica, il codice ISBN di un libro, un numero di telefono col prefisso internazionale.\\
\textbf{UML, diagrammi}: UML è un linguaggio di modellazione e di specifica basato sul paradigma orientato agli oggetti.
 La notazione UML è semi-grafica e semi-formale; un modello UML è costituito da una collezione organizzata di diagrammi correlati, costruiti componendo elementi grafici (con significato
formalmente definito), elementi testuali formali, ed elementi di testo libero. Ha una semantica molto
precisa e un grande potere descrittivo.\\
\textbf{URI (Uniform Resource Identifier)}: URI (Uniform Resource Identifier) è una stringa di caratteri che identifica in modo univoco una risorsa su Internet, come ad esempio un documento, un'immagine o un servizio web. Gli URI sono utilizzati per creare collegamenti tra le risorse, consentendo agli utenti di accedere alle informazioni in modo semplice e strutturato.