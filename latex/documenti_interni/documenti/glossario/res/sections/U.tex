\section{U}
\textbf{UI}: User Interface (in italiano Interfaccia Utente). E' metodo di comunicazione tra un utente e le funzionalità del backend attraverso elementi interattivi come pulsanti, menu e finestre di vario genere.\\
\textbf{UML, diagrammi}: UML è un linguaggio di modellazione e di specifica basato sul paradigma orientato agli oggetti.
 La notazione UML è semi-grafica e semi-formale; un modello UML è costituito da una collezione organizzata di diagrammi correlati, costruiti componendo elementi grafici (con significato
formalmente definito), elementi testuali formali, ed elementi di testo libero. Ha una semantica molto
precisa e un grande potere descrittivo.\\
\textbf{URI}: In informatica, lo Uniform Resource Identifier è una sequenza di caratteri che identifica universalmente ed univocamente una risorsa. Sono esempi di URI: un indirizzo web, un documento, un indirizzo di posta elettronica, il codice ISBN di un libro, un numero di telefono col prefisso internazionale.\\
\text{URL encoded} \textbf{UML, diagrammi}: UML è un linguaggio di modellazione e di specifica basato sul paradigma orientato agli oggetti.
La notazione UML è semi-grafica e semi-formale; un modello UML è costituito da una collezione organizzata di diagrammi correlati, costruiti componendo elementi grafici (con significato
formalmente definito), elementi testuali formali, ed elementi di testo libero. Ha una semantica molto
precisa e un grande potere descrittivo.\\
