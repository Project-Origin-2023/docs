\section{W}
\textbf{Wallet}: Applicazione in cui l'utente memorizza credenziali che può utilizzare presso dei provider.\\ %forse servirà anche la definizione di provider
\textbf{Web App}: Applicazione fruibile via web per mezzo di un network, come Internet, che offre determinati servizi all'utente. Una web app non necessita di essere installata.\\
\textbf{W3C Data Model}: Il "W3C Data Model" è un modello di dati definito dal World Wide Web Consortium (W3C), un'organizzazione che sviluppa standard per il World Wide Web. Il modello di dati del W3C è progettato per rappresentare informazioni strutturate in modo interoperabile e standardizzato, consentendo la condivisione e lo scambio di dati tra diverse applicazioni e piattaforme web.\\
\textbf{World Wide Web Consortium (W3C)}: Il World Wide Web Consortium (W3C) è un'organizzazione internazionale che si occupa dello sviluppo di standard aperti per il World Wide Web. Fondata nel 1994 da Tim Berners-Lee, il creatore del Web, e con sede presso il Massachusetts Institute of Technology (MIT) negli Stati Uniti, il W3C lavora per stabilire linee guida e specifiche tecniche che promuovono l'interoperabilità, l'accessibilità e l'evoluzione del Web.\\
\textbf{Web of Things (WoT)}: Web of Things (WoT) è un concetto che si riferisce alla connessione e all'interoperabilità delle cose fisiche e degli oggetti intelligenti tramite il Web. WoT mira a estendere i principi del World Wide Web per consentire alle cose di essere integrate nel contesto del Web e di comunicare tra di loro utilizzando standard aperti e protocolli web. L'obiettivo principale di WoT è consentire agli oggetti di essere facilmente accessibili e controllabili attraverso il Web, in modo simile a come le pagine web sono accessibili agli utenti.