\section{Processi primari} 

\subsection{Fornitura} 
\subsubsection{Descrizione}
Date le richieste del proponente, i processi di fornitura hanno come scopo di determinare e descrivere le attività e i compiti che il fornitore si impegna a svolgere.
Le seguenti indicazioni dovranno essere seguite durante tutto il ciclo di vita del progetto didattico.

\subsubsection{Obiettivi} 
Gli obiettivi principali che si cerca di raggiungere definendo i processi di fornitura sono: 
\begin{enumerate}
    \item Evitare la creazione di eventuali dubbi durante il progetto; 
    \item definire i bisogni che il prodotto deve soddisfare; 
    \item calcolare i costi per la realizzazione del prodotto; 
    \item definire le procedure necessarie per gestire il progetto; 
    \item ottenere il responso del committente e del proponente sulle decisioni prese e sulle attività svolte.
\end{enumerate}
Una volta individuate le attività da svolgere, queste, insieme alle tempistiche, 
saranno definite nel documento \textit{Piano di Progetto}.


\subsubsection{Inizio dei lavori} 
La prima attività svolta è stata un'analisi dei capitolati delle aziende proponenti, lo studio di questi capitolati è stato esposto nei documenti \textit{Analisi dei capitolati}. \\
Per presentare la nostra candidatura il gruppo ha effettuato uno studio di fattibilità che 
prevede come informazioni principali la stima dei costi degli orari di ogni ruolo necessario, 
insieme alla stima dei costi complessiva e la definizione degli impegni di ogni membro del gruppo. \\
Una volta ricevuta l'aggiudicazione per il capitolato\glo scelto il gruppo ha iniziato a definire il proprio \textit{Way Of Working}. \\
Quest'ultimo è descritto attraverso questo documento, \textit{Norme di Progetto}. \\
Come obiettivo principale il gruppo ha deciso di descrivere inizialmente i Processi di Supporto ed Organizzativi in quanto sono 
importanti per definire gli strumenti e le metodologie necessarie per iniziare il lavoro. 


\subsubsection{Repository} 
Le attività da effettuare comprenderanno un lavoro condiviso e il codice prodotto sarà di pubblico dominio. \\
Pertanto il gruppo ha deciso di utilizzare una repository\glo pubblica su GitHub\glo, piattaforma conosciuta da tutti i membri del gruppo. \\
La visione dei documenti creati è pubblica e solo i membri del gruppo hanno i permessi necessari per contribuire all'interno del progetto. \\
I file che sono soggetti a versionamento includeranno all'interno della loro struttura la numerazione della versione corrente e un riepilogo 
delle versioni precedenti sotto forma di tabella. In questo modo, è possibile tenere traccia dei cambiamenti apportati nel tempo al file e 
fornire un'indicazione chiara della sua storia.
È stata creata un'organizzazione di nome Project Origin, i riferimenti sono \url{https://github.com/Project-Origin-2023}.
Al suo interno esiste la repo \textit{docs} contenente tutta la documentazione di interesse al proponente e al committente.

\subsubsection{Pianificazione}\label{pianificazione}
\paragraph{Piano Di Progetto}
Per favorire la pianificazione è stato creato il documento \textit{Piano Di Progetto} il quale descrive la pianificazione delle attività del gruppo.\\
L’obiettivo di questa attività è di fornire una stima dei costi e dei tempi di consegna del progetto,
nonchè un piano dettagliato corredato di orari per ogni ruolo coinvolto.\\
La pianificazione sarà basata sull’analisi periodica delle attività svolte e una retrospettiva su quelle attività
già completate, al fine di mantenere aggiornate e valide le stime effettuate.\\
Vengono inoltre identificate le possibili problematiche che il team potrebbe incontrare
durante l’intero periodo di svolgimento del progetto.
\paragraph{Gestione della Qualità}
Questa attività riguarda le strategie di verifica e validazione che il gruppo si impegna ad utilizzare.\\
Per garantire efficacia e qualità nei processi e nei prodotti il gruppo ha realizzato il documento \textit{Piano Di Qualifica} il quale descrive l’insieme delle risorse atte al perseguimento della qualità.\\
In particolare definisce le metodologie che il gruppo intende attuare per
garantire la qualità del prodotto per tutta la durata del progetto.\\
La scelta di queste metodologie è a carico dei progettisti.

\subsection{Sviluppo} 
\subsubsection{Scopo} 
Il processo\glo di sviluppo ha come obiettivo di descrivere le attività di analisi, di progettazione e di codifica per il prodotto software da sviluppare. \\
L'obiettivo più importante è la realizzazione del prodotto finale, attraverso l'individuazione dei requisiti del prodotto, 
la descrizione e l'individuazione degli eventuali vincoli da rispettare.


\subsubsection{Analisi dei requisiti} 
\paragraph{Scopo} 
L'attività di analisi dei requisiti si pone come obiettivo la creazione di un documento contenente tutti i requisiti richiesti dal proponente.
Lo scopo è di identificare i requisiti obbligatori e facilitare l'attività di pianificazione della mole di lavoro.\\
Gli Analisti hanno il compito di creare il documento di Analisi dei Requisiti per:
\begin{itemize}
    \item Definire lo scopo del lavoro;
    \item Fornire ai Progettisti indicazioni precise e affidabili;
    \item Stabilire le funzioni e le richieste concordate con il cliente;
    \item Offrire una base per miglioramenti successivi per garantire un costante avanzamento del prodotto e del processo di sviluppo;
    \item Agevolare la revisione del codice;
    \item Fornire ai Verificatori\glo indicazioni sui principali e alternativi casi d'uso per l'attività di test;
    \item Calcolare i costi del progetto.
\end{itemize}   

\paragraph{Casi d'uso}\label{Casi} 
I casi d'uso descrivono l'interazione tra il sistema e uno o più attori. Ogni caso d'uso è univoco, e si distingue con un codice che segue la seguente sintassi: "UC[Codice] - [Titolo]",
dove "UC" sta per "Use Case", "[Codice]" indica un numero intero univoco per ogni caso d'uso, e "[Titolo]" è il titolo del caso d'uso. 
Ogni caso d'uso può includere informazioni aggiuntive come:
\begin{itemize}
    \item Scenario principale
    \item Scenari alternativi
    \item Pre-condizioni
    \item Post-condizioni
    \item Trigger
    \item Attori principali
    \item Attori secondari
    \item Diagramma UML
\end{itemize}

\paragraph{Requisiti} 
I requisiti sono ricavati attraverso l’analisi del capitolato\glo insieme a successive discussioni con il proponente e attraverso l’analisi dei casi d’uso.
Possono essere individuati sia tramite discussioni approfondite all’interno del gruppo che attraverso discussioni con il proponente.
I requisiti possono essere categorizzati in diverse tipologie : 
\begin{itemize}
    \item \textbf{Funzionali}: Requisiti dati dalle funzionalità che il prodotto deve presentare;
    \item \textbf{Qualità}: Requisiti dati per descrivere la qualità che il prodotto deve avere;
    \item \textbf{Vincolo}: Requisiti dati da vincoli tecnologici e strettamente legati all’implementazione del prodotto.
\end{itemize}
I requisiti possono essere categorizzati in base a diversi gradi d'importanza:
\begin{itemize}
    \item \textbf{Desiderabile}: Rappresenta un requisito che apporta valore e miglioramenti al sistema, ma non è strettamente necessario per il suo funzionamento base;
    \item \textbf{Opzionale}: Se relativamente vantaggioso, sarà preso in considerazione in un momento successivo;
    \item \textbf{Obbligatorio}: Denota un requisito essenziale per il cliente che non può essere trascurato.
\end{itemize}

\paragraph{Codifica dei requisiti}

R[Tipologia][Numero]-[Importanza] \\
Di seguito sono elencate le sigle per la codifica dei requisiti: 
\begin{itemize}
    \item \textbf{F}: Funzionale; \textbf{N}: Non Funzionale; \textbf{Q}: Qualità; (\textbf{Tipologia})
    \item \textbf{UC}[codice]; (vedi paragrafo \ref{Casi}) (\textbf{Casi d'uso})
    \item \textbf{O}: Obbligatorio; \textbf{D}: Desiderabile; \textbf{P}: Opzionale. (\textbf{Importanza})
\end{itemize}

\subsubsection{Progettazione} 
I progettisti sono responsabili dell'attività di progettazione, la quale consiste nell'ideazione dell'architettura del sistema sulla base dei requisiti 
identificati nell'Analisi dei Requisiti. \\
Per raggiungere questo obiettivo, il processo di progettazione prevede la realizzazione di un Proof of Concept\glo 
per la \textbf{R}equirements and \textbf{T}echnology \textbf{B}aseline, seguito da una fase di approfondimento dei dettagli per la Product Baseline.

\subsubsection{Codifica}
\begin{itemize}
    \item \textbf{Indentazione}: I blocchi di codice innestati e le varie linee di codice dovranno rispettare una determinata tabulazione al fine di rendere il codice più leggibile.
    \item \textbf{Blocco di codice}: La delimitazione dei vari blocchi di codice tra parentesi graffe dovranno avere una struttura determinata dove le parentesi 
    graffe saranno poste nella seguente maniera:
    \begin{itemize}
        \item \textbf{Parentesi di apertura}: Posta a distanza di 1 spazio rispetto alla dichiarazione del costrutto ma sempre nella stessa linea.
        \item \textbf{Parentesi di chiusura}: Posta nella linea successiva dell’ultima riga di codice del blocco a cui si fa riferimento e indentata secondo il blocco. 
    \end{itemize}
    \item \textbf{Commenti}: I commenti saranno inseriti in lingua italiana, saranno indispensabili nelle porzioni di codice dove non è immediata la comprensibile. Ogni qualvolta il codice 
    venga aggiornato va ricontrollata la validità dei commenti ad esso associato.
    \item \textbf{Nomi univoci}: È necessario utilizzare nomi univoci e significativi per ogni costruttore al fine di garantire la chiarezza e la facilità di identificazione delle diverse entità.
\end{itemize}









