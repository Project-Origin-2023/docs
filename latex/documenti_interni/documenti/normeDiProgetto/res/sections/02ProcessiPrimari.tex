\section{Processi primari} 

\subsection{Fornitura} 
\subsubsection{Descrizione}
Date le richieste del proponente, i processi di fornitura hanno come scopo di determinare e descrivere le attività e i compiti che il fornitore si impegna a svolgere.
Le seguenti indicazioni dovranno essere seguite durante tutto il ciclo di vita del progetto didattico.

\subsubsection{Obiettivi} 
Gli obiettivi principali che si cerca di raggiungere definendo i processi di fornitura sono: 
\begin{enumerate}
    \item evitare la creazione di eventuali dubbi durante il progetto; 
    \item definire i bisogni che il prodotto deve soddisfare; 
    \item calcolare i costi per la realizzazione del prodotto; 
    \item definire le procedure necessarie per gestire il progetto; 
    \item ottenere il responso del committente e del proponente sulle decisioni prese e sulle attività svolte.
\end{enumerate}
Una volta individuate le attività da svolgere, queste, insieme alle tempistiche, 
saranno definite nel documento \textit{Piano di Progetto}.


\subsubsection{Inizio dei lavori} 
La prima attività svolta è stata un'analisi dei capitolati delle aziende proponenti, lo studio di questi capitolati è stato esposto nei documenti \textit{Analisi dei capitolati}. \\
Per presentare la nostra candidatura il gruppo ha effettuato uno studio di fattibilità che 
prevede come informazioni principali la stima dei costi degli orari di ogni ruolo necessario, 
insieme alla stima dei costi complessiva e la definizione degli impegni di ogni membro del gruppo. \\
Una volta ricevuta l'aggiudicazione per il capitolato\glo scelto il gruppo ha iniziato a definire il proprio \textit{Way Of Working}. \\
Quest'ultimo è descritto attraverso questo documento \textit{Norme di Progetto}. \\
Come obiettivo principale il gruppo ha deciso di descrivere inizialmente i Processi di Supporto ed Organizzativi in quanto sono 
importanti per definire gli strumenti e le metodologie necessarie per iniziare il lavoro. 


\subsubsection{Repository} 
Le attività da effettuare comprenderanno un lavoro condiviso e il codice prodotto sarà di pubblico dominio. \\
Pertanto il gruppo ha deciso di utilizzare una repository\glo pubblica su GitHub\glo, piattaforma conosciuta da tutti i membri del gruppo. \\
La visione dei documenti creati è pubblica e solo i membri del gruppo hanno i permessi necessari per contribuire all'interno del progetto. \\
I file che sono soggetti a versionamento includeranno all'interno della loro struttura la numerazione della versione corrente e un riepilogo 
delle versioni precedenti sotto forma di tabella. In questo modo, è possibile tenere traccia dei cambiamenti apportati nel tempo al file e 
fornire un'indicazione chiara della sua storia.
È stata creata un'organizzazione di nome Project Origin. I riferimenti sono \url{https://github.com/Project-Origin-2023}.
Al suo interno esiste la repo \textit{docs} contenente tutta la documentazione di interesse al proponente e al committente.

\subsubsection{Pianificazione}\label{pianificazione}
\paragraph{Piano Di Progetto}
Per favorire la pianificazione è stato creato il documento \textit{Piano Di Progetto} il quale descrive la pianificazione delle attività del gruppo.\\
L’obiettivo di questa attività è di fornire una stima dei costi e dei tempi di consegna del progetto,
nonchè un piano dettagliato corredato di orari per ogni ruolo coinvolto.\\
La pianificazione sarà basata sull’analisi periodica delle attività svolte e una retrospettiva su quelle attività
già completate, al fine di mantenere aggiornate e valide le stime effettuate.\\
Vengono inoltre identificate le possibili problematiche che il team potrebbe incontrare
durante l’intero periodo di svolgimento del progetto.
\paragraph{Gestione della Qualità}
Questa attività riguarda le strategie di verifica e validazione che il gruppo si impegna ad utilizzare.\\
Per garantire efficacia e qualità nei processi e nei prodotti il gruppo ha realizzato il documento \textit{Piano Di Qualifica} il quale descrive l’insieme delle risorse atte al perseguimento della qualità.\\
In particolare definisce le metodologie che il gruppo intende attuare per
garantire la qualità del prodotto per tutta la durata del progetto.\\
La scelta di queste metodologie è a carico dei progettisti.

\subsection{Sviluppo} 
\subsubsection{Scopo} 
Il processo\glo di sviluppo ha come obiettivo di descrivere le attività di analisi, di progettazione e di codifica per il prodotto software da sviluppare. \\
L'obiettivo più importante è la realizzazione del prodotto finale, attraverso l'individuazione dei requisiti del prodotto, 
la descrizione e l'individuazione degli eventuali vincoli da rispettare.


\subsubsection{Analisi dei requisiti} 
\paragraph{Scopo} 
L'attività di analisi dei requisiti si pone come obiettivo la creazione di un documento contenente tutti i requisiti richiesti dal proponente.
Lo scopo è di identificare i requisiti obbligatori e facilitare l'attività di pianificazione della mole di lavoro.\\
Gli Analisti hanno il compito di creare il documento di Analisi dei Requisiti per:
\begin{itemize}
    \item Definire lo scopo del lavoro;
    \item Fornire ai Progettisti indicazioni precise e affidabili;
    \item Stabilire le funzioni e le richieste concordate con il cliente;
    \item Offrire una base per miglioramenti successivi per garantire un costante avanzamento del prodotto e del processo di sviluppo;
    \item Agevolare la revisione del codice;
    \item Fornire ai Verificatori\glo indicazioni sui principali e alternativi casi d'uso per l'attività di test;
    \item Calcolare i costi del progetto.
\end{itemize}   

\paragraph{Casi d'uso}\label{Casi} 
I casi d'uso descrivono l'interazione tra il sistema e uno o più attori. Ogni caso d'uso è univoco, e si distingue con un codice che segue la seguente sintassi: "UC[Codice] - [Titolo]",
dove "UC" sta per "Use Case", "[Codice]" indica un numero intero univoco per ogni caso d'uso, e "[Titolo]" è il titolo del caso d'uso. 
Ogni caso d'uso può includere informazioni aggiuntive come:
\begin{itemize}
    \item Scenario principale
    \item Scenari alternativi
    \item Pre-condizioni
    \item Post-condizioni
    \item Trigger
    \item Attori principali
    \item Attori secondari
    \item Diagramma UML
\end{itemize}

\paragraph{Requisiti} 
I requisiti sono ricavati attraverso l’analisi del capitolato\glo insieme a successive discussioni con il proponente e attraverso l’analisi dei casi d’uso.
Possono essere individuati sia tramite discussioni approfondite all’interno del gruppo che attraverso discussioni con il proponente.
I requisiti possono essere categorizzati in diverse tipologie : 
\begin{itemize}
    \item \textbf{Funzionali}: Requisiti dati dalle funzionalità che il prodotto deve presentare;
    \item \textbf{Qualità}: Requisiti dati per descrivere la qualità che il prodotto deve avere;
    \item \textbf{Vincolo}: Requisiti dati da vincoli tecnologici e strettamente legati all’implementazione del prodotto.
\end{itemize}
I requisiti possono essere categorizzati in base a diversi gradi d'importanza:
\begin{itemize}
    \item \textbf{Desiderabile}: Rappresenta un requisito che apporta valore e miglioramenti al sistema, ma non è strettamente necessario per il suo funzionamento base;
    \item \textbf{Opzionale}: Se relativamente vantaggioso, sarà preso in considerazione in un momento successivo;
    \item \textbf{Obbligatorio}: Denota un requisito essenziale per il cliente che non può essere trascurato.
\end{itemize}

\paragraph{Codifica dei requisiti}

R[Tipologia][Numero]-[Importanza] \\
Di seguito sono elencate le sigle per la codifica dei requisiti: 
\begin{itemize}
    \item \textbf{F}: Funzionale; \textbf{N}: Non Funzionale; \textbf{Q}: Qualità; (\textbf{Tipologia})
    \item \textbf{UC}[codice]; (vedi paragrafo \ref{Casi}) (\textbf{Casi d'uso})
    \item \textbf{O}: Obbligatorio; \textbf{D}: Desiderabile; \textbf{P}: Opzionale. (\textbf{Importanza})
\end{itemize}

\paragraph{Codifica dei rischi}\label{codRischi}

R[Tipologia][Numero] - Rischio [Tipologia] [Numero] \\
Di seguito sono elencate le sigle per la Tipologia: 
\begin{itemize}
    \item \textbf{T}: Tecnologico;
    \item \textbf{O}: Organizzativo;
    \item \textbf{R}: Requisito;
    \item \textbf{I}: Interpersonale
\end{itemize}
Il Numero è un contatore inizializzato a 1 e che incrementa ad ogni rischio individuato.

\subsubsection{Progettazione}\label{Prog}
I progettisti sono responsabili dell'attività di progettazione, la quale consiste nell'ideazione dell'architettura del sistema sulla base dei requisiti 
identificati nell'Analisi dei Requisiti. \\
Per raggiungere questo obiettivo, il processo di progettazione prevede la realizzazione di un Proof of Concept\glo 
per la \textbf{R}equirements and \textbf{T}echnology \textbf{B}aseline, seguito da una fase di approfondimento dei dettagli per la Product Baseline.


\paragraph{Introduzione a Docker e container}
Docker è un programma che permette di creare sistemi virtualizzati e che useremo per creare ambienti di sviluppo isolati, minimali e facilmente distribuibili. Questi singoli ambienti vengono chiamati container e hanno l’obiettivo di  semplificare il deploy delle applicazioni. A differenza dei classici modi di virtualizzare i sistemi operativi , docker si basa su “OS-Level virtualization”, una funzionalità fornita dal kernel linux che si occupa di gestire l’isolamento e le limitazioni delle risorse.
Risulta pertanto l’ambiente perfetto per lo sviluppo dei servizi di nostro interesse legati all’uso di api e comunicazione tra di essi. 

\paragraph{Lavoro in locale con le componenti}
Nel nostro progetto abbiamo deciso di sviluppare le applicazioni web in react, in particolare issuerapp  e verifierapp saranno implementate in react mentre wallet sarà implementato in react native in maniera tale da avere un'applicazione per i dispositivi mobile oltre che una webpp. \\
Per queste componenti abbiamo deciso di lavorare in locale:
\begin{itemize}
\item Issuerapp;
\item Verifierapp;
\item Wallet.
\end{itemize}
Questa scelta è stata attuata al fine di garantire maggiori performance durante lo sviluppo, semplificare la modalità di modifica ed avere una visualizzazione in tempo reale delle modifiche fatte sul browser. 
Quando queste 3 applicazioni vengono ospitate su Docker, ogni volta che viene modificato il codice, è necessario attuare un deploy e ricaricare la pagina web per visualizzare le modifiche fatte. Questo comporta un rallentamento del processo di sviluppo nel tempo andando così ad influire sulla tempistica complessiva dello sviluppo del prodotto. 
Queste componenti sono avverabili singolarmente o tutte assieme attraverso i seguenti comandi:
\begin{itemize}
\item Avvio singolo di Wallet:
\begin{itemize}
\item Aprire il terminale e posizionarsi sulla cartella tramite il seguente comando: 
\begin{verbatim}
    cd /Personal-Identity-Wallet
\end{verbatim}
\item Avviare il deploy richiamando lo script:
\begin{verbatim}
    sh deployWallet.sh
\end{verbatim} 
\item Lo script si posizionerà all'interno della cartella ./wallet, scaricherà le dipendenze necessarie alla webapp attraverso il comando:
\begin{verbatim}
    npm install 
\end{verbatim}
    e avvierà l’ambiente di sviluppo React Native attraverso il comando:
\begin{verbatim}
    npm start
\end{verbatim}
\item L’app sarà disponibile all'endpoint http://localhost:19000
\end{itemize}

\item Avviso singolo di issuer:
\begin{itemize}
\item Aprire il terminale e posizionarsi sulla cartella tramite il seguente comando: 
\begin{verbatim}
    cd /Personal-Identity-Wallet
\end{verbatim}
\item Avviare il deploy richiamando lo script:
\begin{verbatim}
    sh deployIssuerapp.sh
\end{verbatim}
\item Lo script si posizionerà all'interno della cartella ./issuerapp, scaricherà le dipendenze necessarie alla webapp attraverso: 
\begin{verbatim}
    npm install 
\end{verbatim}
    e successivamente sarà quindi avviato l’ambiente di sviluppo del framework React attraverso il comando:
\begin{verbatim}
    npm start 
\end{verbatim}
Sarà visibile all'endpoint http/localhsot:19001 il frontend dell'applicazione.
\item Lo script avvierà anche l’istanza di node.js che esegue funzionalità serverside e sarà disponibile all'endpoint http://localhost:19101
\end{itemize}

\item Avvio singolo di verifier:
\begin{itemize}
\item Aprire il terminale e posizionarsi sulla cartella tramite il seguente comando: 
\begin{verbatim}
    cd /Personal-Identity-Wallet
\end{verbatim}
\item Avviare il deploy richiamando lo script: 
\begin{verbatim}
    sh deployVerifier.sh
\end{verbatim}
\item Lo script si posizionerà all'interno della cartella ./verifierapp, scaricherà le dipendenze necessarie attraverso il comando: 
\begin{verbatim}
    npm install
\end{verbatim}
 e avvierà l’ambiente di sviluppo del framework React con:
\begin{verbatim}
    npm start
\end{verbatim}

\item Sarà quindi visualizzabile all'endpoint http/localhsot:19002 il frontend dell'applicazione.
\end{itemize}

\item Avvio delle 3 componenti simultaneamente:
\begin{itemize}
\item Aprire il terminale e posizionarsi sulla cartella tramite il seguente comando: 
\begin{verbatim}
    cd /Personal-Identity-Wallet
\end{verbatim}
\item Avviare il deploy richiamando lo script: 
\begin{verbatim}
    sh deployAll.sh
\end{verbatim}
\end{itemize}
\end{itemize}

\paragraph{Considerazioni sulle componenti e sui processi}
La parte dell’\textit{Issuer} è composto, oltre che da un interfaccia web frontend di React, anche da una parte serverside in node.js.\\
\textit{Wallet} essendo, come sopra citato, realizzata con React Native, ad ogni modifica del codice, la pagina per la visualizzazione web deve essere aggiornata, richiedendo quindi all'ambiente di sviluppo di rifare la build della webapp.\\
\textit{Issuerapp} e \textit{Verifierapp} sono realizzate con React e, tramite un sistema di watcher,  le modifiche al codice vengono riconosciute automaticamente. La webapp viene ricompilata e la pagina del browser viene aggiornata in automatico con le modifiche apportate.\\
Avviando le webapp utilizzando gli script, in base al caso utilizzato, implica che si avrà un singolo processo in foreground e gli altri saranno in background.\\
Può essere necessario avviarli manualmente e non utilizzando gli script. Per esempio l’\textit{Issuer} ha la parte serverside in node.js ed essa, ad ogni modifica, deve essere fatto il deploy nuovamente.\\ 
Per fare questo si ricorda l’utilizzo per il frontend del comando:
\begin{verbatim}
    npm start 
\end{verbatim}
e per la parte serverside del comando: 
\begin{verbatim}
    node index.js 
\end{verbatim}  
Per riavviare il nodo serverside si preme Ctrl+C sul terminale e si riavvia con il comando: 
\begin{verbatim}
    node index.js
\end{verbatim}
Si ricorda che si avranno dei processi in background che dovranno essere spenti se si vogliono riaprire delle nuove istanze di essi (questo perchè occupano già le porte delle webapp).\\
Per spegnere tutte le istanze delle webapp si utilizza il comando:
\begin{verbatim}
    killall node
\end{verbatim}

\paragraph{Lavoro su Docker}
Per quanto riguarda i container docker, nel nostro progetto utilizzeremo le seguenti componenti:
\begin{itemize}
    \item 3 istanziazioni di Walt id:
    \begin{itemize}
        \item verifierapi con endpoint http://localhost:19012
        \item walletapi con endpoint http://localhost:19010
        \item issuerapi con endpoint http://localhost:19011
    \end{itemize}
    \item Un database per l’issuer chiamato dbissuerapp. Viene usato PostgreSQL per il  DBMS;
    \item Un tool di amministrazione del database che serve allo sviluppo (chiamato adminer).
\end{itemize}
Il database è configurato sulla porta standard 5432 e all’interno della rete di docker ha come ip fisso 10.5.0.5 . E' disponibile anche da localhost.\\
La porta in entrambi i casi è 5432.
Le credenziali di accesso al DB dbissuerapp sono:
\begin{itemize}
    \item username: admin
    \item password: admin
    \item db: issuerapp
    \item server: localhost:5432
\end{itemize}

\paragraph{Comandi componenti Docker}
Per fare il deploy di queste componenti docker:
\begin{itemize}
    \item Aprire il terminale e posizionarsi sulla cartella tramite il seguente comando: 
    \begin{verbatim}
        cd /Personal-Identity-Wallet
    \end{verbatim}
    \item Avviare le componenti tramite il comando:
    \begin{verbatim}
        docker-compose up -d
    \end{verbatim}
\end{itemize}

\paragraph{Considerazioni componenti Docker}
Il comando:
\begin{verbatim}
    docker-compose up -d
\end{verbatim} 
prende le configurazioni all’interno del file docker-compose.yaml e crea le istanze dei container con le nuove configurazioni.\\
Se il container non subisce modifiche esso non viene modificato dal comando. Se il container non esiste viene creato. Se presenta modifiche viene ricreato.\\
Il funzionamento dei container si basa su un'immagine e un'istanza di essa detto container. Può essere necessario in alcuni momenti legati al mantenimento di eliminare anche le immagini per ricreare un ambiente di sviluppo con la giusta configurazione.\\
La memoria di archiviazione nei container può avvenire in due maniere:
\begin{itemize}
    \item se il container presenta volumi, i file saranno memorizzati in locale;
    \item se non presenta volumi saranno memorizzati nella istanza del container.
\end{itemize} 
Una volta ricreato il container si perdono anche i file memorizzati al suo interno, a meno che essi non siano memorizzati in un volume oppure all’interno di una copia dell’immagine.
Una volta terminato lo sviluppo in vista del POC, il progetto (composto dai file htm,css e js) verrà trasportato in un container docker adibito a server http (es. Apache) mentre la parte serverside dell'Issuerapp sarà trasportata su un container di Node.js.






\subsubsection{Codifica}
\begin{itemize}
    \item \textbf{Indentazione}: I blocchi di codice innestati e le varie linee di codice dovranno rispettare una determinata tabulazione al fine di rendere il codice più leggibile;
    \item \textbf{Blocco di codice}: La delimitazione dei vari blocchi di codice tra parentesi graffe dovranno avere una struttura determinata dove le parentesi 
    graffe saranno poste nella seguente maniera:
    \begin{itemize}
        \item \textbf{Parentesi di apertura}: Posta a distanza di 1 spazio rispetto alla dichiarazione del costrutto ma sempre nella stessa linea;
        \item \textbf{Parentesi di chiusura}: Posta nella linea successiva dell’ultima riga di codice del blocco a cui si fa riferimento e indentata secondo il blocco. 
    \end{itemize}
    \item \textbf{Commenti}: I commenti saranno inseriti in lingua italiana, saranno indispensabili nelle porzioni di codice dove non è immediata la comprensibile. Ogni qualvolta il codice 
    venga aggiornato va ricontrollata la validità dei commenti ad esso associato;
    \item \textbf{Nomi univoci}: È necessario utilizzare nomi univoci e significativi per ogni costruttore al fine di garantire la chiarezza e la facilità di identificazione delle diverse entità.
\end{itemize}









