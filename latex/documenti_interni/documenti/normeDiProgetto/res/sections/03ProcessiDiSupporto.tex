\section{Processi di Supporto}\label{sup}

\subsection{Documentazione}
\subsubsection{Scopo}
Sarà necessario documentare ogni fase e attività coinvolta nello sviluppo del progetto.\\
Questa sezione illustrerà le regole e gli standard da seguire durante la produzione 
dei documenti per l'intero ciclo di vita del software. \\
Qui esamineremo approfonditamente i processi adottati per la stesura, la verifica\glo, il mantenimento e l’approvazione di tutta la documentazione prodotta dal nostro team. \\
Lo scopo è di fornire una descrizione accurata di tutte le regole, le convenzioni e i limiti che ci impegniamo a rispettare per ottenere una documentazione efficace, coerente e formale.\\

\subsubsection{Documenti prodotti}
\begin{itemize}
    \item Norme di progetto;
    \item Glossario;
    \item Piano di progetto;
    \item Piano di Qualifica;
    \item Analisi dei requisiti;
    \item Verbali.
\end{itemize}

\subsubsection{Ciclo di vita di un documento}
\begin{itemize}
    \item \textbf{Creazione Struttura}: A partire da un template comune viene creata la struttura del documento inserendo il registro delle modifiche e l’indice dei contenuti;
    \item \textbf{Stesura}: Durante questa fase i redattori producono le sezioni assegnate a loro dal Responsabile\glo di Progetto aggiornando progressivamente il documento;
    \item \textbf{Approvazione}: Quando un redattore ha terminato le sue modifiche il Responsabile del progetto assegna ad un Verificatore\glo il compito di esaminare ed approvare le modifiche.
Il documento viene pubblicato se tutte le modifiche hanno ricevuto l’approvazione da parte dei Verificatori e successivamente dal Responsabile del Progetto.
\end{itemize}

\subsubsection{Verbali}\label{Verbali}
I verbali sono suddivisi in Interni ed Esterni.\\
Dovranno contenere le seguenti informazioni:
\begin{itemize}
    \item Motivo della riunione;
    \item Luogo della riunione (se in presenza o da remoto);
    \item Data, ora e durata della riunione;
    \item Partecipanti della riunione;
    \item Resoconto della riunione con le decisioni intraprese.
\end{itemize}

\subsubsection{Strumenti utilizzati}
Gli strumenti utilizzati per la stesura dei documenti sono:
\begin{itemize}
    \item \textbf{LATEX}: Tutti i documenti prodotti dal gruppo verranno redatti usando il linguaggio di markup LateX;
    \item \textbf{Visual Studio Code con l’estensione Latex Workshop}: Il gruppo adotterà questo strumento per scrivere i documenti e per creare i file;
    \item \textbf{Draw.io}: Sito online per la creazione di grafici UML;
    \item \textbf{GanttProject}: Programma usato per la realizzazione di diagrammi di Gantt.
\end{itemize}
Per semplificare le operazioni di verifica\glo e di lettura della documentazione dovrà essere reso disponibile il documento completo in formato PDF.


\newpage
\subsection{Directory Documenti}
\subsubsection{Struttura delle directory e dei file}\label{Dir}
I documenti saranno archiviati con la seguente struttura della directory: \\

\begin{forest}
  for tree={
    font=\ttfamily,
    grow'=0,
    child anchor=west,
    parent anchor=south,
    anchor=west,
    calign=first,
    edge path={
      \noexpand\path [draw, \forestoption{edge}]
      (!u.south west) +(7.5pt,0) |- node[fill,inner sep=1.25pt] {} (.child anchor)\forestoption{edge label};
    },
    before typesetting nodes={
      if n=1
        {insert before={[,phantom]}}
        {}
    },
    fit=band,
    before computing xy={l=15pt},
  }
[docs
  [documenti\_esterni
      [verbali
        [VE\_YYYY\_MM\_DD.pdf]
      ]
      [documenti
        [PdP.pdf]
      ]
  ]
  [documenti\_interni
    [verbali
      [VI\_YYYY\_MM\_DD.pdf]
    ]
    [documenti
      [NdP.pdf]
    ]
  ]
  [latex
    [documenti\_esterni
        [verbali
          [VE\_YYYY\_MM\_DD
            [VE\_YYYY\_MM\_DD.pdf]
            [VE\_YYYY\_MM\_DD.tex]
          ]
        ]
        [documenti
          [PdP
            [PdP.pdf]
            [PdP.tex]
          ]
        ]
    ]
    [documenti\_interni
      [verbali
        [VI\_YYYY\_MM\_DD
          [VI\_YYYY\_MM\_DD.pdf]
          [VI\_YYYY\_MM\_DD.tex]
        ]
      ]
      [documenti
        [NdP
          [NdP.pdf]
          [NdP.tex]
        ]
      ]
    ]
  ]
]
\end{forest}\\
I documenti saranno redatti in formato Latex e la loro stesura averrà all'interno della cartella \code{docs/latex}.\\
Saranno suddivisi nelle rispettive cartelle \code{documenti\_esterni} e \code{documenti\_interni} e per ciascuna categoria ripartiti
nelle cartelle \code{verbali} e \code{documenti}.\\
Un documento è archiviato nella sua rispettiva cartella \code{nomeDocumento} dove al suo interno si può trovare il file Latex
\code{nomeDocumento.tex}, \code{nomeDocumento.pdf} con il file compilato, la cartella \code{res} contenente file di risorse del documento come allegati
e le sezioni utilizzate, ed infine la cartella \code{config} contenente la configurazione specifica del documento che instanzia il template.\\
La cartella \code{docs/template} contiene i file creati appositamente per avere un template utilizzabile per tutti i documenti.\\
Per facilitare la lettura e separare i file Latex e le loro rispettive dipendenze si è deciso di creare una struttura directory
parallela contenente solo i file PDF.\\
Le rispettive cartelle sono \code{docs/documenti\_esterni} e \code{docs/documenti\_interni} anch'esse individualmente ripartite in documenti e verbali.\\
Un documento sarà quindi archiviato per esempio come\\
\code{docs/documenti\_interni/documenti/nomeDocumento.pdf}.\\
Si è deciso di mantenere i documenti in formato PDF sia nella cartella latex che nelle cartelle\\
\code{docs/documenti\_esterni} e \code{docs/documenti\_interni}.\\
Per ovviare al ovvio problema della inconsistenza dei file PDF ed evitare che qualcuno legga un documento non aggiornato si è attuato una azione che 
mantiene la consistenza e la coerenza dei file PDF spiegata di seguito.\\

\subsubsection{Github Actions copyONPush.yaml}
Si è utilizzata lo strumento GitHub Actions, in particolare esiste uno script contenuto nella cartella\\
\code{.github/workflows/copyOnPush.yaml} che ad ogni push copia e sovrascrive
i file PDF dalla cartella \code{docs/latex} nelle rispettive cartelle parallele.\\
Utilizzando questo strumento si ha sempre i file aggiornati e non è necessario copiarli manualmente, i file vengono spostati automaticamente dalla cartella latex nelle cartelle dedicate solo
ai file PDF.\\
Per questo motivo è essenziale spiegare che è dovere solo del script copiare i file PDF, i membri del gruppo non devono spostare file PDF nelle cartelle dedicate, essi vengono spostati solo e solamente attraverso 
lo script.\\
Tutti i file Latex all'interno della cartella \code{docs/latex} devono stare alla stessa altezza della gerarchia, questo per non creare problemi legati alla incompatibilità 
con i percorsi URL nel file di template.
Nelle cartelle parallele dedicate solo ai PDF tuttavia si è deciso che è possibile creare delle cartelle contenitori per ripartire diversi documenti.
Essi sono spostati automaticamente dallo script.
I file legati alla candidatura iniziale sono stati ragruppati dallo script dentro
la sottocartella\\
\code{docs/documenti\_esterni/documenti/candidatura}.\\
I file legati a questioni strettaente tecniche sono stati raggruppati dallo script dentro la sottocartella\\
\code{docs/documenti\_interni/documenti/documenti\_tecnici}.

\subsection{Forma Documenti}
\subsubsection{Struttura Documenti}\label{strutturaDocumenti}
La prima pagina è composta dal logo del gruppo, email di riferimento, titolo del documento.
Sotto si può trovare una struttura indicante la versione attuale, il responsabile che ha approvato il documento, i redattori e i verificatori che hanno contribuito al documento, l'uso che ne si fa e i destinatari del documento.
In basso si trova una descrizione breve del documento.
Tutte queste informazioni sono istanziabili per ogni documento attraverso il file di configurazione \code{titlePageInput.tex}.
In qualsiasi pagina del documento si può trovare il logo del gruppo in alto a sinistra, il nome del documento in alto a destra ed in fine a piè di pagina si può trovare la pagina corrente con il numero totali di pagine.

\subsubsection{Registro delle modifiche}\label{Registro}
All’inizio del documento deve essere presente una tabella riassuntiva della cronologia delle versioni del documento dove sono specificate le modifiche apportate.
Nella tabella ogni riga corrisponde ad una modifica apportata, mentre le colonne sono le seguenti:
\begin{itemize}
    \item \textbf{Versione}: Versione del documento dopo la modifica;
    \item \textbf{Data}: Data della modifica;
    \item \textbf{Autore}: Nome dell’autore della modifica o della azione apportata;
    \item \textbf{Ruolo}: Redattore nel caso di modifiche del documento, Verificatore\glo per chi verifica\glo le modifiche apportate e Responsabile\glo per l’approvazione finale da parte del Responsabile del Progetto;
    \item \textbf{Descrizione}: Breve descrizione della modifica apportata.
\end{itemize}

\subsubsection{Forma Tipografica}
\begin{itemize}
  \item \textbf{Nomi dei documenti}: I nomi dei documenti iniziano sempre con la lettera minuscola. Se presenti più parole, queste saranno attaccate ma distinguibili dalla lettera maiuscola (convenzione "CamelCase");
  \item \textbf{Nomi dei verbali}: I verbali avranno la seguente struttura \code{V[Tipologia]\_[YYYY]\_[MM]\_[DD]} con la tipologia che può essere Interno \code{I} oppure Esterno \code{E}.
\end{itemize}

\subsection{Versionamento}
I documenti dovranno supportare il versionamento, in modo da permettere l’accesso ad ogni singola versione prodotta durante il loro ciclo di vita. \\
Il formato identificativo del versionamento dovrà seguire questa forma: \\
v[X].[Y].[Z] \\
Il valore "X" rappresenta la versione pubblicamente rilasciata e approvata dal responsabile del progetto. Questa numerazione inizia da zero e viene incrementata ad ogni nuova versione approvata. \\
Il valore "Y" indica una revisione da parte del Verificatore effettuata per assicurarsi che, dopo l'implementazione di una modifica, il prodotto sia ancora integro e coerente. La numerazione inizia anch'essa da zero e viene azzerrata ogni volta che viene incrementato il valore "X". \\
Il valore "Z" viene incrementato ad ogni modifica del prodotto. Anche questa numerazione inizia da zero e viene azzerata ogni volta che viene incrementato il valore "X" o "Y". \\