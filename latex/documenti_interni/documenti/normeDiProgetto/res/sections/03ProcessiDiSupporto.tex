\section{Processi di Supporto}

\subsection{Documentazione}
\subsubsection{Scopo}
Sarà necessario documentare ogni fase e attività coinvolta nello sviluppo del progetto. Questa sezione illustrerà le regole e gli standard da seguire durante la produzione 
dei documenti per l'intero ciclo di vita del software. \\
Qui esamineremo approfonditamente i processi adottati per la stesura, la verifica\glo, il mantenimento e l’approvazione di tutta la documentazione prodotta dal nostro team. \\
Lo scopo è di fornire una descrizione accurata di tutte le regole, le convenzioni e i limiti che ci impegniamo a rispettare per ottenere una documentazione efficace, coerente e formale.\\

\subsubsection{Documenti prodotti}
\begin{itemize}
    \item \textbf{Norme di progetto}.
    \item \textbf{Glossario}.
    \item \textbf{Piano di progetto}.
    \item \textbf{Piano di Qualifica}.
    \item \textbf{Analisi dei requisiti}.
    \item \textbf{Verbali}.
\end{itemize}

\subsubsection{Ciclo di vita di un documento}
\begin{itemize}
    \item \textbf{Creazione Struttura}: A partire da un template comune viene creata la struttura del documento inserendo il registro delle modifiche e l’indice dei contenuti;
    \item \textbf{Stesura}: Durante questa fase i redattori producono le sezioni assegnate a loro dal Responsabile\glo di Progetto aggiornando progressivamente il documento;
    \item \textbf{Approvazione}: Quando un redattore ha terminato le sue modifiche il Responsabile del progetto assegna ad un Verificatore\glo il compito di esaminare ed approvare le modifiche.
Il documento viene pubblicato se tutte le modifiche hanno ricevuto l’approvazione da parte dei Verificatori e successivamente dal Responsabile del Progetto.
\end{itemize}
\newpage
\subsubsection{Struttura delle directory e dei file}
I documenti saranno archiviati con la seguente struttura della directory: \\

\begin{forest}
  for tree={
    font=\ttfamily,
    grow'=0,
    child anchor=west,
    parent anchor=south,
    anchor=west,
    calign=first,
    edge path={
      \noexpand\path [draw, \forestoption{edge}]
      (!u.south west) +(7.5pt,0) |- node[fill,inner sep=1.25pt] {} (.child anchor)\forestoption{edge label};
    },
    before typesetting nodes={
      if n=1
        {insert before={[,phantom]}}
        {}
    },
    fit=band,
    before computing xy={l=15pt},
  }
[docs
  [documenti esterni
     [verbali
        [VE\_XXXX\_YY\_ZZ.pdf]
     ]
    [PdP.pdf]
    [PdQ.pdf]
    [glossario.pdf]
  ]
  [documenti interni
    [verbali
        [VI\_XXXX\_YY\_ZZ.pdf]
    ]
    [NdP.pdf]
  ]
]
\end{forest}\\

\subsubsection{Registro delle modifiche}\label{Registro}
All’inizio del documento deve essere presente una tabella riassuntiva della cronologia delle versioni del documento dove sono specificate le modifiche apportate.
Nella tabella ogni riga corrisponde ad una modifica apportata, mentre le colonne sono le seguenti:
\begin{itemize}
    \item \textbf{Versione}: Versione del documento dopo la modifica;
    \item \textbf{Data}: Data della modifica;
    \item \textbf{Autore}: Nome dell’autore della modifica o della azione apportata;
    \item \textbf{Ruolo}: Redattore nel caso di modifiche del documento, Verificatore\glo per chi verifica\glo le modifiche apportate e Responsabile\glo per l’approvazione finale da parte del Responsabile del Progetto;
    \item \textbf{Descrizione}: Breve descrizione della modifica apportata.
\end{itemize}

\subsubsection{Verbali}\label{Verbali}
I verbali sono suddivisi in Interni ed Esterni.\\
Dovranno contenere le seguenti informazioni:
\begin{itemize}
    \item \textbf{Motivo della riunione};
    \item \textbf{Luogo della riunione (se in presenza o da remoto)};
    \item \textbf{Data, ora e durata della riunione};
    \item \textbf{Partecipanti della riunione};
    \item \textbf{Resoconto della riunione con le decisioni intraprese}.
\end{itemize}

\subsubsection{Strumenti utilizzati}
Gli strumenti utilizzati per la stesura dei documenti sono:
\begin{itemize}
    \item \textbf{LATEX}: Tutti i documenti prodotti dal gruppo verranno redatti usando il linguaggio di markup LateX;
    \item \textbf{Visual Studio Code con l’estensione Latex Workshop}: Il gruppo adotterà questo strumento per scrivere i documenti e per creare i file;
    \item \textbf{Draw.io}: Sito online per la creazione di grafici UML;
    \item \textbf{GanttProject}: Programma usato per la realizzazione di diagrammi di Gantt.
\end{itemize}
Per semplificare le operazioni di verifica\glo e di lettura della documentazione dovrà essere reso disponibile il documento completo in formato PDF.

\subsection{Versionamento}
I documenti dovranno supportare il versionamento, in modo da permettere l’accesso ad ogni singola versione prodotta durante il loro ciclo di vita. \\
Il formato identificativo del versionamento dovrà seguire questa forma: \\
v[X].[Y].[Z] \\
Il valore "X" rappresenta la versione pubblicamente rilasciata e approvata dal responsabile del progetto. Questa numerazione inizia da zero e viene incrementata ad ogni nuova versione approvata. \\
Il valore "Y" indica una revisione da parte del Verificatore effettuata per assicurarsi che, dopo l'implementazione di una modifica, il prodotto sia ancora integro e coerente. La numerazione inizia anch'essa da zero e viene azzerrata ogni volta che viene incrementato il valore "X". \\
Il valore "Z" viene incrementato ad ogni modifica del prodotto. Anche questa numerazione inizia da zero e viene azzerata ogni volta che viene incrementato il valore "X" o "Y". \\