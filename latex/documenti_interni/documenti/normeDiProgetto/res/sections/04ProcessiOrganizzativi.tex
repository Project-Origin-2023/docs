\section{Processi Organizzativi}

\subsection{Gestione di processo}
Secondo lo standard ISO-12207:1995, questa gestione contiene attività e compiti generici, tra cui la definizione dell'obiettivo del processo\glo, la pianificazione e la stima 
dei tempi, delle risorse e dei costi, l'assegnazione di compiti e responsabilità, l'esecuzione e il controllo delle attività, la revisione e valutazione delle attività svolte 
e la determinazione della fine del processo. \\
Gli obiettivi della gestione di processo includono la semplificazione e la gestione della comunicazione tra i membri del gruppo e con l'esterno, la coordinazione 
dell'assegnazione dei ruoli e dei compiti, il monitoraggio del lavoro del gruppo e la pianificazione delle attività da svolgere, nonché la definizione delle linee guida 
generali per la formazione dei membri. \\
Lo scopo della gestione di processo è l'istanziazione dei processi di progetto, la stima dei costi e delle risorse necessarie per la loro esecuzione, la pianificazione 
delle attività e dei task ad essi associati, l'assegnazione del personale e il controllo e la verifica\glo delle attività dei processi di progetto durante la loro esecuzione. 
Tali attività mirano a garantire la coerenza e la coesione del prodotto, verificando che sia sempre integro e conforme agli obiettivi iniziali.

\subsection{Comunicazione}
Le comunicazioni all'interno del progetto avvengono su due livelli distinti: tra i membri del gruppo e tra i membri del gruppo e soggetti esterni. I soggetti esterni 
identificati sono l'azienda Infocert in qualità di proponente e i professori Tullio Vardanega e Riccardo Cardin come committenti. Per quanto riguarda le comunicazioni interne, 
il principale mezzo utilizzato è Telegram\glo. Le videochiamate interne, come le riunioni di progetto, si svolgono principalmente su Discord\glo, scelto per la sua semplicità e 
versatilità multi-piattaforma. In alternativa, può essere utilizzato Zoom\glo. Tramite un accordo con l'azienda si è deciso di comunicare nella piattaforma Teams(Microsoft)\glo. Per quanto riguarda le comunicazioni esterne, è stato creato un indirizzo e-mail 
apposito, projectorigin2023@gmail.com, a cui tutti i membri del gruppo hanno accesso. Ogni membro è tenuto a controllare regolarmente la casella di posta elettronica e a 
notificare il gruppo in caso di nuovi messaggi. La stesura e l'invio dei messaggi è compito del Responsabile\glo di Progetto, previa breve verifica\glo e approvazione del gruppo. 
Tutte le e-mail verranno firmate con "Project Origin". 

\subsection{Pianificazione}

\subsubsection{Scopo}
L'obiettivo della pianificazione è quello di creare un piano di progetto che sia un documento esterno, il cui contenuto descrive i seguenti aspetti:
\begin{itemize}
    \item le risorse che sono disponibili;
    \item come queste risorse sono state assegnate alle varie attività dei processi di progetto;
    \item la sequenza temporale delle attività che verranno svolte.
\end{itemize}

\subsubsection{Obiettivi}
L'obiettivo principale della pianificazione del progetto è quello di redigere un Piano di progetto che permetta di:
\begin{itemize}
    \item organizzare le attività in modo efficiente, in modo da raggiungere risultati efficaci;
    \item definire gli obiettivi dei processi e semplificare il monitoraggio del progresso attraverso l'individuazione di milestone temporali;
\end{itemize}
La struttura del Piano di progetto è così composta:
\begin{itemize}
    \item introduzione e finalità;
    \item organizzazione generale del progetto;
    \item analisi dei rischi;
    \item risorse a disposizione;
    \item struttura di scomposizione del lavoro (WBS) delle attività dei processi istanziati;
    \item pianificazione del progetto (project schedule);
    \item strumenti di controllo e di reportistica.
\end{itemize}

\subsubsection{Procedura}
\begin{itemize}
    \item \textbf{Identificazione delle attività sulla base dei requisiti da soddisfare}: È opportuno organizzare le attività individuate in un elenco. Le attività dei processi saranno decomposte mediante una struttura di scomposizione del lavoro, in una struttura gerarchica. I task ottenuti possono così essere identificati in modo univoco;
    \item \textbf{Identificazione ed analisi dei rischi}: I rischi considerati sono lo sforamento di costi, sforamento dei tempi e risultati insoddisfacenti. La gestione dei rischi è costituita da tre fasi in fase di pianificazione, ed un'ulteriore fase da iterare durante tutta l'esecuzione di un processo. \\
    Le tre fasi in fase di pianificazione sono: 
    \begin{itemize}
     \item \textbf{Individuazione dei rischi:} Si considerano tutti i possibili rischi che possono emergere a causa di fattori interni ed esterni. I rischi identificati vengono riportati in una tabella esplicativa;
     \item \textbf{Analisi dei rischi individuati:} Ai rischi presenti nell'elenco è necessario associare una probabilità di occorrenza ed una stima dell'impatto sulla corretta esecuzione del processo;
     \item \textbf{Pianificazione dei rischi individuati:} Le attività devono essere pianificate in modo tale da minimizzare sia la probabilità di occorrenza dei rischi che l'effetto che essi possono avere sul progetto. Le attività sono pianificate su compiti brevi in rapporto alla quantità di lavoro da svolgere (tempo/persona), al fine di ridurre al minimo l'impatto dei rischi associati.
    \end{itemize}
     \item \textbf{Un insieme di attività coerenti è associato ad una specifica tappa temporale nel calendario.}\\ Tuttavia, queste attività devono essere sufficientemente brevi o suddivise in sotto-attività brevi per consentire un controllo e una verifica\glo più facili ed efficaci degli effetti sul prodotto associati all'incremento risultante, garantendo un maggiore controllo durante la fase di verifica\glo sull'introduzione di eventuali errori nel prodotto complessivo. \\ 
     I fattori da considerare per minimizzare i rischi includono: la completezza dei requisiti, il coinvolgimento del cliente, un'opportuna allocazione delle risorse, la fondatezza delle aspettative, la presenza di supporto esecutivo, la corretta gestione della fluttuazione dei requisiti.\\
     Durante l'esecuzione del processo si misurano degli indicatori. Questo può portare a dover rivedere la pianificazione delle attività in corso. \\
     Riordinamento delle attività identificate in base alle dipendenze ingresso-uscita per comprendere, attraverso i diagrammi di Gantt: la sequenzialità temporale delle attività rispetto alle loro dipendenze, il possibile parallelismo tra le varie attività, come la durata effettiva di un'attività si sovrapponga alla durata pianificata, come le stime fatte corrispondano ai progressi, come ogni attività può essere associata al tempo di calendario, limitato superiormente dall'ultima scadenza contrattuale e discretizzato in unità di tempo/persona, il margine di slack assegnabile a ciascuna attività per poter ammortizzare più ritardi possibili.
    \item \textbf{Stima delle risorse da assegnare a ciascuna attività}: Sono considerate due tipologie di risorse: sforzo, tempo di calendario. \\
     Come stimare le risorse: assegnando a ciascuna attività un valore, secondo la metrica tempo/persona, della quantità di lavoro necessaria per portarla a termine, identificando delle tappe temporali nel tempo di calendario e assegnando le attività alle corrispondenti tappe.\\
     La pianificazione delle attività (cioè l'assegnazione del tempo/persona sul tempo di calendario) viene effettuata seguendo due criteri: il primo criterio è la pianificazione all'indietro per capire se sia possibile pianificare all'interno dei limiti di tempo di calendario imposti dalle scadenze contrattuali, sfruttando dove possibile la parallelizzazione delle attività. Applicata nella stesura del Piano di progetto v1.0.0 D. Si pianifica all'indietro a partire dalle tappe identificate nel calendario. Il secondo criterio è la pianificazione in avanti per rispettare le dipendenze ingresso-uscita delle attività. Applicata nella stesura del Piano di progetto v1.0.0 D e dinamicamente durante la gestione dei ticket.
    \item \textbf{Assegnazione delle risorse stimate ad ogni attività}: Ogni risorsa umana viene destinata ad un ruolo di progetto. Il personale quindi viene assegnato ai task associati al ruolo assunto, attraverso un sistema di segnalazione dei ticket su un repository\glo. Il repository di segnalazione viene inizialmente popolato dai task derivati dalla WBS e viene dinamicamente arricchito e svuotato a mano a mano che emergono e vengono risolti i problemi e i task. E' importante che i task all'interno del repository siano ordinati in base alle dipendenze ingresso-uscita e in base ad una priorità loro assegnata.
\end{itemize}

\subsubsection{Ruoli}
Il processo\glo di sviluppo software coinvolge molteplici figure professionali, ognuna con specifiche responsabilità. Tra queste figure troviamo l'Analista\glo, il Progettista\glo, il Programmatore\glo, il Verificatore\glo, il Responsabile\glo e l'Amministratore di progetto. \\
L'Analista, presente soprattutto nella fase iniziale del progetto, si occupa di comprendere il problema e definire i requisiti espliciti ed impliciti. \\
Il Progettista, invece, effettua lo studio di fattibilità del prodotto e costruisce l'architettura, partendo dal lavoro svolto dall'Analista e perseguendo efficienza\glo ed efficacia\glo. \\
Il Programmatore si occupa di implementare le specifiche fornite dal Progettista, scrivendo codice orientato alla futura manutenzione e alla riusabilità. \\
Il Verificatore\glo, invece, ha il compito di controllare ciò che viene prodotto dagli altri membri del team, individuando eventuali errori e segnalando al responsabile del prodotto analizzato. \\
Il Responsabile di progetto, figura fondamentale, rappresenta il team presso il committente e guida e coordina il team verso il raggiungimento degli obiettivi di progetto. Si occupa di prendere decisioni e approvare documenti, coordinare i membri del team e valutare i rischi e i costi, mantenendo le relazioni con i soggetti esterni e rispettando le scadenze e l'allocazione delle risorse. \\
Infine, l'Amministratore di progetto gestisce e controlla l'ambiente di lavoro, definendo le norme e le procedure alla base del lavoro, regolando le infrastrutture e i servizi utili per lo svolgimento dei processi, gestendo il versionamento dei prodotti e la loro configurazione, individuando strumenti utili a migliorare e/o automatizzare i processi e gestendo la documentazione di progetto. 

\subsection{Formazione dei membri del team}
Lo scopo della formazione è di garantire che ogni componente del team abbia le competenze necessarie per svolgere con successo le proprie attività e che vi sia un costante aggiornamento delle conoscenze del personale nel tempo, in modo da mantenere un team altamente qualificato.

\subsubsection{Formazione interna}
È richiesto che ogni componente del team acquisisca le competenze necessarie per svolgere i compiti assegnati in modo autonomo, cercando di colmare eventuali lacune attraverso lo studio individuale. I membri più esperti sono incoraggiati a condividere le proprie conoscenze e risorse con il resto del gruppo. Nel caso in cui un membro riscontri difficoltà nell'esecuzione di un compito, è possibile rivolgersi al Responsabile\glo di Progetto per richiedere supporto nell'organizzazione di attività di apprendimento. Sebbene siano disponibili documenti di riferimento, è consigliabile integrare la formazione con materiali di approfondimento individuale, in quanto la documentazione fornita potrebbe non essere esaustiva.

\subsubsection{Strumenti a supporto}
Il processo di gestione organizzativa del team sarà supportato da una serie di strumenti che includono:
\begin{itemize}
    \item \textbf{Telegram\glo}: Una piattaforma di messaggistica utilizzata dal team per comunicazioni meno urgenti e decisioni di minor importanza;
    \item \textbf{Git\glo}: Uno strumento di controllo versione utilizzato dal team per tenere traccia dei cambiamenti nei documenti;
    \item \textbf{GitHub\glo}: Una piattaforma online utilizzata dal team per il controllo versione e per il salvataggio di tutti i file creati dai membri del team;
    \item \textbf{GitHub Issues\glo}: Un sistema integrato in GitHub che consente la gestione dei ticket e la segnalazione dei problemi;
    \item \textbf{GitHub Actions\glo} è uno strumento fornito da GitHub che permette l’automazione di compiti di varia natura.
    \item \textbf{GitHub Desktop\glo}: GitHub Desktop è un'applicazione gratuita e open source per Windows e Mac per gestire senza problemi i tuoi progetti, creare commit significativi e tenere traccia della cronologia del progetto in un'applicazione anziché nella riga di comando.
    \item \textbf{Google Drive\glo}: Un servizio di condivisione di documenti che consente la collaborazione in tempo reale e la modifica condivisa;
    \item \textbf{Gmail\glo}: Un servizio di posta elettronica scelto dal team per la comunicazione interna ed esterna;
    \item \textbf{Zoom\glo}: Uno strumento che si può utilizzare per le riunioni virtuali tra i membri del team;
    \item \textbf{Discord\glo}: Uno strumento standard per le riunioni tra i membri del team e per la comunicazione interna;
    \item \textbf{Teams (Microsoft)\glo}: Uno strumento che si può utilizzare per le riunioni virtuali.
\end{itemize}