\section{Problema generale}
Gli Stati membri hanno adottato diverse soluzioni tecnologiche, il che ha portato a problemi di interoperabilità tra confini nazionali e settori diversi. Inoltre, manca un quadro giuridico comune per stabilire l'affidabilità delle entità che emettono le identità elettroniche (eID). Questa mancanza di un quadro giuridico comune crea incertezza legale nell'uso transfrontaliero delle eID e nella definizione delle responsabilità in caso di identità errate.

\section{eSignatures}
Prima dell'adozione del regolamento eIDAS, le normative europee in questo settore si applicavano solo alle firme elettroniche e ai dispositivi utilizzati per crearle, come stabilito dalla direttiva sulle firme elettroniche.
\begin{itemize}
    \item Problema dei costi elevati: i prestatori di servizi che desideravano offrire i propri servizi in un altro Stato membro hanno dovuto affrontare costi elevati dovuti a requisiti tecnici variabili e alla necessità di conformarsi al Paese di destinazione.
    \item Problema degli standard obsoleti: si trattava di un altro problema non affrontato nella direttiva sulle firme elettroniche. La direttiva non ha stabilito norme comuni per le firme elettroniche, dando luogo a un mosaico di legislazioni nazionali.
    
\end{itemize}

\section{Mutual recognition of eID and cross-border access}
Lista dei 7 servizi chiave:

\begin{itemize}
    \item Dichiarare l'imposta
    \item Controllo del casellario giudiziario
    \item Richiedere o convertire la patente di guida
    \item Domanda di pensione
    \item Ottenere il certificato di residenza
    \item Accedere ai servizi di previdenza sociale
    \item Domande universitarie
\end{itemize}

Nei Paesi membri dell'Unione Europea, solo il 14\% dei fornitori dei principali sette servizi ha permesso l'autenticazione transfrontaliera tramite eIDAS/eID, mentre il 44\% ha consentito l'accesso solo tramite un eID nazionale. In sostanza, la maggioranza dei fornitori di servizi pubblici chiave non offre l'autenticazione eIDAS agli utenti che si spostano tra i confini nazionali. Questo indica che il quadro normativo eIDAS non ha efficacemente implementato il riconoscimento reciproco dell'identificazione elettronica e l'accesso ai servizi pubblici transfrontalieri. Di conseguenza, i cittadini europei incontrano molteplici ostacoli nell'utilizzare i loro schemi di identità elettronica oltre i confini nazionali.
\section{Idee chiave e conclusioni della valutazione}
\begin{itemize}
    \item È stato notificato solo un numero limitato di eID, limitando la copertura del sistema di eID notificato
    a circa il 59\% della popolazione dell'UE.
    \item L'accettazione degli eID notificati sia a livello di Stati membri che di fornitori di servizi è limitata: non tutti i nodi eIDAS sono attivi e funzionanti e un numero limitato di servizi pubblici offre l'autenticazione eIDAS.
    \item L'interoperabilità di una serie di sistemi di identificazione elettronica è stata raggiunta a livello dell'UE.
    \item Incentivi limitati per gli Stati membri e i fornitori di servizi per la connessione all'infrastruttura eIDAS.
    \item Mancanza di obblighi di monitoraggio e comunicazione che limitino l'accesso a dati affidabili sulle connessioni attive e l'utilizzo degli eID notificati.
    \item L'effettivo uso cross-border degli eID è molto limitato, ma l'evoluzione del numero di
    transazioni in alcuni Stati membri conferma la tendenza all'aumento dell'uso di eID notificati
    regimi da settembre 2018.
    \item Mancanza di conoscenza di eIDAS tra i cittadini e utilizzo di eID notificati da parte di fornitori di servizi privati.
    \item Gli eID basati su eIDAS non sono stati in grado di espandersi sufficientemente nel settore privato.
    \item Il modello di governance degli eID è complesso: manca l'armonizzazione della certificazione, la revisione della
    procedure di notifica e peer review, chiarimenti sui requisiti di sicurezza, gli strumenti e le procedure per gestire gli incidenti relativi all'identificazione elettronica.
    
\end{itemize}