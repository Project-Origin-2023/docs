\section{Problema generale}
Gli Stati membri hanno utilizzato diverse soluzioni tecnologiche, il che ha comportato una mancanza di interoperabilità cross-border e cross-sector, la mancanza di un quadro giuridico comune per determinare l'affidabilità dell'entità che rilascia l'eID, la mancanza di certezza del diritto a livello trasversale - uso cross-border degli eID e mancanza di regole di responsabilità, per quanto riguarda la correttezza dell'identità.

\section{eSignatures}
Prima dell'adozione del regolamento eIDAS, il quadro normativo europeo in questo settore copriva solo le firme elettroniche (firme elettroniche) e i dispositivi per la creazione di firme elettroniche disciplinati dalla direttiva sulle firme elettroniche.

\begin{itemize}
    \item Problema dei costi elevati: i prestatori di servizi che desideravano offrire i propri servizi in un altro Stato membro hanno dovuto affrontare costi elevati dovuti a requisiti tecnici variabili e alla necessità di conformarsi al paese di destinazione.
    \item Problema degli standard obsoleti: si trattava di un altro problema non affrontato nella direttiva sulle firme elettroniche. La direttiva non ha stabilito norme comuni per le firme elettroniche, dando luogo a un mosaico di legislazioni nazionali.
    
\end{itemize}

\section{Mutual recognition of eID and cross-border access}
Lista dei 7 servizi chiavi:

\begin{itemize}
    \item Dichiarare l'imposta
    \item Controllo del casellario giudiziario
    \item Richiedere o convertire la patente di giuda
    \item Domanda di pensione
    \item Ottenere il certificato di residenza
    \item Accedere ai servizi di previdenza sociale
    \item Domande universitarie
\end{itemize}

In tutti gli Stati membri dell'UE, solo il 14\% dei fornitori di questi 7 servizi chiave ha consentito l'autenticazione transfrontaliera con eIDAS/eID, mentre il 44\% dei fornitori ha consentito l'accesso solo tramite un eID nazionale. In conclusione, la grande maggioranza degli i fornitori di sette servizi pubblici chiave non offrono l'autenticazione eIDAS agli utenti transfrontalieri. Ciò suggerisce la conclusione che il quadro eIDAS non è stato in grado di attuare efficacemente il riconoscimento reciproco dell'identificazione elettronica e l'accesso transfrontaliero ai servizi pubblici e che i cittadini europei si trovano di fronte a molteplici ostacoli per utilizzare i loro schemi di identità elettronica notificati oltre i confini.

\section{Key conclusions of th eidas evalutation}
\begin{itemize}
    \item È stato notificato solo un numero limitato di eID, limitando la copertura del sistema di eID notificato
    a circa il 59\% della popolazione dell'UE
    \item L'accettazione degli eID notificati sia a livello di Stati membri che di fornitori di servizi è limitata: non tutti i nodi eIDAS sono attivi e funzionanti e un numero limitato di servizi pubblici offre l'autenticazione eIDAS
    \item L'interoperabilità di una serie di sistemi di identificazione elettronica è stata raggiunta a livello dell'UE
    \item Incentivi limitati per gli Stati membri e i fornitori di servizi per la connessione all'infrastruttura eIDAS
    \item Mancanza di obblighi di monitoraggio e comunicazione che limitino l'accesso a dati affidabili sulle connessioni attive e l'utilizzo degli eID notificati
    \item L'effettivo uso cross-border degli eID è molto limitato, ma l'evoluzione del numero di
    transazioni in alcuni Stati membri conferma la tendenza all'aumento dell'uso di eID notificati
    regimi da settembre 2018
    \item Mancanza di conoscenza di eIDAS tra i cittadini e utilizzo di eID notificati da parte di fornitori di servizi privati
    \item Gli eID basati su eIDAS non sono stati in grado di espandersi sufficientemente nel settore privato
    \item Il modello di governance degli eID è complesso: manca l'armonizzazione della certificazione, la revisione della
    procedure di notifica e peer review, chiarimenti sui requisiti di sicurezza, gli strumenti e le procedure per gestire gli incidenti relativi all'identificazione elettronica
    
\end{itemize}