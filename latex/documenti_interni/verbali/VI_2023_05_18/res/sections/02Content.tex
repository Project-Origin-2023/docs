\section{Ordine del giorno}
\begin{enumerate}
\item Licenza
\item Metodo Agile
\item Discussione documenti tecnici
\item Coordinamento interno 
\item Stesura \textit{Diario di Bordo}
\end{enumerate}

\subsection{Licenza}
Il gruppo, dopo essersi informato, ha discusso le diverse licenze che si potrebbero apportare al codice sorgente prodotto.\\
Le due scelte principali sono state la licenza GPL e MIT, è stata realizzata una piccola presetazione PTT che ne ha mostrato i vantaggi e gli svantaggi di ciascuna.\\
Dopo una breve discussione e una votazione il gruppo ha scelto la licenza GPLv3. 
Questa scelta è stata fatta nonostante la licenza GPL abbia alcune limitazioni, tuttavia ai fini del progetto non dovrebbero creare problemi
di incompatibilità con altri software, non è previsto in futuro il brevetto del codice prodotto ed in più è una licenza che permette una tutela maggiore del proprio
codice rispetto al MIT.

\subsection{Metodo Agile}
In vista del RTB il gruppo ha discusso quale metodo per la gestione del progetto utilizzare.\\
Il gruppo ha ribadito il voler utilizzare il metodo Agile ed ha discusso alcuni suoi dettagli.\\
In particolare sono stati discussi i suddetti modelli:
\begin{enumerate}
    \item Scrum
    \item Kanban
    \item Lean
    \item Extreme Programming (XP)
\end{enumerate}
In sintesi, Scrum è focalizzato sulla consegna iterativa di funzionalità, 
Kanban sul flusso continuo di lavoro, 
Lean sull'eliminazione degli sprechi e l'ottimizzazione del valore e XP sulla comunicazione, 
la collaborazione e la qualità del software.\\
Ognuno di questi metodi agili offre un set specifico di strumenti e pratiche per gestire i progetti in modo flessibile, efficiente e orientato al valore.\\
Abbiamo valutato queste diverse opzioni e la decisione finale è stata rimandata alla prossima riunione.\\
Il gruppo necessita ancora di tempo per capire quale è l'approccio migliore da utilizzare per questo progetto.

\subsection{Discussione documenti tecnici}
Durante le scorse settimane alcuni membri del gruppo hanno redatto alcuni documenti categorizzabili come documenti tecnici.\\
Alcuni membri hanno esposto al gruppo il contenuto e lo hanno spiegato in quanto è strettamente schematico e riassuntivo.\\
È stato assegnato a tutti i membri del gruppo il compito per il fine settimana di leggere con attenzione i documenti prodotti ed eventualmente confrontarsi con i concetti meno chiari.\\
I documenti da analizzare sono i seguenti in ordine di importanza
\begin{enumerate}
\item ricercaOpenID: Documento molto importante in quanto descrive brevemente uno standard obbligatorio da utilizzare per i Requisiti Non Funzionali del capitolato.
\item architetturaEUDI: Documento da analizzare in quanto comprende le linee guida della UE per il progetto EUDI, da cui è stato inspirato il capitolato.
\item W3CDataModel: Documento da analizzare in quanto descrive uno standard da utilizzare per i Requisiti Non Funzionali del capitolato.
\item valutazioneEuropeaEID: Documento da leggere brevemente in quanto descrive solo la situazione attuale del utilizzo di eID nella UE, serve solo a capire meglio il dominio del problema.
\item ricercaRequisitiSPID: solo da leggere brevemente in quanto contiene delle informazioni non implementabili a pieno nel nostro progetto.
\end{enumerate}
Dopo la riunione tutti i membri del gruppo sanno brevemente il contenuto di questi documenti, ed entro Domenica 21-05-2023 tutti dovranno studiarne il contenuto.

\subsection{Coordinamento interno}
Si è fatto il punto della situazione sul lavoro effetuato durante la settimana corrente ed il gruppo si è sincronizzato e allineato; il responsabile inoltre ha suddiviso i ruoli e le mansioni da svolgere la settimana seguente. 

\subsection{Stesura \textit{Diario di Bordo}}
il Team ha steso il \textit{Diario di Bordo} della settimana corrente da presentare al Professor Vardanega.