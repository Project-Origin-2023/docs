\section{Ordine del giorno}
\begin{enumerate}
\item Casi d'uso
\item Analisi dei requisiti
\end{enumerate}

\subsection{Casi d'uso}
Durante la riunione con il Professor Cardin, una delle prime domande riguardava i casi d'uso in cui sono stati chiariti i ruoli dell'attore primario e secondario.
Vengono definiti i ruoli degli attori primari e secondari e si discute dell'identificazione dell'issuer come sistema o essere umano, 
influenzando la rappresentazione dei casi d'uso.
L'attore secondario è un sistema esterno che non interagisce attivamente con il sistema principale, ma fornisce assistenza o viene sollecitato dal sistema. 
Rimane esterno al sistema e non può modificarne le caratteristiche. Il suo ruolo principale è di supportare l'attore primario nell'adempimento delle ù
sue funzioni e si sottolinea la necessità di trovare un nome più appropriato per l'attore secondario.
\\In particolare sono stati esaminati UC01, UC02, UC03 e UC04 in cui:
\begin{itemize}
  \item L'issuer viene identificato come una piattaforma nei casi d'uso UC01 e UC03, indicando una rappresentazione come applicazione nel diagramma dei casi d'uso. 
  \item Nel caso d'uso UC02, l'issuer è un essere umano, richiedendo una rappresentazione come persona nel diagramma dei casi d'uso.
  \item Per quanto riguarda UC04, si consiglia di espandere gli esempi dei casi d'uso, introducendo ulteriori sotto funzionalità come UC04.1, al fine di fornire una comprensione più dettagliata delle azioni e delle responsabilità associate a ciascuna funzionalità.
\end{itemize}

\subsection{Analisi dei requisiti}
L'analisi dei requisiti attuale, che occupa solo 9 pagine, viene considerata insufficiente. Si raccomanda di espandere l'analisi dei requisiti per fornire una copertura più completa delle necessità del sistema.
\\ Un'analisi dettagliata consentirà di identificare e documentare in modo esauriente tutti i requisiti funzionali e non funzionali, stabilendo una solida base per lo sviluppo del sistema.
È importante dedicare l'attenzione necessaria all'analisi e all'espansione dei requisiti per identificarli, analizzarli e documentarli correttamente, fornendo una visione chiara e completa del sistema proposto.
