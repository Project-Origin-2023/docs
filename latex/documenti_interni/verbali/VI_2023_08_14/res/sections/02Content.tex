\section{Ordine del giorno}
\begin{enumerate}
\item Discussione commento RTB;
\item Prossimo incontro proponente;
\item Walt id.%qui metto le cose di Andrei con walt id
\end{enumerate}
%da aggiungere a glossario: UI, branch, postman
\subsection{Discussione commento RTB}
All'inizio della riunione si sono analizzati alcuni punti del commento, in particolare la sezione sulla documentazione.
\subsection{Prossimo incontro proponente}
Successivamente si è deciso di scrivere al proponente per comunicare l'esito della revisione e per chiedere un incontro per chiarire alcuni dubbi. %qui riscrivo meglio
\subsection{Walt id}

\begin{itemize}
    \item Per quanto riguarda walt id, si è deciso di utilizzare il wallet kit al posto dell'SSI Kit, perché è una sovrastruttura dell'SSI Kit e fornisce delle API ad alto livello per l'interazione con i frontend e i protocollo di scambio delle credenziali.
    \item Al momento le nuove funzionalità sono state implementate in un branch separato, ma verranno integrate nel branch principale una volta che saranno state testate.
    \item Nel frattempo la demo\footnote[1]{\url{https://wallet.walt.id/}} di wallet creata da Waltid è stata aggiornata.
    \item Un problema che subito si è presentato è che l'architettura di questo componente è stata pensata per avere un'unica implementazione, dove le UI di wallet, issuer e verifier fanno riferimento alle stesse API. Per fortuna si è visto che è possibile avere deploy separati.
    \item Per quanto riguarda l'issuing di credenziali, si è deciso di utilizzare Open ID Connect, che ha un flusso di issuing di credenziali diverso rispetto a quello di SSI, perché la richiesta parte dal wallet.
\end{itemize}



In seguito è stato mostrato il completo flusso di funzionamento delle varie richieste con postman. Una cosa interessante è il fatto che la configurazione dei container degli issuer viene fatta con un'api apposita, configurata per avere multi tenancy (quindi più issuer, anche se a noi ne serve uno solo), e non a mano, altrimenti non funziona.
Una criticità che è stata risolta è dovuta al fatto che, quando dal wallet si autorizza l'emissione di una credenziale, e il backend del wallet deve comunicare tramite POST una SIOPv2 response to redirecturi, se ho entrambi gli attori hostati su localhost, su porte diverse, un'istanza non è in grado di comunicare con le altre.
Questo problema è stato risolto con nginx proxy manager, che fa da intermediario tra gli attori, e associare degli url a localhost modificando il file hosts.