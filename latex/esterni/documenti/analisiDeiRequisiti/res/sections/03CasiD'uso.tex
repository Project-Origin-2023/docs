\section{Casi d'Uso}
\subsection{Introduzione}
In questa sezione sono presentati i casi d'uso che risultano rilevanti per il prodotto Personal Identity Wallet. 
Essi sono stati individuati e definiti attraverso l'analisi del capitolato d'appalto, gli incontri con il proponente e le riunioni interne del team Project Origin.
 Ciascun caso d'uso rappresenta un insieme di scenari che hanno lo stesso obiettivo finale per un utente generico del sistema, definito attore.
 Le norme e le convenzioni adottate per la stesura di ogni caso d'uso sono descritte in dettaglio all'interno del documento Norme di Progetto.
 \subsection{Codice identificativo}
Ciascun caso d'uso viene categorizzato utilizzando la seguente notazione:\\ 
\begin{center}\begin{verbatim}
    CU{XX}.{YY}
\end{verbatim}\end{center}
Ogni caso d’uso è inoltre definito secondo la seguente struttura:
\begin{itemize}
    \item ID: il codice del caso d’uso secondo la convenzione specificata precedentemente;
    \item Nome:  specifica il titolo del caso d’uso;
    \item Attori:  indica gli attori principali (ad esempio l’utente generico) e secondari (ad esempio entità di autenticazione esterne) del caso d’uso;
    \item Descrizione:  riporta una breve descrizione del caso d’uso;
    \item Precondizione:  specifica le condizioni che sono identificate come vere prima del verificarsi degli eventi del caso d’uso;
    \item Postcondizione:  specifica  le  condizioni  che  sono  identificate  come  vere  dopo  il verificarsi degli eventi del caso d’uso;
    \item Scenario  principale:  rappresenta  il  flusso  degli  eventi,  a  volte  attraverso  l'uso di  una  lista  numerata,  specificando  per  ciascun  evento:  titolo,  descrizione,  attori coinvolti e casi d’uso generati;
    \item Inclusioni:  usate per non descrivere più volte lo stesso flusso di eventi, inserendo il comportamento comune in un caso d’uso a parte;
    \item   Estensioni:  descrivono i casi d’uso che non fanno parte del flusso principale degli eventi, allo stesso modo di quanto descritto in “Scenario principale”.
\end{itemize}
Alcuni  casi  d’uso  possono  essere  associati  ad  un Diagramma UML  dei  casi  d'uso riportante lo stesso titolo e codice.
 \subsection{Attori}


 \subsection{Elenco dei casi d'uso}