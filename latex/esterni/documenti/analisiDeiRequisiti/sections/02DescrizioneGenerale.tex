\section{Descrizione generale}

\subsection{Obiettivo del prodotto}
L’obiettivo del prodotto è quello di permettere all’utilizzatore dell’applicativo (Holder) di raccogliere le proprie credenziali dall’istituzione interessata (Issuer) e di 
memorizzarle nel loro portafoglio identità. Successivamente le credenziali appena create verranno verificate dalle entità interessate (Verifier) per permettere l’accesso 
all'Holder all’area interessata. Il Verifier verifica le credenziali d’accesso tramite un’infrastruttura chiamata Verifiable Data Registry.

\subsection{Funzioni del prodotto}
Per quanto riguardano le credenziali d’accesso, dovrà essere possibile: 
\begin{itemize}
    \item Richiedere una credenziale d’accesso;
    \item Creare e consegnare la credenziale.
\end{itemize}
Per quanto riguarda l’amministrazione delle credenziali, dovrà essere possibile: 
\begin{itemize}
    \item Vedere le credenziali;
    \item Eliminare le credenziali.
\end{itemize}
Per quanto riguarda il Verifier: 
\begin{itemize}
    \item Dovrà richiedere la credenziale;
    \item L'holder (l’utilizzatore dell’applicativo) dovrà essere capace di consegnare le credenziali appena richieste dal Verifier;
    \item Il Verifier dovrà validare le credenziali e permettere l’accesso all’utente.
\end{itemize}

\subsection{Caratteristiche degli utenti}
L’applicativo potrà essere utilizzato da ogni Holder. \\
L'Holder potrebbe essere (ma non solo): 
\begin{itemize}
    \item Un’amministrazione pubblica (centrale o locale);
    \item Un cittadino italiano maggiorenne, oppure un cittadino estero con codice fiscale italiano;
    \item Un’impresa o un’organizzazione (pubblica o privata);
    \item Un professionista (avvocato, commercialista, notaio, ecc.);
    \item Un’università o un centro di ricerca;
    \item Un’associazione o un’organizzazione no profit;
    \item Un servizio di pubblica utilità (acqua, gas, energia elettrico, ecc.), finanziario (banca, ecc.), sanitario (Fascicolo Sanitario Elettronico, ecc.), 
di trasporto pubblico (Trenitalia, ecc.).
\end{itemize}
