\documentclass[a4paper]{article}
\usepackage[normalem]{ulem}
\usepackage{eurosym}
\usepackage[font=small,labelfont=bf]{caption}

% impostazioni generali
\input{../../../template/package.tex}

% dati della prima pagina
% Configurazione della pagina iniziale
\newcommand{\doctitle}{\textit{Verbale} interno del 11 maggio 2023}
\newcommand{\rev}{0.1.0} % versione
\newcommand{\resp}{Beschin Michele} % inserire responsabile
\newcommand{\red}{Andreetto Alessio} % cognome nome dei redattori
\newcommand{\ver}{Lotto Riccardo} % cognome nome del verificatore
\newcommand{\uso}{Interno} %interno o esterno
\newcommand{\dest}{\textit{ProjectOrigin}
	\\ Prof. Vardanega Tullio 
	\\ Prof. Cardin Riccardo}
\newcommand{\describedoc}{\textit{Verbale} riguardante il meeting tenuto il 11 maggio 2023}

 % per modificare la prima pagina editare questo file

\makeindex

\makeatletter
\renewcommand\paragraph{
\@startsection {paragraph}{4}{0mm}{-\baselineskip}{.5\baselineskip}{\normalfont \normalsize \bfseries }}
\makeatother

\begin{document}

% Prima pagina
\thispagestyle{empty}
\renewcommand{\arraystretch}{1.3}

\begin{titlepage}
	\begin{center}
		
	\includegraphics[scale = 0.20]{../../../template/images/logo.png}
	\\[1cm]
	\href{mailto:projectorigin2023@gmail.com}		      	
	{\large{\textit{projectorigin2023@gmail.com} } }\\[2.5cm]
	\Huge \textbf{\doctitle} \\[1cm]
	 \large
			 \begin{tabular}{r|l}
                        \textbf{Versione} & \rev{} \\
                        \textbf{Responsabile} & \resp{} \\
                        \textbf{Redattori} & \red{} \\ 
                        \textbf{Verificatori} &  \ver{} \\
                        \textbf{Uso} & \uso{} \\                        
                        \textbf{Destinatari} & \parbox[t]{5cm}{ \dest{} }
                \end{tabular} 
                \\[3.3cm]
                \large \textbf{Descrizione} \\ \describedoc{} 
     \end{center}
\end{titlepage}

% Diario delle modifiche
\input{../../../template/changelog.tex}
\changelogTable{
0.3.0 & 2023-05-10 & Corbu Teodor & Verificatore & Verifica documento \\
0.2.1 & 2023-05-10 & Ibra Elton & Analista & Stesura sottocapitoli dei \S\ Casi d'uso \\
0.2.0 & 2023-05-09 & Corbu Teodor & Verificatore & Verifica documento \\
0.1.1 & 2023-05-09 & Ibra Elton & Analista & Inizio stesura \S\ Casi d'uso \\  
0.1.0 & 2023-05-04 & Ibra Elton & Verificatore & Verifica documento\\    
0.0.3 & 2023-05-03 & Corbu Teodor & Analista & Stesura \S\ Descrizione Generale \\
0.0.2 & 2023-05-03 & Corbu Teodor & Analista & Stesura \S\ Introduzione \\
0.0.1 & 2023-05-02 & Corbu Teodor & Analista & Creazione struttura documento\\
} % editare questo
\pagebreak


% Indice
{
    \hypersetup{linkcolor=black}
    \tableofcontents
    %\listoffigures %elenco figure
    %\listoftables  %elenco tabelle
}
\pagebreak

% sezioni comuni 
%si può aggiungere una sezione dei riferimenti, dove si citano il capitolato, wikipedia, treccani e i libri di SWE
\section{A}
\textbf{Analista}: Persona che conosce il dominio del problema e definisce i requisiti espliciti e impliciti. Si occupa di redigere il documento Analisi dei Requisiti.\\
\textbf{Asincrona}: Modalità di svolgimento delle attività in cui i partecipanti si scambiano informazioni senza dover comunicare in tempo reale.\\
\textbf{Attore}: L'attore rappresenta un ruolo o una responsabilità in un determinato scenario di utilizzo di un sistema software e viene identificato durante l'analisi dei requisiti e nella modellizzazione dei casi d'uso. L'attore può essere sia principale, ovvero l'attore principale che utilizza il sistema, sia secondario, ovvero un attore che supporta l'attore principale nell'utilizzo del sistema o fornisce informazioni o servizi al sistema stesso.\\

\pagebreak
\section{D}
\textbf{Deflate}: Deflate (stilizzato come DEFLATE) è un algoritmo per la compressione dati senza perdita che è stato introdotto dal programma PKZIP, e quindi formalizzato nella RFC 1951. È tuttora ampiamente utilizzato per le sue ottime prestazioni e l'assenza di brevetti.
\textbf{Discord}: Piattaforma gratuita che fornisce servizi di chat vocale, testuale e video tra singoli membri o in server dedicati.\\

\pagebreak
\section*{K}
\textbf{Keep (Google)}: È un servizio di Google per prendere annotazioni.
\pagebreak
\section{P}
\textbf{Processo}: Insieme delle attività correlate e coese che trasformano i bisogni in prodotti (il risultato di un processo si
chiama prodotto). Opera secondo regole consumando risorse.\\
\textbf{Progettista}: Si occupa di definire l'architettura del sistema alla base del prodotto software. Segue la fase dello sviluppo del prodotto.
\pagebreak
\section{R}
\textbf{Repository}: Archivio centralizzato dove vengono memorizzate le informazioni e i dati in formato digitale sulla base di metadati che ne permettono la rapida individuazione.\\
\textbf{Responsabile}: Ha il compito di pianificare le attività, coordinare e controllare tutti i membri del team.
Si occupa anche di approvare i documenti e rappresenta il team presso l'azienda proponente.\\

\pagebreak
\section{T}
\textbf{Teams (Microsoft)}: Piattaforma di comunicazione e collaborazione unificata che combina chat di lavoro persistente, teleconferenza, condivisione di contenuti.\\
\textbf{Telegram}: Servizio di messaggistica istantanea e broadcasting basato su cloud.\\
\textbf{TLS}: Transport Layer Security (TLS) e il suo predecessore Secure Sockets Layer (SSL) sono dei protocolli crittografici di presentazione usati nel campo delle telecomunicazioni e
dell'informatica che permettono una comunicazione sicura dalla sorgente al destinatario (end-to-end) su reti TCP/IP (come ad esempio Internet) fornendo autenticazione, 
integrità dei dati e confidenzialità operando al di sopra del livello di trasporto.

\pagebreak
\section{V}
\textbf{Verifica}: Accertamento che l’esecuzione delle attività di processi svolti nella fase in esame non causino errori.\\
\textbf{Verificatore}: È presente per l'intera durata del progetto e si occupa di svolgere le attività di Verifica e Validazione.\\
\textbf{Verifier}: Entità interessate a consumare credenziali (ad esempio, una banca online che chiede le credenziali di registrazione universitaria per offrire un conto studente). Le credenziali fornite a un verificatore da un titolare possono essere confezionate in "presentazioni verificabili".\\
\pagebreak
\section{W}
\textbf{Wallet}: Applicazione in cui l'utente memorizza credenziali che può utilizzare presso dei provider.\\ %forse servirà anche la definizione di provider
\textbf{Web App}: Applicazione fruibile via web per mezzo di un network, come Internet, che offre determinati servizi all'utente. Una web app non necessita di essere installata.\\
\textbf{W3C Data Model}: Il "W3C Data Model" è un modello di dati definito dal World Wide Web Consortium (W3C), un'organizzazione che sviluppa standard per il World Wide Web. Il modello di dati del W3C è progettato per rappresentare informazioni strutturate in modo interoperabile e standardizzato, consentendo la condivisione e lo scambio di dati tra diverse applicazioni e piattaforme web.\\
\textbf{World Wide Web Consortium (W3C)}: Il World Wide Web Consortium (W3C) è un'organizzazione internazionale che si occupa dello sviluppo di standard aperti per il World Wide Web. Fondata nel 1994 da Tim Berners-Lee, il creatore del Web, e con sede presso il Massachusetts Institute of Technology (MIT) negli Stati Uniti, il W3C lavora per stabilire linee guida e specifiche tecniche che promuovono l'interoperabilità, l'accessibilità e l'evoluzione del Web.\\
\textbf{Web of Things (WoT)}: Web of Things (WoT) è un concetto che si riferisce alla connessione e all'interoperabilità delle cose fisiche e degli oggetti intelligenti tramite il Web. WoT mira a estendere i principi del World Wide Web per consentire alle cose di essere integrate nel contesto del Web e di comunicare tra di loro utilizzando standard aperti e protocolli web. L'obiettivo principale di WoT è consentire agli oggetti di essere facilmente accessibili e controllabili attraverso il Web, in modo simile a come le pagine web sono accessibili agli utenti.

\end{document}
