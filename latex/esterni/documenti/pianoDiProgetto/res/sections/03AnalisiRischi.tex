\section{Analisi di rischi}

Durante lo svolgimento dell’intero progetto vi è la possibilità di incontrare problematiche di varia tipologia e di varia rilevanza. 
Esse potrebbero essere causate da insieme di persone o cose che potrebbero portare al rallentamento dello sviluppo del progetto. 
Pertanto viene effettuata un'attenta analisi dei fattori di rischio. 

\subsection{Classificazione}
Al fine di tenere monitorato e permetterne il tracciamento ogni  rischio tracciato viene identificato tramite la seguente codifica:
\begin{itemize}
    \item \textbf{Codice identificativo}: la descrizione della codifica viene riportata nella sezione x.x.x del documento \textit{Norme di Progetto} v x.x.x (da modificare)
    \item \textbf{Titolo}
    \item \textbf{Descrizione}
    \item \textbf{Precauzioni}: definisce la modalità con cui vengono gestite le problematiche al fine di mitigarle nel caso in cui si presentino
    \item \textbf{Probabilità}
    \item \textbf{Rilevanza}: gravità del rischio:
    \begin{itemize}
        \item \textbf{Accettabile}: entità minore, lieve sostenibile senza risoluzione
        \item \textbf{Tollerabile}: entità modesta, la risoluzione che richiede non è urgente
        \item \textbf{Inaccettabile}: entità ggrande, danno grave e la risoluzione deve avvenire il prima possibile
    \end{itemize}
\end{itemize}

\subsection{Lista dei rischi}

\subsubsection{RT1 - Rischio Tecnologico 1}
\begin{tabular}{|p {0.2\textwidth}|p {0.75\textwidth}|}  
    \hline
    \textbf{Titilo} & Inesperienza tecnologica \\
    \hline
    \textbf{Descrizione} & Lo svolgimento del progetto richiede tecnologie che possono risultare nuove ad alcuni componenti del gruppo, portando a difficoltà nell’utilizzo. \\
    \hline
    \textbf{Precauzioni} & Fissare la scadenza tenendo conto di tale rischio e suddividere tali compiti su diversi componenti del gruppo al fine di 
        stimolare l’auto-apprendimento reciproco nonché documentarsi in autonomia su tali tecnologie. \\
    \hline
    \textbf{Probabilità} & Alta \\
    \hline
    \textbf{Rilevanza} & Inacettabile \\
    \hline
\end{tabular}

\subsubsection{RO1 - Rischio Organizzativo 1}
\begin{tabular}{|p {0.2\textwidth}|p {0.75\textwidth}|}  
    \hline
    \textbf{Titilo} & Calcolo tempistiche \\
    \hline
    \textbf{Descrizione} & La scarsa esperienza del team può portare a imprecisioni riguardo la valutazione in termini di tempo di determinate attività \\
    \hline
    \textbf{Precauzioni} & Ogni componente si impegna a tener traccia delle proprie ore produttive \\
    \hline
    \textbf{Probabilità} & Media \\
    \hline
    \textbf{Rilevanza} & Inacettabile \\
    \hline
\end{tabular}

\subsubsection{RO2 - Rischio Organizzativo 2}
\begin{tabular}{|p {0.2\textwidth}|p {0.75\textwidth}|}  
    \hline
    \textbf{Titilo} & Impegni personali \\
    \hline
    \textbf{Descrizione} & Possono verificarsi situazioni in cui , per motivi personali, 
        i componenti del gruppo non sono disponibili per procedere con le varie attività del progetto. \\
    \hline
    \textbf{Precauzioni} & Fondamentale è l'assegnazione degli incarichi ai vari componenti del team nonché 
        la precisione nella comunicazione di tali impegni. Possibilità di riassegnazione delle attività 
        da svolgere ad altri componenti in caso di imprevisti. \\
    \hline
    \textbf{Probabilità} & Alta \\
    \hline
    \textbf{Rilevanza} & Inacettabile \\
    \hline
\end{tabular}