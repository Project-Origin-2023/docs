\section{Qualità di Processo}
Per garantire la qualità dei processi il gruppo utilizza come riferimento lo standard \textbf{ISO/IEC 12207:1995}.
Dopo aver studiato i vari processi, si sono scelti quelli più adatti alle necessità del progetto.
\subsection{Processi primari}


\begin{longtable}{ 
    >{\centering}M{0.20\textwidth} 
    >{\centering}M{0.50\textwidth}
    >{\centering}M{0.17\textwidth} 
    }
\rowcolorhead
\headertitle{Processo} &
\centering \headertitle{Descrizione} &	
\headertitle{Metriche} 
\endfirsthead
\endhead

Fornitura & Scelta delle procedure e delle risorse necessarie per garantire che i servizi siano forniti in modo tempestivo & - \tabularnewline
Sviluppo & Realizzazione di un prodotto software di qualità, che soddisfi le esigenze del cliente& - \tabularnewline
\end{longtable}

\subsection{Processi di supporto}
\begin{longtable}{ 
    >{\centering}M{0.20\textwidth} 
    >{\centering}M{0.50\textwidth}
    >{\centering}M{0.17\textwidth} 
    }
\rowcolorhead
\headertitle{Processo} &
\centering \headertitle{Descrizione} &	
\headertitle{Metriche} 
\endfirsthead
\endhead

Verifica & Si determina se i requisiti del prodotto sono soddisfatti& - \tabularnewline
Gestione della qualità & Viene garantita la conformità dei processi e la conformità con gli standard prefissati& - \tabularnewline
Documentazione & Controllo della leggibilità della documentazione prodotta& - \tabularnewline %per questo usiamo l'indice di gulpease
\end{longtable}
\subsection{Processi organizzativi}
\begin{longtable}{ 
    >{\centering}M{0.20\textwidth} 
    >{\centering}M{0.50\textwidth}
    >{\centering}M{0.17\textwidth} 
    }
\rowcolorhead
\headertitle{Processo} &
\centering \headertitle{Descrizione} &	
\headertitle{Metriche} 
\endfirsthead
\endhead

Gestione organizzativa & Controllo e organizzazione delle prestazioni di un processo& - \tabularnewline
\end{longtable}