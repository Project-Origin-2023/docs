\section{Ordine del giorno}
\begin{enumerate}
\item Discussione in dettaglio del capitolato
\item Analisi dei requisiti non funzionali
\item Presentazione di un esempio
\item Chiarimenti su usabilità e testing
\item Chiarimenti sulle differenze con la prima versione e tempistiche di sviluppo
\end{enumerate}

\subsection{Discussione in dettaglio del capitolato}
Durante la riunione con l’azienda proponente del progetto, si è discusso in dettaglio il
capitolato, con l’obiettivo di fornire una maggiore comprensione del concetto di "verifiable
credential", ovvero certificati digitali utilizzati per diverse informazioni quali la residenza,
lo status di studente, la vaccinazione, ecc. Tali credenziali sono rilasciate da diversi
emittenti (ad esempio, università, comuni e ospedali) e vengono conservate all’interno di
un wallet da parte del titolare. Il terzo attore coinvolto, ovvero il "verifyer", ha il compito
di verificare l’affidabilità delle credenziali.

\subsection{Analisi dei requisiti non funzionali}
Durante la riunione, sono stati inoltre evidenziati i requisiti non funzionali, tra cui l’utiliz-
zo di JSON WRC e OpenID Verified credential e verified presentation, senza la richiesta
di verifica delle credenziali e senza la preoccupazione della correttezza della firma.

\subsection{Presentazione di un esempio}
È stato presentato un esempio di demo wallet wallet.example.chapi.io. Si è suggerito
che per le diverse webapp si possa effettuare un’unica implementazione, ma con diversi
deploy e configurazioni al fine di evitare sovrapposizioni tra le parti che, essenzialmente,
sono uguali per quanto riguarda la comunicazione. Si è inoltre evidenziato che la creazione
di diverse implementazioni sarebbe utile anche dal punto di vista didattico. Inoltre, è stato
menzionato il prodotto di Infocert, dizme.io, che potrebbe essere utile come esempio per
il progetto.

\subsection{Chiarimenti su usabilità e testing}
Durante la riunione, si è sottolineato l’importanza di seguire i criteri di usabilità, senza
tuttavia perseguire linee guida precise che possano appesantire il lavoro. Inoltre, si è
chiarita la necessità di avere una modalità di test per il prototipo, che non richieda una
copertura totale ma solo una minima copertura funzionale. Per quanto riguarda la licenza
del progetto, è stata suggerita l’opzione open source per il prototipo.

\subsection{Chiarimenti sulle differenze con la prima versione e tempistiche di sviluppo}
Si sono inoltre discussi i cambiamenti rispetto al progetto precedente del primo lotto,
con la pubblicazione delle linee guida architetturali della commissione europea e le limi-
tazioni per quanto riguarda la creazione di webapp e non app. Infine, si sono definite le
tempistiche per il completamento del progetto, fissando la scadenza entro la fine di agosto.